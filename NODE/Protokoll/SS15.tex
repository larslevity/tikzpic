\documentclass[12pt,a4paper]{article}
\usepackage[utf8]{inputenc}
\usepackage[german]{babel}
\usepackage{amsmath}
\usepackage{amsfonts}
\usepackage{amssymb}
\usepackage{graphicx}
\usepackage{lmodern}
\author{***}
\title{Numerics of ordinary differential equations - oral exam SS15}

\begin{document}
\maketitle

\section{Einschrittverfahren}
\begin{itemize}
\item Kennen Sie ein einfaches Verfahren um eine DGL zu lösen?
\item Illustrieren Sie die Vorgehensweise des expl. Euler-Verfahrens.
\item Wir haben zwei unterschiedliche Arten von Fehlern besprochen. Wo kann man diese Fehler in Ihrer Skizze sehen?
\item Wie haben wir diese Fehler genannt? \textit{Lokaler und globaler Fehler}
\item Ist es denn ein gutes Verfahren?
\item Wie kann man Konvergenz nachweisen?
\item Was ist Stabilität?
\item Was ist Konsistenz?
\item Mit welchem Satz lässt sich Stabilität und Konsistenz in Konvergenz überführen?
\item Welcher Art von Fehler wird verwendet um Konsistenz nachzuweisen? Welche für Konvergenz?
\item Wie legt sich die Fehlerordung eines Verfahrens fest?
\item Wie lassen sie Verfahren höherer Ordung konstruieren?
\item Wie lassen sich die Koeffizienten des Butcher Schemas ermitteln?


\end{itemize}

\section{Mehrschrittverfahren}
\begin{itemize}
\item Wie sieht ein Mehrschrittverfahren allgemein aus?
\item Wie lässt sich hier Stabilität nachweisen?
\item Wie ist der Lösungsansatz für $y$, um auf die charakteristische Gleichung zu kommen? $y_{n+1} = \lambda y_n = \lambda \lambda^n y_0$
\item Wie sieht die Testgleichung aus?
\item Welche zwei Bedingungen müssen für $\lambda_i$ gelten, damit das Verfahren stabil ist?
\item Wäre ein Verfahren mit der char. Gleichung : $(\lambda -1)^2 = 0$ stabil ?
\item Wie lässt sich hier Konsistenz nachweisen?
\item Ein Verfahren der mit der nachgewiesenen Ordnung 5 liefert angewandt auf eine DGL nur noch Ergebnisse der Fehlerordnung 2. Woran könnte das liegen?



\end{itemize}

\end{document}