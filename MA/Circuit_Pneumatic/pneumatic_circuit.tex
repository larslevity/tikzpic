% Pneumatic Circuit

\documentclass[10pt]{standalone}
\input{../../tikzpic_packages.tex}



\newcommand{\pvalve}[2]{
\path (#1,#2)coordinate(4);
\draw (4)--++(0,-.2)++(-.1,0)--++(.2,0);
\draw (4)++(-.1,-.25)--($(4)+(.1,-.75)$)--++(-.2,0)--($(4)+(.1,-.25)$)--cycle;
\draw (4)++(0,-1)coordinate(1)--++(0,.2)++(-.1,0)--++(.2,0);
\draw (4)++(0,-.5)--++(-.2,0)++(0,-.15)rectangle++(-.3,.3)node[midway]{\textsf{S}};
}

\newcommand{\sens}{
\draw (sens)--++(.2,0)++(.25,0)circle(.25)++(-45:.2)coordinate(X);
\draw[arrow] (X)--++(90+45:.4);
}

\begin{document}
\begin{tikzpicture}[
scale=1,
spring/.style={decoration = {zigzag, segment length = 1mm, amplitude = 1.5mm}, decorate},
help/.style={very thin, lightgray},
arrow/.style={-triangle 45},
spy/.style={very thick, gray!20},
spyfill/.style={draw, very thick, gray!20, fill=gray!05}
]

%%%%%%%%%%%%%%%%% PROPORTIONAL %%%%%%%%%%%%%%%%

% Motor
\path (-1,-.5)coordinate(0);
\draw (0)--++(0,-.25)--++(270-30:.3)--++(0:.3)--++(90+30:.3)++(0,-0.75)circle(.75)++(0,-.75)--++(0,-.25);
\fill(0)++(0,-2)circle(.03);
\draw (0)++(0,-1)++(185:.75)--++(-.5,0)coordinate(X);
\draw (0)++(0,-1)++(175:.75)--++(-.5,0)--(X);
\path (0)++(.75,-1)node[right]{Compressor};

% Valves
\foreach[count=\i] \x in {0,1.5,3,4.5,6,7.5}{
	\pvalve{\x}{.5}
	\draw (0)--(1);
	\draw[fill] (4)--++(0,.2)coordinate(sens)circle(.03)--++(0,.5)circle(.03)node[above]{P\i};
	\sens
	\ifnum\i<6 
		\fill (1)circle(.03);
	\fi
	\path (1)coordinate(0);
}

% Nahaufnahme
\path (10,0)node[right, spyfill](Pic){\includegraphics[scale=1]{5_3_proportional_valve.pdf}};
\draw[spy] (1)++(-.7,.1)rectangle++(1,.8)++(0,-.4)--(Pic); 

%%%%%%%%%%%%%%%%% SEPERATOR %%%%%%%%%%%%%%%

\draw[help] (-2.3,2.9)node[above right,scale=3, align=left]{\textsf{Discrete} \\ \textsf{Channel}}node[below right, scale=3]{\textsf{Continous Channel}}--++(20.5,0);

%%%%%%%%%%%%%%%%% DISCRETE %%%%%%%%%%%%%%%%
\begin{scope}[yshift=5cm]
% Vacuum Pump
\path (15,1)coordinate(0);
\draw (0)--++(0,-.25)++(0,-0.75)coordinate(C)circle(.75)++(0,-.75)--++(0,-.25);
\draw ($(C)+(270-50:.75)$)--($(C)+(90+15:.75)$);
\draw ($(C)+(270+50:.75)$)--($(C)+(90-15:.75)$);
\draw (0)++(0,-1)++(185:.75)--++(-.5,0)coordinate(X);
\draw (0)++(0,-1)++(175:.75)--++(-.5,0)--(X);
\path (0)++(.75,-1)node[right]{Vacuum Pump};
\fill (0)circle(.03)++(0,-2)coordinate(0);

% Discrete Valves
\foreach[count=\i] \x in {13,11.5,10,8.5}{
	\pvalve{\x}{0}
	\draw (0)--(1);
	\pgfmathtruncatemacro{\j}{5-\i}
	\draw[fill] (4)--++(0,.7)circle(.03)node[above]{D\j};
	\ifnum\i<4 
		\fill (1)circle(.03);
	\fi
	\path (1)coordinate(0);
}

% Nahaufnahme Discrete Valve
\path (2,-.5)node[right, spyfill](Pic2){\includegraphics[scale=1]{3_2_discrete_valve.pdf}};
\draw[spy] (1)++(-.7,.1)rectangle++(1,.8)++(-1,-.4)--(Pic2); 
\end{scope}

\end{tikzpicture}
\end{document}