% Pneumatic Circuit

\documentclass[10pt]{standalone}

%% For sans serfen Schrift:
%\usepackage[onehalfspacing]{setspace}
%\usepackage{helvet}
%\renewcommand{\familydefault}{\sfdefault}


\usepackage[utf8]{inputenc}
\usepackage[german]{babel}
\usepackage[T1]{fontenc}
\usepackage{graphicx}
\usepackage{lmodern}

\usepackage{amsmath}
\usepackage{amsfonts}
\usepackage{amssymb}
\usepackage{gensymb}
\usepackage{bm}
\usepackage{ifthen}
\usepackage{array}
\usepackage{eurosym}

\usepackage{tikz}
\usetikzlibrary{calc,patterns,
                 decorations.pathmorphing,
                 decorations.markings,
                 decorations.pathreplacing}
\usetikzlibrary{trees,arrows, scopes}
\usetikzlibrary{shapes}
\usetikzlibrary{positioning}
\usetikzlibrary{fadings}
\usepackage{tikz-3dplot}
\usetikzlibrary{matrix,fit, backgrounds}


\usepackage{pgfplots}
\usepgfplotslibrary{groupplots}
\pgfplotsset{compat=newest}

                 
\newcommand*\circled[1]{\tikz[baseline=(char.base)]{
            \node[shape=circle,draw,inner sep=2pt,solid,fill=white] (char) {#1};}}
            
            
            
\usetikzlibrary{patterns}

\pgfdeclarepatternformonly[\LineSpace]{my north east lines}{\pgfqpoint{-1pt}{-1pt}}{\pgfqpoint{\LineSpace}{\LineSpace}}{\pgfqpoint{\LineSpace}{\LineSpace}}%
{
    \pgfsetlinewidth{0.4pt}
    \pgfpathmoveto{\pgfqpoint{0pt}{0pt}}
    \pgfpathlineto{\pgfqpoint{\LineSpace + 0.1pt}{\LineSpace + 0.1pt}}
    \pgfusepath{stroke}
}


\pgfdeclarepatternformonly[\LineSpace]{my north west lines}{\pgfqpoint{-1pt}{-1pt}}{\pgfqpoint{\LineSpace}{\LineSpace}}{\pgfqpoint{\LineSpace}{\LineSpace}}%
{
    \pgfsetlinewidth{0.4pt}
    \pgfpathmoveto{\pgfqpoint{0pt}{\LineSpace}}
    \pgfpathlineto{\pgfqpoint{\LineSpace + 0.1pt}{-0.1pt}}
    \pgfusepath{stroke}
}

\newdimen\LineSpace
\tikzset{
    line space/.code={\LineSpace=#1},
    line space=3pt
}

\newcommand\centerofmass{%
    \tikz[radius=0.4em] {%
        \fill (0,0) -- ++(0.4em,0) arc [start angle=0,end angle=90] -- ++(0,-0.8em) arc [start angle=270, end angle=180];%
        \draw (0,0) circle;%
    }%
}


\begin{document}
\begin{tikzpicture}[
spring/.style={decoration = {zigzag, segment length = 1mm, amplitude = 1.5mm}, decorate},
help/.style={very thin, lightgray},
arrow/.style={-triangle 45}
]

% States
\foreach[count=\i] \x in {-.5,.5}{
	\draw (\x-.5,.5)--++(1,0)coordinate[pos=.25](\i2)--++(0,-1)--++(-1,0)coordinate[pos=.25](\i3)coordinate[pos=.75](\i1)--cycle;
}
% State1
\draw (13)--++(0,.2)++(-.1,0)--++(.2,0);
\draw[arrow] (11)--(12);
%State2
\draw (21)--++(0,.2)++(-.1,0)--++(.2,0);
\draw[arrow] (22)--(23);

% In- and Outlets
\draw[fill] (22)--++(0,.3)circle(.025)node[right]{2}coordinate(2);
\draw[fill] (21)--++(0,-.3)circle(.025)node[left]{1}coordinate(1);
\draw[fill] (23)--++(0,-.3)circle(.025)node[right]{3}coordinate(3);

% Frame
\draw[spring](1,-.25)--++(.5,0);
\draw (-1,-.5)--++(-.5,0)coordinate[pos=.1](ur)
coordinate[pos=.9](ul)--++(0,.4)--++(.5,0)coordinate[pos=.4](ol)coordinate[pos=.6](or);
\draw (ul)--(ol);




%%%%%%% Schaltung
\draw (3)--++(0,-.2)--++(-.1,0)--++(0,-.4)coordinate[pos=.5](l)--++(0.2,0)coordinate[pos=.25](u)coordinate[pos=.75](uu)--++(0,.4)coordinate[pos=.25](rr)coordinate[pos=.75](r)--++(-.1,0);
\draw (l)--++(.1,0) (r)--++(-.1,0) (rr)--++(-.1,0);
\draw (u)--++(-45:.07)--(uu);





\end{tikzpicture}
\end{document}