% Pneumatic Circuit

\documentclass[10pt]{standalone}
\input{../../tikzpic_packages.tex}

\begin{document}
\begin{tikzpicture}[
spring/.style={decoration = {zigzag, segment length = 1mm, amplitude = 1.5mm}, decorate},
help/.style={very thin, lightgray},
arrow/.style={-triangle 45}
]

% States
\foreach[count=\i] \x in {-.5,.5}{
	\draw (\x-.5,.5)--++(1,0)coordinate[pos=.25](\i2)--++(0,-1)--++(-1,0)coordinate[pos=.25](\i3)coordinate[pos=.75](\i1)--cycle;
}
% State1
\draw (13)--++(0,.2)++(-.1,0)--++(.2,0);
\draw[arrow] (11)--(12);
%State2
\draw (21)--++(0,.2)++(-.1,0)--++(.2,0);
\draw[arrow] (22)--(23);

% In- and Outlets
\draw[fill] (22)--++(0,.3)circle(.025)node[right]{2}coordinate(2);
\draw[fill] (21)--++(0,-.3)circle(.025)node[left]{1}coordinate(1);
\draw[fill] (23)--++(0,-.3)circle(.025)node[right]{3}coordinate(3);

% Frame
\draw[spring](1,-.25)--++(.5,0);
\draw (-1,-.5)--++(-.5,0)coordinate[pos=.1](ur)
coordinate[pos=.9](ul)--++(0,.4)--++(.5,0)coordinate[pos=.4](ol)coordinate[pos=.6](or);
\draw (ul)--(ol);




%%%%%%% Schaltung
\draw (3)--++(0,-.2)--++(-.1,0)--++(0,-.4)coordinate[pos=.5](l)--++(0.2,0)coordinate[pos=.25](u)coordinate[pos=.75](uu)--++(0,.4)coordinate[pos=.25](rr)coordinate[pos=.75](r)--++(-.1,0);
\draw (l)--++(.1,0) (r)--++(-.1,0) (rr)--++(-.1,0);
\draw (u)--++(-45:.07)--(uu);





\end{tikzpicture}
\end{document}