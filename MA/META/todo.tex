\documentclass[]{standalone}
\usepackage[utf8]{inputenc}
\usepackage[german]{babel}
\usepackage[T1]{fontenc}
\usepackage{amsmath}
\usepackage{amsfonts}
\usepackage{amssymb}
\usepackage{lmodern}
\usepackage{fourier}
\usepackage{tikz}
\usepackage{pgf}
\usetikzlibrary{mindmap}

\author{Lars Schiller}
\date{\today}
\title{Mindmap Master Thesis}


\begin{document}

\begin{tikzpicture}	[mindmap,
every node/.style={concept, execute at begin node=\hskip0pt , fill=white, line width=1ex, text=black},
concept color=black!80,
grow cyclic,
level 1/.append style={level distance=4.5cm,sibling angle=90},
level 2/.append style={level distance=3cm,sibling angle=72},
level 3/.append style={line width=0.5ex, fill=red}]
		
\node[root concept]{Laufen auf schiefer Ebene} %root
child 	{node{GeckoBot 2.0}
			child 	{node{Entwurf}}
			child 	{node{Fertigung}
						child {node{Blasen- vermeidung}}
						child {node{Toleranzen}}
					}
			child	{node{Konstruktion}} 
		}
child 	{ node {ControlBoard}
			child 	{node{Löten}
					}
			child 	{node{Gehäuse}
						child{node{LaserCutter}} 
						child{node{CAD}}
						child{node{Nachbearbeitung}} 
						child{node{...}}
					}
			child 	{node{Komponenten- auswahl} 
						child{node{Ventile}} 
						child{node{Recheneinheit}}
						child{node{Sensoren}} 
						child{node{...}}
					}
			child 	{node{Sensorik}
						child{node{Analog}}
						child{node{Digital}}
					}
		}
child 	{node{Theretische Grundlagen} 
			child	{node{Laufmuster}
						child {node{Bionik}}
						child {node{weiche Robotik}} 
					} 
			child	{node{Regelung} 
						child {node{PID}}
						child {node{Implementierung}} 
					}
		}
child 	{ node {Software}
			child 	{node{GUI}
						child{node{GTK3+}}
						child{node{...}}
					}
			child 	{node{Signale}
						child{node{I2C}}
						child{node{PWM}}
						child{node{OPAMP}}
					}
			child 	{node {Kommunkation} 
						child{node{Berkeley-Sockets}}
					}
			child 	{node {Python} 
					}
		}
;
\end{tikzpicture}

\end{document}