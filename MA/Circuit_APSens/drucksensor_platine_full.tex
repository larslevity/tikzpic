\documentclass[10pt]{standalone}
\input{../../tikzpic_packages.tex}

\begin{document}
\begin{tikzpicture}[
scale = .5,
lochraster/.style={very thin, lightgray},
screw/.style={very thick, fill=gray},
gebohrt/.style={fill, pattern=north west lines, opacity=.5},
fontnode/.style={scale=.5, rotate=90},
resistor/.style={sloped, scale=.5, fill=red!50, minimum width=2cm},
capacity/.style={sloped, scale=.5, fill=blue!50},
led/.style={sloped, scale=.5, fill=green!50},
pluscable/.style={line width=.1cm, red},
minuscable/.style={line width=.1cm, blue},
signalcable/.style={line width=.1cm, orange}
]


%%%%%%%% LOCHRASTER %%%%%%%%
%\newcommand{\calcidx}{\pgfmathtruncatemacro{\idx}{(\y-1)*40 + \x}}

\foreach \y in {1, 2, ..., 20}{
	\foreach \x in {1, 2, ..., 40}{
		\draw[lochraster] (\x, \y) circle (.2);
	}
}
\foreach \x in {1,2,...,40}{
	\draw[lochraster] (\x+.5,0.5)--++(0,20);
}

%%%%%%%% SCREWs %%%%%%%%


\draw[screw] (2,2) circle (.8);
\path[gebohrt] (.5, 3.5) rectangle++(3,1);
\draw[screw] (2,19) circle (.8);
\path[gebohrt] (.5, 17.5) rectangle++(3,-1);
\draw[screw] (39,15) circle (.8);
\path[gebohrt] (40.5, 12.5) rectangle++(-3,1);
\path[gebohrt] (40.5, 16.5) rectangle++(-3,1);


%%%%%%%% Power Supply %%%%%%%%

\def\y{6}
\draw (2-1.3, \y-1.3) rectangle++(4.6,2.6);
\draw[fill=black] (2, \y) circle(.2) ++ (0,-1)node[fontnode]{VCC$^+$};
\draw[fill=black] (2+2, \y) circle(.2) ++ (0,-1)node[fontnode]{VCC$^-$};

%%%%%%%% Power LED %%%%%%%%

\draw (2,\y+3) node{$\bullet$}--++(1,3)node{$\bullet$} node[midway, resistor]{R1};
\draw (2+1,\y+2) node{$\bullet$}--++(1,0)node{$\bullet$} node[midway, led]{LED};



%%%%%%%% REF BLOCK %%%%%%%%

\def\y{15}
\draw (2-1.3, \y-1.3) rectangle++(4.6,2.6);
\draw[fill=black] (2, \y) circle(.2) ++ (0,1)node[fontnode]{NC};
\draw[fill=black] (2+2, \y) circle(.2) ++ (0,1)node[fontnode]{REF};

\draw[minuscable] (2+2,\y-3)node{$\bullet$}--++(2,0)node{$\bullet$};
\path[gebohrt] (2+2,\y-4)++(-.5,-.5)rectangle++(1,1);



%%%%%%%% Circuit %%%%%%%%

\foreach[count=\i] \x in {0,5,10,15,20,25}{
%\foreach[count=\i] \x in {0}{
	\def\xo{7}
	\def\yo{8}
	
	% OPAMPs
	\draw (\x+\xo,\yo)++(-.5,-.5)rectangle++(4,4);
	\path[gebohrt] (\x+\xo-.5,\yo+.5)rectangle++(4,2);
	\path (\x+\xo,\yo)++(0,3) node[fontnode]{OUT};
	\path (\x+\xo,\yo)++(1,3) node[fontnode]{NC (GAIN)};
	\path (\x+\xo,\yo)++(2,3) node[fontnode]{IN$^+$};
	\path (\x+\xo,\yo)++(3,3) node[fontnode]{VCC$^+$};
	\path (\x+\xo,\yo)++(0,0) node[fontnode]{GND};
	\path (\x+\xo,\yo)++(1,0) node[fontnode]{GAIN};
	\path (\x+\xo,\yo)++(2,0) node[fontnode]{IN$^-$};
	\path (\x+\xo,\yo)++(3,0) node[fontnode]{VCC$^-$};
	\draw (\x+\xo,\yo)++(-.5,1)arc(270:450:.5);
	
	%%% Elements
	% LOWPASS
	
	\draw (\x+\xo,\yo)++(0,4) node{$\bullet$}--++(-1,3)node{$\bullet$} node[midway, resistor]{R3};	
	\draw (\x+\xo,\yo)++(0,6) node{$\bullet$}--++(3,1)node{$\bullet$} node[midway, resistor]{R2};
	\draw (\x+\xo,\yo)++(3,8) node{$\bullet$}--++(1,0)node{$\bullet$} node[midway, capacity]{C};
	\draw (\x+\xo,\yo)++(3,9) node{$\bullet$}--++(-3,-1)node{$\bullet$} node[midway, resistor]{R2};
	\draw (\x+\xo,\yo)++(0,10) node{$\bullet$}--++(-1,0)node{$\bullet$} node[midway, capacity]{C};

	% In+ to IN+
	\draw[signalcable, rounded corners] (\x+\xo, 4) node{$\bullet$} --++(-.65,0) |-(\x+\xo+2,\yo+4)node{$\bullet$};
	% Cut the circuit
	\path[gebohrt] (\x+\xo,4)++(-.5,.5)rectangle++(1,1);
	\path[gebohrt] (\x+\xo,\yo)++(.5,5.5)rectangle++(3,1);
	\path[gebohrt] (\x+\xo,\yo)++(-.5,6.5)rectangle++(1,1);


	
	% GND to GAIN
	\draw (\x+\xo,\yo)++(0,-1) node{$\bullet$}--++(1,0)node{$\bullet$} ;
	% GND to GND
	\draw (\x+\xo,\yo)++(0,-2) node{$\bullet$}--++(-1,0)node{$\bullet$} ;
	% GND to GND
	\draw[minuscable] (\x+\xo,\yo)++(1,-2) node{$\bullet$}--++(3,1)node{$\bullet$} ;




	% Cables
	\ifthenelse{\i=1}{\def\dx{-6}}{\def\dx{-5}}
	\draw[minuscable] (\x+\xo+3, 5)node{$\bullet$}--++(\dx,-1)node{$\bullet$};
	\ifthenelse{\i=1}{\def\dx{-8}}{\def\dx{-5}}
	\draw[pluscable] (\x+\xo+3, \yo+4)node{$\bullet$}--++(\dx,1)node{$\bullet$};
	
	


	% INPUT BLOCK
	\draw (\x+\xo-1.3, 0.7) rectangle++(4.6,2.6);
	\draw[fill=black] (\x+\xo, 2) circle(.2) ++ (0,-1)node[fontnode]{IN$^+$};
	\draw[fill=black] (\x+\xo+2, 2) circle(.2) ++ (0,-1)node[fontnode]{IN$^-$};


	% PWM IN
	\path (\x+\xo-1, \yo-6) coordinate (PWM\i);
}


%%%%% TABLE %%%%%%

\path (2,-2) node[left]{R1:} node[right]{300 Ohm};
\path (2,-3) node[left]{LED:} node[right]{green 2V};
\path (2,-4) node[left]{R2:} node[right]{510 Ohm green orange black black};
\path (2,-5) node[left]{C:} node[right]{220 uF};
\path (2,-6) node[left]{C:} node[right]{22 kOhm red red orange gold };




\end{tikzpicture}
\end{document}