\documentclass[10pt]{standalone}

\def\path{../../}

%% For sans serfen Schrift:
%\usepackage[onehalfspacing]{setspace}
%\usepackage{helvet}
%\renewcommand{\familydefault}{\sfdefault}


\usepackage[utf8]{inputenc}
\usepackage[german]{babel}
\usepackage[T1]{fontenc}
\usepackage{graphicx}
\usepackage{lmodern}

\usepackage{amsmath}
\usepackage{amsfonts}
\usepackage{amssymb}
\usepackage{gensymb}
\usepackage{bm}
\usepackage{ifthen}
\usepackage{array}
\usepackage{eurosym}

\usepackage{tikz}
\usetikzlibrary{calc,patterns,
                 decorations.pathmorphing,
                 decorations.markings,
                 decorations.pathreplacing}
\usetikzlibrary{trees,arrows, scopes}
\usetikzlibrary{shapes}
\usetikzlibrary{positioning}
\usetikzlibrary{fadings}
\usepackage{tikz-3dplot}
\usetikzlibrary{matrix,fit, backgrounds}


\usepackage{pgfplots}
\usepgfplotslibrary{groupplots}
\pgfplotsset{compat=newest}

                 
\newcommand*\circled[1]{\tikz[baseline=(char.base)]{
            \node[shape=circle,draw,inner sep=2pt,solid,fill=white] (char) {#1};}}
            
            
            
\usetikzlibrary{patterns}

\pgfdeclarepatternformonly[\LineSpace]{my north east lines}{\pgfqpoint{-1pt}{-1pt}}{\pgfqpoint{\LineSpace}{\LineSpace}}{\pgfqpoint{\LineSpace}{\LineSpace}}%
{
    \pgfsetlinewidth{0.4pt}
    \pgfpathmoveto{\pgfqpoint{0pt}{0pt}}
    \pgfpathlineto{\pgfqpoint{\LineSpace + 0.1pt}{\LineSpace + 0.1pt}}
    \pgfusepath{stroke}
}


\pgfdeclarepatternformonly[\LineSpace]{my north west lines}{\pgfqpoint{-1pt}{-1pt}}{\pgfqpoint{\LineSpace}{\LineSpace}}{\pgfqpoint{\LineSpace}{\LineSpace}}%
{
    \pgfsetlinewidth{0.4pt}
    \pgfpathmoveto{\pgfqpoint{0pt}{\LineSpace}}
    \pgfpathlineto{\pgfqpoint{\LineSpace + 0.1pt}{-0.1pt}}
    \pgfusepath{stroke}
}

\newdimen\LineSpace
\tikzset{
    line space/.code={\LineSpace=#1},
    line space=3pt
}

\newcommand\centerofmass{%
    \tikz[radius=0.4em] {%
        \fill (0,0) -- ++(0.4em,0) arc [start angle=0,end angle=90] -- ++(0,-0.8em) arc [start angle=270, end angle=180];%
        \draw (0,0) circle;%
    }%
}

\usepackage{url}

\begin{document}
\begin{tikzpicture}[
stretch/.style={line width=.15cm, color=red!50}, % change only with comination with a \h
dstretch/.style={line width=.3cm, color=red!50},
nonstretch/.style={thick},
block/.style={minimum width=3cm, minimum height=1.2cm, fill=gray!10, draw, align=center},
camera/.style={fill=black},
stativ/.style={very thick},
arrow/.style={-triangle 45}
]

\def\la{1}		% Laenge der Arme
\def\lb{1.1}		% Laenge des Bauches
\def\h{.15}		% Hoehe der nicht dehnbaren Schicht. Change only in combination with stretch style
\def\R{.13}			% Radius der Hände

\pgfmathsetmacro{\hh}{\h*.5}
\pgfmathsetmacro{\Rh}{\R*.5}

\def\pi{3.1416}

\def\YSHIFT{2}	% Verschiebung der Zeilen
\def\XSHIFT{3.8}	% Verschiebung der Einzelbilder in x


\path (0,0)coordinate(OR)coordinate(OL)coordinate(O);


\foreach[count=\i] \gam/\fp/\yshift in {
%90/-1/0,
%20/-1/0,
%-90/-1/0, 
%-20/1/0, 
90/1/0}{

	% Calculations
	\pgfmathsetmacro{\alp}{45 + \gam*.5}	% Winkel der rechten Arme
	\pgfmathsetmacro{\bet}{45 - \gam*.5}	% Winkel der linken Arme
	\ifthenelse{\gam=90}{					% Radius der rechten Arme
		\pgfmathsetmacro{\rb}{\la}}{
		\pgfmathsetmacro{\rb}{\la/\bet*180/\pi}}	
	\ifthenelse{\gam=-90}{					% Radius der rechten Arme
		\pgfmathsetmacro{\ra}{\la}}{
		\pgfmathsetmacro{\ra}{\la/\alp*180/\pi}}	
	\ifthenelse{\gam=0}{
		\pgfmathsetmacro{\rB}{\lb}	% Radius des Bauches
		\pgfmathsetmacro{\gamh}{0}
		}{
		\pgfmathsetmacro{\signgam}{\gam/sqrt(\gam*\gam)}
		\pgfmathsetmacro{\absgam}{\gam*\signgam}
		\pgfmathsetmacro{\rB}{\signgam*\lb*180/\pi/\absgam}	% Radius des Bauches
		\pgfmathsetmacro{\gamh}{\gam*.5}					% Gamma Halbe
		}

	% Check ob verschoben werden soll oder nicht:
	\pgfmathsetmacro{\dXSHIFT}{\XSHIFT*2}
	\ifthenelse{1=\yshift}{\def\xs{-\dXSHIFT} \def\ys{-\YSHIFT}}{\def\xs{\XSHIFT} \def\ys{0}}
	\ifnum\i=1
		\def\xs{0}\def\ys{0} \fi	

	% Koordinatensystem verschieben
	\path (O)++(\xs,\ys)coordinate(O);
	
%	% Frame
%	\begin{scope}[very thin,lightgray]
%	\def\step{.68}
%	\path (O)++(-.9,-2.55)node(X){};
%	\foreach[count=\i] \row in {0,1,2,3,4,5}{
%		\foreach[count=\j] \col in {0,1,2,3,4}{
%			\pgfmathtruncatemacro{\idx}{mod(\row*5+\col,2)}
%			\ifnum\idx=1 \def\fcol{black!20} \fi
%			\ifnum\idx=0 \def\fcol{white} \fi
%			\pgfmathsetmacro{\x}{\col*\step}
%			\pgfmathsetmacro{\y}{\row*\step}			
%			\path[fill=\fcol](X)++(\x,\y)coordinate(x)rectangle++(\step,\step);
%		}	
%	}
%	\pgfmathsetmacro{\steph}{.5*\step}
%	\path (x)++(-\step-\steph,\step)node[above,black]{$\gamma=\gam ^\circ$};
%
%	\end{scope}
	
	
	%\begin{scope}[thick]


%%%%%%%%%%%%%%%%%%%%%% Konturen %%%%%%%%%%%%%%%%%%%%%%%%%%%%%%%%%%%%%

	% Start oben rechts oder links, je nach Vorzeichen von Gamma:
	\ifthenelse{0 < \fp}{
		% Rechter oberer Arm
		\ifthenelse{\gam=-90}{
			\draw[stretch] (OR)++(\xs,\ys)coordinate(OR)++(180-\gamh+90:\hh)--++(180-\gamh:\ra);
			\draw (OR) --++ (180-\gamh:\ra) coordinate(OM); }{
			\draw[stretch] (OR)++(\xs,\ys)coordinate(OR)++(-90+\alp-\gamh:\hh)arc(90+\alp-\gamh:90-\gamh:-\ra-\hh);
			\draw (OR) arc (90+\alp-\gamh:90-\gamh:-\ra) coordinate(OM);}
		% Linker oberer Arm
		\ifthenelse{\gam=90}{
			\draw[stretch] (OM)++(180+\gamh:\hh)--++(90+\gamh:\rb);
			\draw (OM)--++ (90+\gamh:\rb) coordinate(OL); }{
			\draw[stretch] (OM)++(-90-\gamh:\hh)arc (-90-\gamh:-90-\gamh-\bet:\rb+\hh);
			\draw (OM) arc (-90-\gamh:-90-\gamh-\bet:\rb) coordinate(OL);}
	}{
		% Linker oberer Arm
		\ifthenelse{\gam=90}{
			\draw[stretch] (OL)++(\xs,\ys)coordinate(OL)++(-\gamh-90:\hh)--++(-\gamh:\rb);
			\draw (OL) --++ (-\gamh:\rb) coordinate(OM); }{
			\draw[stretch] (OL)++(\xs,\ys)coordinate(OL)++(-90-\bet-\gamh:\hh)arc(-90-\bet-\gamh:-90-\gamh:\rb+\hh);
			\draw (OL) arc (-90-\bet-\gamh:-90-\gamh:\rb) coordinate(OM);}
		% Rechter oberer Arm
		\ifthenelse{\gam=-90}{
			\draw[stretch] (OM)++(-\gamh-90:\hh)--++(-\gamh:\ra);
			\draw (OM) --++(-\gamh:\ra) coordinate(OR); }{
			\draw[stretch] (OM)++(-90-\gamh:\hh)arc (-90-\gamh:-90-\gamh+\alp:\ra+\hh);
			\draw (OM) arc (-90-\gamh:-90-\gamh+\alp:\ra) coordinate(OR);}
	}
	
	% body body
	\ifthenelse{\gam=0}{
		\draw[dstretch] (OM)--++(-90:\rB);
		\draw (OM)--++(-90:\rB) coordinate(UM); }{
		\draw[dstretch] (OM)arc(180-\gamh:180+\gamh:\rB);
		\draw (OM) arc (180-\gamh:180+\gamh:\rB) coordinate(UM);}

	
	% Rechter unterer Arm
	\ifthenelse{\gam=-90}{
		\draw[stretch] (UM)++(\gamh+90:\hh)--++ (\gamh:\ra);
		\draw (UM)--++ (\gamh:\ra) coordinate(UR); }{
		\draw[stretch] (UM)++(90+\gamh:\hh)arc (90+\gamh:90+\gamh-\alp:\ra+\hh);
		\draw (UM) arc (90+\gamh:90+\gamh-\alp:\ra) coordinate(UR);}

	
	% Linker unterer Arm
	\ifthenelse{\gam=90}{
		\draw[stretch] (UM)++(180+\gamh-90:\hh)--++ (180+\gamh:\rb);
		\draw (UM)--++ (180+\gamh:\rb) coordinate(UL); }{
		\draw[stretch] (UM)++(90+\gamh:\hh)arc (90+\gamh:90+\gamh+\bet:\rb+\hh);
		\draw (UM) arc (90+\gamh:90+\gamh+\bet:\rb) coordinate(UL);}


%%%%%%%%%%%%%%%%%%%%%%%%%%%%%%%%%%%%%%%%%%%%%%%%%%%%%%%%%%%%%%
% Nachzeichnen der Kontur fuer bessere Optik
	% Start oben rechts oder links, je nach Vorzeichen von Gamma:
	\ifthenelse{0 < \fp}{
		% Rechter oberer Arm
		\ifthenelse{\gam=-90}{
			\draw[nonstretch] (OR) --++ (180-\gamh:\ra) coordinate(OM); }{
			\draw[nonstretch] (OR) arc (90+\alp-\gamh:90-\gamh:-\ra) coordinate(OM);}
		% Linker oberer Arm
		\ifthenelse{\gam=90}{
			\draw[nonstretch] (OM)--++ (90+\gamh:\rb) coordinate(OL); }{
			\draw[nonstretch] (OM) arc (-90-\gamh:-90-\gamh-\bet:\rb) coordinate(OL);}
	}{
		% Linker oberer Arm
		\ifthenelse{\gam=90}{
			\draw[nonstretch] (OL) --++ (-\gamh:\rb) coordinate(OM); }{
			\draw[nonstretch] (OL) arc (-90-\bet-\gamh:-90-\gamh:\rb) coordinate(OM);}
		% Rechter oberer Arm
		\ifthenelse{\gam=-90}{
			\draw[nonstretch] (OM) --++(-\gamh:\ra) coordinate(OR); }{
			\draw[nonstretch] (OM) arc (-90-\gamh:-90-\gamh+\alp:\ra) coordinate(OR);}
	}
	
	% body body
	\ifthenelse{\gam=0}{
		\draw[nonstretch] (OM)--++(-90:\rB); }{
		\draw[nonstretch] (OM) arc (180-\gamh:180+\gamh:\rB);}
	
	% Rechter unterer Arm
	\ifthenelse{\gam=-90}{
		\draw[nonstretch] (UM)--++ (\gamh:\ra); }{
		\draw[nonstretch] (UM) arc (90+\gamh:90+\gamh-\alp:\ra);}

	% Linker unterer Arm
	\ifthenelse{\gam=90}{
		\draw[nonstretch] (UM)--++ (180+\gamh:\rb); }{
		\draw[nonstretch] (UM) arc (90+\gamh:90+\gamh+\bet:\rb);}


%%%%%%%%%%%%%% Haende %%%%%%%%%%%%%%%%%%%%%%%%%%%%%%%%%%%%%%%%
	
	\ifthenelse{0<\fp}{	
		\def\colA{black}
		\def\colB{white}
	}{
		\def\colA{white}
		\def\colB{black}
	}
	\draw [nonstretch,fill=\colA] (OR) ++ (\alp-\gamh:\Rh) ++ (\alp-\gamh-90:\hh) circle(\R);
	\draw [nonstretch,fill=\colB] (OL) ++ (180-\gamh-\bet:\Rh) ++ (180-\gamh-\bet+90:\hh) circle(\R);
	
	\draw [nonstretch,fill=\colB] (UR) ++ (-\alp+\gamh:\Rh) ++ (-\alp+\gamh+90:\hh) circle(\R);
	\draw [nonstretch,fill=\colA] (UL) ++ (180+\gamh+\bet:\Rh) ++ (180+\gamh+\bet-90:\hh) circle(\R);
	
} % End Foreach loop

%% Massstab
%\draw[latex-latex] (-.22,1)--++(.68,0)node[midway,above]{8\,cm};

\path (-6,0)node[block](C){ControlBoard};

\path (-6,3)node[block](M){kinetisches \\ Modell};

\draw[arrow] (M)--(C)node[midway, right]{$r$};


% Kamera
\def\angle{30}
\path (3,1.5)coordinate(K);
\begin{scope}[rotate=\angle]
\draw[camera] (K)++(0,.15)rectangle++(.2,-.3)coordinate(K1);
\draw[camera] (K1)++(0.02,.4)rectangle++(.4,-.6)coordinate(K)++(0,.6)coordinate(K2)++(-.2,-.6)coordinate(Joint);
\draw[camera] (K2)++(0,0.02)rectangle++(-.2,.1);
\end{scope}
\draw[stativ] (Joint)--++(0,-1.5)coordinate(X);
\draw[stativ] (X)--++(-1,-2);
\draw[stativ] (X)--++(1,-2);
\draw[stativ] (X)--++(.2,-2.2);


\draw[double, rounded corners=3mm] (C.east)--++(.5,0)--(-1.7,-.5)node[midway, sloped, above]{Druck}--++(0:.5);

\draw[arrow] (K)++(-.2,.6)|-(M.east)node[pos=.75,above]{$\alpha_1, \alpha_2, \gamma, \beta_1, \beta_2$};

% User
\umlactor[x=-10, y=3, scale=1.5]{User}
\draw[arrow] (User)--(M)node[midway, above]{Bahn};



\end{tikzpicture}
\end{document}
