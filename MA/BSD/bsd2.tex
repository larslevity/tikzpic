\documentclass[10pt]{standalone}

%% For sans serfen Schrift:
%\usepackage[onehalfspacing]{setspace}
%\usepackage{helvet}
%\renewcommand{\familydefault}{\sfdefault}


\usepackage[utf8]{inputenc}
\usepackage[german]{babel}
\usepackage[T1]{fontenc}
\usepackage{graphicx}
\usepackage{lmodern}

\usepackage{amsmath}
\usepackage{amsfonts}
\usepackage{amssymb}
\usepackage{gensymb}
\usepackage{bm}
\usepackage{ifthen}
\usepackage{array}
\usepackage{eurosym}

\usepackage{tikz}
\usetikzlibrary{calc,patterns,
                 decorations.pathmorphing,
                 decorations.markings,
                 decorations.pathreplacing}
\usetikzlibrary{trees,arrows, scopes}
\usetikzlibrary{shapes}
\usetikzlibrary{positioning}
\usetikzlibrary{fadings}
\usepackage{tikz-3dplot}
\usetikzlibrary{matrix,fit, backgrounds}


\usepackage{pgfplots}
\usepgfplotslibrary{groupplots}
\pgfplotsset{compat=newest}

                 
\newcommand*\circled[1]{\tikz[baseline=(char.base)]{
            \node[shape=circle,draw,inner sep=2pt,solid,fill=white] (char) {#1};}}
            
            
            
\usetikzlibrary{patterns}

\pgfdeclarepatternformonly[\LineSpace]{my north east lines}{\pgfqpoint{-1pt}{-1pt}}{\pgfqpoint{\LineSpace}{\LineSpace}}{\pgfqpoint{\LineSpace}{\LineSpace}}%
{
    \pgfsetlinewidth{0.4pt}
    \pgfpathmoveto{\pgfqpoint{0pt}{0pt}}
    \pgfpathlineto{\pgfqpoint{\LineSpace + 0.1pt}{\LineSpace + 0.1pt}}
    \pgfusepath{stroke}
}


\pgfdeclarepatternformonly[\LineSpace]{my north west lines}{\pgfqpoint{-1pt}{-1pt}}{\pgfqpoint{\LineSpace}{\LineSpace}}{\pgfqpoint{\LineSpace}{\LineSpace}}%
{
    \pgfsetlinewidth{0.4pt}
    \pgfpathmoveto{\pgfqpoint{0pt}{\LineSpace}}
    \pgfpathlineto{\pgfqpoint{\LineSpace + 0.1pt}{-0.1pt}}
    \pgfusepath{stroke}
}

\newdimen\LineSpace
\tikzset{
    line space/.code={\LineSpace=#1},
    line space=3pt
}

\newcommand\centerofmass{%
    \tikz[radius=0.4em] {%
        \fill (0,0) -- ++(0.4em,0) arc [start angle=0,end angle=90] -- ++(0,-0.8em) arc [start angle=270, end angle=180];%
        \draw (0,0) circle;%
    }%
}


\begin{document}
\begin{tikzpicture}[
scale = 2,
socket/.style={above, align=center, draw, fill=gray!10},
base/.style={fill=yellow!30},
queue/.style={fill=gray!10, thick}
]

% Defs
\def\r{1}
\pgfmathsetmacro{\rh}{\r*.5}
\pgfmathsetmacro{\dy}{\r/sqrt(2)}
\path (-1.5,0)coordinate(client)++(0,\r)node[above]{Client};
\path (1.5,0)coordinate(server)++(0,\r)node[above]{Server};

% Basics
\draw[base] (client)++(0:\r)--($(client)+(.2,0)$)|-($(client)+(45:\r)$)arc(45:360:\r);
\draw[base] (server)++(180:\r)--($(server)+(-.2,0)$)|-($(server)+(90+45:\r)$)coordinate(X)arc(90+45:-180:\r);

% Nodes
\path (client)++(-\rh+.1,0)node[socket]{bound \\ socket A};
\path (server)++(\rh-.15,0)node[socket]{bound \& \\ listening \\ socket B};

% Queue
\draw[queue] (server)++(-\r-.25,0.05)--++(\r,0)coordinate(Qur)|-++(-\r,\dy-0.1)coordinate(Qol);
\path (Qur)coordinate(start);
\foreach \el in {1,2,3}{
\draw[queue] (start)rectangle++(-.2,\dy-.1)++(0,-\dy+.1)coordinate(start);}
\path (Qol)node[above, align=center]{listen queue};


%% Clip and notes
\path[clip] ($(client)+(-\r-.3,\r+.3)$)rectangle($(server)+(\r+.3,-\r-1.6)$);

\path (client)++(\r+.5,-\r)node[below]{\vbox{
\begin{itemize}
\item \textsl{Client} hat eine \\ Verbindungsanfrage  verschickt.
\end{itemize}}};

\path (server)++(\r+.5,-\r)node[below]{\vbox{
\begin{itemize}
\item \textsl{Server} hat eine Anfrage in \\ \textsl{listen} \textsl{queue} erhalten.
\end{itemize}}
};




\end{tikzpicture}
\end{document}