\documentclass[10pt]{standalone}
\input{../../tikzpic_packages.tex}

\begin{document}
\begin{tikzpicture}[
scale = 2,
socket/.style={above, align=center, draw, fill=gray!10},
base/.style={fill=yellow!30},
queue/.style={fill=gray!10, thick}
]

% Defs
\def\r{1}
\pgfmathsetmacro{\rh}{\r*.5}
\pgfmathsetmacro{\dy}{\r/sqrt(2)}
\path (-1.5,0)coordinate(client)++(0,\r)node[above]{Client};
\path (1.5,0)coordinate(server)++(0,\r)node[above]{Server};

% Basics
\draw[base] (client)++(0:\r)--($(client)+(.2,0)$)|-($(client)+(45:\r)$)arc(45:360:\r);
\draw[base] (server)++(180:\r)--($(server)+(-.2,0)$)|-($(server)+(90+45:\r)$)coordinate(X)arc(90+45:-180:\r);

% Nodes
\path (client)++(-\rh+.1,0)node[socket]{bound \\ socket A};
\path (server)++(\rh-.15,0)node[socket]{bound \& \\ listening \\ socket B};

% Queue
\draw[queue] (server)++(-\r-.25,0.05)--++(\r,0)coordinate(Qur)|-++(-\r,\dy-0.1)coordinate(Qol);
\path (Qur)coordinate(start);
\foreach \el in {1,2,3}{
\draw[queue] (start)rectangle++(-.2,\dy-.1)++(0,-\dy+.1)coordinate(start);}
\path (Qol)node[above, align=center]{listen queue};


%% Clip and notes
\path[clip] ($(client)+(-\r-.3,\r+.3)$)rectangle($(server)+(\r+.3,-\r-1.6)$);

\path (client)++(\r+.5,-\r)node[below]{\vbox{
\begin{itemize}
\item \textsl{Client} hat eine \\ Verbindungsanfrage  verschickt.
\end{itemize}}};

\path (server)++(\r+.5,-\r)node[below]{\vbox{
\begin{itemize}
\item \textsl{Server} hat eine Anfrage in \\ \textsl{listen} \textsl{queue} erhalten.
\end{itemize}}
};




\end{tikzpicture}
\end{document}