\documentclass[10pt]{standalone}

%% For sans serfen Schrift:
%\usepackage[onehalfspacing]{setspace}
%\usepackage{helvet}
%\renewcommand{\familydefault}{\sfdefault}


\usepackage[utf8]{inputenc}
\usepackage[german]{babel}
\usepackage[T1]{fontenc}
\usepackage{graphicx}
\usepackage{lmodern}

\usepackage{amsmath}
\usepackage{amsfonts}
\usepackage{amssymb}
\usepackage{gensymb}
\usepackage{bm}
\usepackage{ifthen}
\usepackage{array}
\usepackage{eurosym}

\usepackage{tikz}
\usetikzlibrary{calc,patterns,
                 decorations.pathmorphing,
                 decorations.markings,
                 decorations.pathreplacing}
\usetikzlibrary{trees,arrows, scopes}
\usetikzlibrary{shapes}
\usetikzlibrary{positioning}
\usetikzlibrary{fadings}
\usepackage{tikz-3dplot}
\usetikzlibrary{matrix,fit, backgrounds}


\usepackage{pgfplots}
\usepgfplotslibrary{groupplots}
\pgfplotsset{compat=newest}

                 
\newcommand*\circled[1]{\tikz[baseline=(char.base)]{
            \node[shape=circle,draw,inner sep=2pt,solid,fill=white] (char) {#1};}}
            
            
            
\usetikzlibrary{patterns}

\pgfdeclarepatternformonly[\LineSpace]{my north east lines}{\pgfqpoint{-1pt}{-1pt}}{\pgfqpoint{\LineSpace}{\LineSpace}}{\pgfqpoint{\LineSpace}{\LineSpace}}%
{
    \pgfsetlinewidth{0.4pt}
    \pgfpathmoveto{\pgfqpoint{0pt}{0pt}}
    \pgfpathlineto{\pgfqpoint{\LineSpace + 0.1pt}{\LineSpace + 0.1pt}}
    \pgfusepath{stroke}
}


\pgfdeclarepatternformonly[\LineSpace]{my north west lines}{\pgfqpoint{-1pt}{-1pt}}{\pgfqpoint{\LineSpace}{\LineSpace}}{\pgfqpoint{\LineSpace}{\LineSpace}}%
{
    \pgfsetlinewidth{0.4pt}
    \pgfpathmoveto{\pgfqpoint{0pt}{\LineSpace}}
    \pgfpathlineto{\pgfqpoint{\LineSpace + 0.1pt}{-0.1pt}}
    \pgfusepath{stroke}
}

\newdimen\LineSpace
\tikzset{
    line space/.code={\LineSpace=#1},
    line space=3pt
}

\newcommand\centerofmass{%
    \tikz[radius=0.4em] {%
        \fill (0,0) -- ++(0.4em,0) arc [start angle=0,end angle=90] -- ++(0,-0.8em) arc [start angle=270, end angle=180];%
        \draw (0,0) circle;%
    }%
}


\begin{document}
\begin{tikzpicture}[
scale = .5,
lochraster/.style={very thin, lightgray},
screw/.style={very thick, fill=gray},
gebohrt/.style={fill, pattern=north west lines, opacity=.5},
fontnode/.style={scale=.5, rotate=90},
resistor/.style={sloped, scale=.5, fill=red!50, minimum width=2cm},
capacity/.style={sloped, scale=.5, fill=blue!50},
pluscable/.style={line width=.1cm, red},
minuscable/.style={line width=.1cm, blue},
signalcable/.style={line width=.1cm, orange}
]


%%%%%%%% LOCHRASTER %%%%%%%%
%\newcommand{\calcidx}{\pgfmathtruncatemacro{\idx}{(\y-1)*40 + \x}}

\foreach \y in {1, 2, ..., 20}{
	\foreach \x in {1, 2, ..., 40}{
		\draw[lochraster] (\x, \y) circle (.2);
	}
}
\foreach \x in {1,2,...,40}{
	\draw[lochraster] (\x+.5,0.5)--++(0,20);
}

%%%%%%%% SCREWs %%%%%%%%


\draw[screw] (2,2) circle (.8);
\path[gebohrt] (.5, 3.5) rectangle++(3,1);
\draw[screw] (2,19) circle (.8);
\path[gebohrt] (.5, 17.5) rectangle++(3,-1);
\draw[screw] (39,2) circle (.8);
\path[gebohrt] (40.5, 3.5) rectangle++(-3,1);
\draw[screw] (39,19) circle (.8);
\path[gebohrt] (40.5, 17.5) rectangle++(-3,-1);


%%%%%%%% Circuit %%%%%%%%

\foreach[count=\i] \x in {0,5,10,15,20,25}{
	\def\xo{7}
	\def\yo{10}
	
	% OPAMPs
	\draw (\x+\xo,\yo)++(-.5,-.5)rectangle++(4,4);
	\draw (\x+\xo-.5,\yo+1)arc(-90:90:.5);
	\path[gebohrt] (\x+\xo-.5,\yo+.5)rectangle++(4,2);
	\path (\x+\xo,\yo)++(1,3) node[fontnode]{VCC$^+$};
	\path (\x+\xo,\yo)++(2,3) node[fontnode]{OUT};
	\path (\x+\xo,\yo)++(1,0) node[fontnode]{IN$^-$};
	\path (\x+\xo,\yo)++(2,0) node[fontnode]{IN$^+$};
	\path (\x+\xo,\yo)++(3,0) node[fontnode]{VCC$^-$};
	\draw (\x+\xo,\yo)++(-.5,1)arc(270:450:.5);
	
	%%% Elements
	% R2
	\draw (\x+\xo,\yo)++(2,4) node{$\bullet$}--++(-3,1)node{$\bullet$} node[midway, resistor]{R2};
	% In- to OUT
	\draw[signalcable] (\x+\xo,\yo)++(-1,-1) node{$\bullet$}--++(2,0)node{$\bullet$};
	% R1
	\draw (\x+\xo,\yo)++(-1,-2) node{$\bullet$}--++(4,0)node{$\bullet$} node[midway, resistor]{R1};
	% C
	\draw (\x+\xo,\yo)++(2,-3) node{$\bullet$}--++(1,0)node{$\bullet$} node[midway, capacity]{C};
	% R
	\draw (\x+\xo,\yo)++(2,-5) node{$\bullet$}--++(-3,1)node{$\bullet$} node[midway, resistor]{R};
	% Cut the circuit
	\path[gebohrt] (\x+\xo,\yo)++(-1.5,-2.5)rectangle++(1,-1);

	% Cables
	\draw[minuscable] (\x+\xo, 17)node{$\bullet$}--++(5,-1)node{$\bullet$};
	\ifthenelse{\i=6}{\def\dx{6}}{\def\dx{5}}
	\draw[pluscable] (\x+\xo+1, 17)node{$\bullet$}--++(\dx,-1)node{$\bullet$};
	
	\ifthenelse{\i=6}{\def\dx{2}}{\def\dx{5}}
	\draw[minuscable] (\x+\xo, \yo)++(3,-5)node{$\bullet$}--++(\dx,-1)node{$\bullet$};
	
	


	% OUTPUT BLOCK
	\draw (\x+\xo-1.5, 20.5) rectangle++(5,-3);
	\draw[fill=black] (\x+\xo, 19) circle(.2) ++ (0,1)node[fontnode]{OUT$^-$};
	\draw[fill=black] (\x+\xo+2, 19) circle(.2) ++ (0,1)node[fontnode]{OUT$^+$};


	% PWM IN
	\path (\x+\xo-1, \yo-6) coordinate (PWM\i);
}


%% PWR Supply Block
\def\x{36} \def\y{10}
\draw (\x-.5,\y-.5) rectangle++(5,3);
\draw[fill=black] (\x+1,\y+1) circle(.2) ++ (0,1)node[fontnode]{VCC$^-$};
\draw[fill=black] (\x+3,\y+1) circle(.2) ++ (0,1)node[fontnode]{VCC$^+$};


%% PWM IN


\foreach \i in {1,2,...,6}{
	\draw[signalcable] (PWM\i)--(17+\i,2)coordinate(x);
	\ifthenelse{\i=4}{}{
	\path[gebohrt] (x)++(-.5,.5)rectangle++(1,1);}
	}




\end{tikzpicture}
\end{document}