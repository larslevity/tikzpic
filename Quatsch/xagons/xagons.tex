\documentclass{standalone}
\usepackage{amsmath,tikz}
\usepackage{xcolor}
\usetikzlibrary{backgrounds,calc}

\begin{document}
% The colour of the xagons:
\newcommand{\colA}{white};
\newcommand{\colB}{orange};


\begin{tikzpicture}[background rectangle/.style={fill=\colA},show background rectangle]
%% Gebe hier die Anzahl der Ecken des höchsten Vielecks ein:
\newcommand{\nn}{9}
\pgfmathsetmacro{\abs}{\nn-2}
\foreach \i in {3,...,\nn}
{
% The number of corners
\pgfmathtruncatemacro{\x}{\i}
% The number of xagons
\newcommand{\n}{20};
% The radius of the circle were the xagons live in :
\newcommand{\R}{10};
% The start point where all begins
\pgfmathsetmacro{\angle}{mod(\i,\abs)}
\pgfmathsetmacro{\spanne}{360/\abs}
\path (0,0)--(\angle*\spanne:\nn*2.5) coordinate(center);
% Draw the circle which is the home:
%\draw (center)circle(\R);

% Compute the corners of the xagons:
\foreach \i in {1,...,\x}
	{
		\coordinate (10\i) at ($(center)+(360*\i/\x:\R)$);
	}
	
% Draw the xagon \n -times:
\foreach \i in {1,...,\n}
	{
		\pgfmathsetmacro{\density}{100-\i/\n*100}
		\pgfmathsetmacro{\position}{\i/\n*0.3}
		\foreach \j in {2,...,\x}				
			{
				\pgfmathtruncatemacro{\lastpoint}{10\j-1}
				\fill[\colA!\density!\colB](\lastpoint)--(10\j)--(center);
			}
		% Draw the xagon of this part of the loop
		\draw[\colA!\density!\colB,fill=\colA!\density!\colB] (10\x)--(101)--(center);		
		
		% New startpoints:
		% First we need to save the first point :
		\coordinate (z) at (101);
		\pgfmathtruncatemacro{\y}{\x-1}
		\foreach \j in {1,...,\y}
			{
				\pgfmathtruncatemacro{\next}{10\j+1}
				\coordinate (10\j) at ($(10\j)!\position!(\next)$);
			}
		%at least the last corner:
		\coordinate (10\x) at ($(10\x)!\position!(z)$);		
	}

}
\end{tikzpicture}
\end{document}
