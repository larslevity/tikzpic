%% Definition der KOS
\documentclass[11pt]{standalone}
\usepackage{tikz}
\usepackage{tikz-3dplot}
\begin{document}

\tdplotsetmaincoords{60}{-120}

\begin{tikzpicture}[tdplot_main_coords,scale=2.2, rotate=130.7]




\coordinate (O) at (0,0,0);
\path[tdplot_screen_coords] (O)++(225:.3)node{};
\tdplotsetrotatedcoords{0}{0}{0}
\draw[tdplot_rotated_coords,thick,->] (O) --++ (-3,0,0) node[right]{$z_0=z'$};
\draw[tdplot_rotated_coords,thick,->] (O) --++ (0,3,0) node[pos=.9, rotate=-30, above]{$y_0$};
\draw[tdplot_rotated_coords,thick,->] (O) --++ (0,0,3) node[below]{$x_0$};
\path (0,1.8,0) coordinate(y);

%% Drehung um alpha
\coordinate (O) at (0,0,0);
\path[tdplot_screen_coords] (O)++(225:.3)node{};
\tdplotsetrotatedcoords{-90}{10}{90}
\draw[tdplot_rotated_coords,dashed] (O) --++ (-2,0,0) node[right]{};
\draw[tdplot_rotated_coords,dashed] (O) --++ (0,2,0) node[pos=.87, rotate=-41, below right]{$y' = y''$};
\draw[tdplot_rotated_coords,dashed] (O) --++ (0,0,2.1) node[left]{$x'$};
\path[tdplot_rotated_coords] (0,1.8,0)coordinate(alphay);
\tdplotsetrotatedcoords{0}{-90}{0}
\tdplotdrawarc[tdplot_rotated_coords,help lines]{(O)}{1.8}{80}{-10}{right}{};
\tdplotdrawarc[tdplot_rotated_coords,-latex]{(O)}{1.8}{90}{90-10}{right}{$\varphi$};
\tdplotdrawarc[tdplot_rotated_coords,-latex]{(O)}{1.8}{0}{-10}{below}{$\varphi$};
\foreach \i in {0,.1,...,1.8}{
\tdplotdrawarc[tdplot_rotated_coords,very thin]{(O)}{\i}{90}{90-10}{right}{};
\tdplotdrawarc[tdplot_rotated_coords,very thin]{(O)}{\i}{0}{-10}{}{};
}


%% Drehung um beta
\path[tdplot_screen_coords] (O)++(225:.3)node{};
\tdplotsetrotatedcoords{-44.561}{14.106}{45.4385}
\draw[tdplot_rotated_coords,thin,dashed] (O) --++ (-2.5,0,0) node[above]{$z''$};
\draw[tdplot_rotated_coords,thin,dashed] (O) --++ (0,2,0) node[right]{};
\draw[tdplot_rotated_coords,thin,dashed] (O) --++ (0,0,2.5) node[above]{};
\tdplotsetrotatedcoords{-90}{100}{90}
\tdplotdrawarc[tdplot_rotated_coords,help lines]{(O)}{1.8}{80}{180-10}{right}{};
\tdplotdrawarc[tdplot_rotated_coords,-latex]{(O)}{1.8}{90}{90-10}{left}{$\theta$};
\tdplotdrawarc[tdplot_rotated_coords,-latex]{(O)}{1.8}{180}{180-10}{above}{$\theta$};
\foreach \i in {0,.1,...,1.8}{
\tdplotdrawarc[tdplot_rotated_coords,very thin]{(O)}{\i}{90}{90-10}{right}{};
\tdplotdrawarc[tdplot_rotated_coords,very thin]{(O)}{\i}{180}{180-10}{}{};
}

%% Drehung um gamma
\path[tdplot_screen_coords] (O)++(225:.3)node{};
\tdplotsetrotatedcoords{-44.5615}{14.106}{55.4385}
%\draw[tdplot_rotated_coords,help lines] (O)circle (1.8);
\draw[tdplot_rotated_coords,->, red, thick] (O) --++ (-3,0,0) node[right]{$z_1$};
\draw[tdplot_rotated_coords,->, red, thick] (O) --++ (0,3,0) node[below]{$y_1$};
\draw[tdplot_rotated_coords,->, red, thick] (O) --++ (0,0,3) node[pos=.9, rotate=37, above]{$x_1=x''$};
%\tdplotsetrotatedcoords{-90}{100}{90}
\tdplotdrawarc[tdplot_rotated_coords,help lines]{(O)}{1.8}{80}{180-10}{right}{};
\tdplotdrawarc[tdplot_rotated_coords,latex-]{(O)}{1.8}{90}{90-10}{left}{$\psi$};
\tdplotdrawarc[tdplot_rotated_coords,latex-]{(O)}{1.8}{180}{180-10}{above}{$\psi$};
\foreach \i in {0,.1,...,1.8}{
\tdplotdrawarc[tdplot_rotated_coords,very thin]{(O)}{\i}{90}{90-10}{right}{};
\tdplotdrawarc[tdplot_rotated_coords,very thin]{(O)}{\i}{180}{180-10}{}{};
}
\draw[tdplot_screen_coords,fill=white] (O)circle (.02);

\end{tikzpicture}

\end{document}