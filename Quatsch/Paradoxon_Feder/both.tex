%% Paradoxon Feder

\documentclass[tikz,convert={outfile=\jobname.png}]{standalone}

%% For sans serfen Schrift:
%\usepackage[onehalfspacing]{setspace}
%\usepackage{helvet}
%\renewcommand{\familydefault}{\sfdefault}


\usepackage[utf8]{inputenc}
\usepackage[german]{babel}
\usepackage[T1]{fontenc}
\usepackage{graphicx}
\usepackage{lmodern}

\usepackage{amsmath}
\usepackage{amsfonts}
\usepackage{amssymb}
\usepackage{gensymb}
\usepackage{bm}
\usepackage{ifthen}
\usepackage{array}
\usepackage{eurosym}

\usepackage{tikz}
\usetikzlibrary{calc,patterns,
                 decorations.pathmorphing,
                 decorations.markings,
                 decorations.pathreplacing}
\usetikzlibrary{trees,arrows, scopes}
\usetikzlibrary{shapes}
\usetikzlibrary{positioning}
\usetikzlibrary{fadings}
\usepackage{tikz-3dplot}
\usetikzlibrary{matrix,fit, backgrounds}


\usepackage{pgfplots}
\usepgfplotslibrary{groupplots}
\pgfplotsset{compat=newest}

                 
\newcommand*\circled[1]{\tikz[baseline=(char.base)]{
            \node[shape=circle,draw,inner sep=2pt,solid,fill=white] (char) {#1};}}
            
            
            
\usetikzlibrary{patterns}

\pgfdeclarepatternformonly[\LineSpace]{my north east lines}{\pgfqpoint{-1pt}{-1pt}}{\pgfqpoint{\LineSpace}{\LineSpace}}{\pgfqpoint{\LineSpace}{\LineSpace}}%
{
    \pgfsetlinewidth{0.4pt}
    \pgfpathmoveto{\pgfqpoint{0pt}{0pt}}
    \pgfpathlineto{\pgfqpoint{\LineSpace + 0.1pt}{\LineSpace + 0.1pt}}
    \pgfusepath{stroke}
}


\pgfdeclarepatternformonly[\LineSpace]{my north west lines}{\pgfqpoint{-1pt}{-1pt}}{\pgfqpoint{\LineSpace}{\LineSpace}}{\pgfqpoint{\LineSpace}{\LineSpace}}%
{
    \pgfsetlinewidth{0.4pt}
    \pgfpathmoveto{\pgfqpoint{0pt}{\LineSpace}}
    \pgfpathlineto{\pgfqpoint{\LineSpace + 0.1pt}{-0.1pt}}
    \pgfusepath{stroke}
}

\newdimen\LineSpace
\tikzset{
    line space/.code={\LineSpace=#1},
    line space=3pt
}

\newcommand\centerofmass{%
    \tikz[radius=0.4em] {%
        \fill (0,0) -- ++(0.4em,0) arc [start angle=0,end angle=90] -- ++(0,-0.8em) arc [start angle=270, end angle=180];%
        \draw (0,0) circle;%
    }%
}


\begin{document}
\begin{tikzpicture}
% FederMAKRO_______________

%	SECHS PARAMETER:
				%% 1. x-Koord. Start
				%% 2. y-Koord. Start
				%% 3. Arbeitslänge
				%% 4. Windungen
				%% 5. Drahtbreite
				%% 6. Außendurchmesser

\renewcommand{\L}{\l+\D};		% Arbeitslänge
\renewcommand{\l}{8cm}			 %{\L - \D} % wirksame Länge
\newcommand{\D}{3cm};			% Mittlerer Duchmesser
\renewcommand{\d}{0.2cm};		%Drahtdurchmesser	
\newcommand{\n}{7};				%wirksame Windungen
\newcommand{\m}{6} 				%   {\n-1};	
\pgfmathsetmacro{\steigung}{atan(\l/2/\n/\D)};
\coordinate(Start) at (0,0) ;	% Startpunkt
\coordinate(LO) at ($(Start) + (-\D/2,-\D/2)$);
\coordinate(RO) at ($(Start) + (\D/2,-\D/2-\l/\n/2)$);
\coordinate(RU) at ($(LO) + (\D,-\l-\l/\n/2)$);

\newcommand{\feder}{ 
				%%%ENDEN%%
		\draw[rounded corners=\d*3/2] (LO)--++(\D/2+\d/2 ,\l/4/\n)--($(Start)+(\d/2,0)$);
		\draw[rounded corners=\d] ($(LO)+(0,\d)$)--++(\D/2-\d/2,\l/4/\n)--($(Start)+(-\d/2,0)$);
		\draw[rounded corners=\d] (RU)--++(-\D/2+\d/2 ,-\l/4/\n)--($(Start)+(\d/2,-\l-\D)$);
		\draw[rounded corners=\d*3/2] ($(RU)+(0,\d)$)--++(-\D/2-\d/2,-\l/4/\n)--($(Start)+(-\d/2,-\l-\D)$);			
				%%%Hintere Federteile
		\foreach \x in {0,1,2,...,\m}{		
			\draw ($(RO)+(0,-\l/\n*\x)$)--++(-\D,-\l/\n/2)arc(-90-\steigung:-270-\steigung:\d/2)--++(\D,\l/\n/2)arc(+90-\steigung:-90-\steigung:\d/2);
			}	
				%%% Voredere Federteile
		\foreach \x in {0,1,...,\n}{		
			\draw[fill=white] ($(LO)+(0,-\l/\n*\x)$)--++(\D,-\l/\n/2)arc(-90-\steigung:90-\steigung:\d/2)--++(-\D,\l/\n/2)arc(+90-\steigung:270-\steigung:\d/2)--cycle;				
			}		
		};
		
%%%%% Gewicht Makro &&&&&________________________
%FLASCHE: #1 -> x , #2 -> y , #3 -> breite   höhe = 1.6*#3 , #4 -> options
\newcommand{\flasche}[4]{
	\draw[#4,rounded corners=3pt] (#1,#2)--++(#3*1/2,0)--++(0,1*#3)to[]++(-#3*0.4,0.2*#3)--++(0,0.3*#3)--++(0.1*#3,0)--++(0,0.1*#3)--(#1,#2+1.6*#3);
	\draw[#4,rounded corners=3pt] (#1,#2)--++(-#3*1/2,0)--++(0,1*#3)to[]++(#3*0.4,0.2*#3)--++(0,0.3*#3)--++(-0.1*#3,0)--++(0,0.1*#3)--(#1,#2+1.6*#3);
		
	\fill[black,path fading=west,rounded corners=3pt]	(#1,#2)--++(#3*1/2,0)--++(0,1*#3)to[]++(-#3*0.4,0.2*#3)--++(0,0.3*#3)--++(0.1*#3,0)--++(0,0.1*#3)--(#1,#2+1.6*#3)--++(-0.2*#3,0)--++(0,-0.1*#3)--++(0.1*#3,0)--++(0,-0.3*#3)to[]++(-#3*0.4,-0.2*#3)--++(0,-1*#3)--cycle;
;}


%%%%%__________________________________________BILD_________&&&&&&&

\draw[thick] (-3,0)--(3,0);
\fill[pattern=north east lines] (-3,0)rectangle(3,0.5);
\draw [fill=black,rounded corners=1pt] (-0.3,0)rectangle(0.3,-0.1);
\draw [fill=black,rounded corners=1pt] (2.5,0)rectangle(1.9,-0.1);

\feder;

\draw [thick] (-2.6,-11)--(1.5,-11);
\draw [fill=black,rounded corners=1pt] (-0.3,-11)rectangle(0.3,-11.1);
\draw [fill=black,rounded corners=1pt] (-2.5,-11)rectangle(-1.9,-11.1);   %%% --> -2.2 geht Schnur ab.

\draw[thin] (0,-11)--(0,-20);

\draw [thick] (-1.5,-20)--(2.6,-20);
\draw [fill=black,rounded corners=1pt] (-0.3,-20)rectangle(0.3,-20.1);
\draw [fill=black,rounded corners=1pt] (2.5,-20)rectangle(1.9,-20.1);   %%% --> 2.2 geht Schnur ab.


\coordinate(Start) at (0,-20) ;	% Startpunkt
\coordinate(LO) at ($(Start) + (-\D/2,-\D/2)$);
\coordinate(RO) at ($(Start) + (\D/2,-\D/2-\l/\n/2)$);
\coordinate(RU) at ($(LO) + (\D,-\l-\l/\n/2)$);

\feder

\draw [thick] (-2.6,-31)--(1.5,-31);
\draw [fill=black,rounded corners=1pt] (-0.3,-31)rectangle(0.3,-31.1);
\draw [fill=black,rounded corners=1pt] (-2.5,-31)rectangle(-1.9,-31.1);   %%% --> -2.2 geht Schnur ab.

\draw[thin](0,-31)--(0,-35+2*1.6);
\flasche{0}{-35}{2}{} ;

%%%% FADEN &&&&& _____________________________________________________________&&&&&
\draw[thin](2.2,0).. controls (3.3,-20) and (2.9,-22)..(2.2,-20);
\draw[thin](-2.2,-11).. controls (-3.3,-31) and (-2.9,-33)..(-2.2,-31);













\begin{scope}[xshift=10cm]

% FederMAKRO_______________

%	SECHS PARAMETER:
				%% 1. x-Koord. Start
				%% 2. y-Koord. Start
				%% 3. Arbeitslänge
				%% 4. Windungen
				%% 5. Drahtbreite
				%% 6. Außendurchmesser

\renewcommand{\L}{\l+\D};		% Arbeitslänge
\renewcommand{\l}{5cm}			 %{\L - \D} % wirksame Länge
\renewcommand{\D}{3cm};			% Mittlerer Duchmesser
\renewcommand{\d}{0.2cm};		%Drahtdurchmesser	
\renewcommand{\n}{7};				%wirksame Windungen
\renewcommand{\m}{6} 				%   {\n-1};	
\renewcommand{\steigung}{atan(\l/2/\n/\D)};
\coordinate(Start) at (0,0) ;	% Startpunkt
\coordinate(LO) at ($(Start) + (-\D/2,-\D/2)$);
\coordinate(RO) at ($(Start) + (\D/2,-\D/2-\l/\n/2)$);
\coordinate(RU) at ($(LO) + (\D,-\l-\l/\n/2)$);

\renewcommand{\feder}{ 
				%%%ENDEN%%
		\draw[rounded corners=\d*3/2] (LO)--++(\D/2+\d/2 ,\l/4/\n)--($(Start)+(\d/2,0)$);
		\draw[rounded corners=\d] ($(LO)+(0,\d)$)--++(\D/2-\d/2,\l/4/\n)--($(Start)+(-\d/2,0)$);
		\draw[rounded corners=\d] (RU)--++(-\D/2+\d/2 ,-\l/4/\n)--($(Start)+(\d/2,-\l-\D)$);
		\draw[rounded corners=\d*3/2] ($(RU)+(0,\d)$)--++(-\D/2-\d/2,-\l/4/\n)--($(Start)+(-\d/2,-\l-\D)$);			
				%%%Hintere Federteile
		\foreach \x in {0,1,2,...,\m}{		
			\draw ($(RO)+(0,-\l/\n*\x)$)--++(-\D,-\l/\n/2)arc(-90-\steigung:-270-\steigung:\d/2)--++(\D,\l/\n/2)arc(+90-\steigung:-90-\steigung:\d/2);
			}	
				%%% Voredere Federteile
		\foreach \x in {0,1,...,\n}{		
			\draw[fill=white] ($(LO)+(0,-\l/\n*\x)$)--++(\D,-\l/\n/2)arc(-90-\steigung:90-\steigung:\d/2)--++(-\D,\l/\n/2)arc(+90-\steigung:270-\steigung:\d/2)--cycle;				
			}		
		};
		
%%%%% Gewicht Makro &&&&&________________________
%FLASCHE: #1 -> x , #2 -> y , #3 -> breite   höhe = 1.6*#3 , #4 -> options
\renewcommand{\flasche}[4]{
	\draw[#4,rounded corners=3pt] (#1,#2)--++(#3*1/2,0)--++(0,1*#3)to[]++(-#3*0.4,0.2*#3)--++(0,0.3*#3)--++(0.1*#3,0)--++(0,0.1*#3)--(#1,#2+1.6*#3);
	\draw[#4,rounded corners=3pt] (#1,#2)--++(-#3*1/2,0)--++(0,1*#3)to[]++(#3*0.4,0.2*#3)--++(0,0.3*#3)--++(-0.1*#3,0)--++(0,0.1*#3)--(#1,#2+1.6*#3);
		
	\fill[black,path fading=west,rounded corners=3pt]	(#1,#2)--++(#3*1/2,0)--++(0,1*#3)to[]++(-#3*0.4,0.2*#3)--++(0,0.3*#3)--++(0.1*#3,0)--++(0,0.1*#3)--(#1,#2+1.6*#3)--++(-0.2*#3,0)--++(0,-0.1*#3)--++(0.1*#3,0)--++(0,-0.3*#3)to[]++(-#3*0.4,-0.2*#3)--++(0,-1*#3)--cycle;
;}

%%%%%__________________________________________BILD_________&&&&&&&

\draw[thick] (-3,0)--(3,0);
\fill[pattern=north east lines] (-3,0)rectangle(3,0.5);
\draw [fill=black,rounded corners=1pt] (-0.3,0)rectangle(0.3,-0.1);
\draw [fill=black,rounded corners=1pt] (2.5,0)rectangle(1.9,-0.1);

\feder;

\draw [thick] (-2.6,-8)--(1.5,-8);
\draw [fill=black,rounded corners=1pt] (-0.3,-8)rectangle(0.3,-8.1);
\draw [fill=black,rounded corners=1pt] (-2.5,-8)rectangle(-1.9,-8.1);   %%% --> -2.2 geht Schnur ab.


%%%% Faden RISS
\draw [rounded corners,thin] (0,-8)
  \foreach \i in {1,...,10} {
    -- ++(rnd*-170:rnd)};
    
\draw [rounded corners,thin] (0,-21)
  \foreach \i in {1,...,15} {
    -- ++(rnd*170:rnd)};


\draw [thick] (-1.5,-21)--(2.6,-21);
\draw [fill=black,rounded corners=1pt] (-0.3,-21)rectangle(0.3,-21.1);
\draw [fill=black,rounded corners=1pt] (2.5,-21)rectangle(1.9,-21.1);   %%% --> 2.2 geht Schnur ab.


\coordinate(Start) at (0,-21) ;	% Startpunkt
\coordinate(LO) at ($(Start) + (-\D/2,-\D/2)$);
\coordinate(RO) at ($(Start) + (\D/2,-\D/2-\l/\n/2)$);
\coordinate(RU) at ($(LO) + (\D,-\l-\l/\n/2)$);

\feder

\draw [thick] (-2.6,-29)--(1.5,-29);
\draw [fill=black,rounded corners=1pt] (-0.3,-29)rectangle(0.3,-29.1);
\draw [fill=black,rounded corners=1pt] (-2.5,-29)rectangle(-1.9,-29.1);   %%% --> -2.2 geht Schnur ab.

\draw[thin](0,-29)--(0,-33+2*1.6);
\flasche{0}{-33}{2}{} ;

%%%% FADEN &&&&& _____________________________________________________________&&&&&
\draw[thin](2.2,0)--(2.2,-21);
\draw[thin](-2.2,-8)--(-2.2,-29);


\end{scope}



\end{tikzpicture}
\end{document}