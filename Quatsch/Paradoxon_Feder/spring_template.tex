\documentclass[]{article}
\usepackage{tikz}
\usepackage{pgfplots}
\pgfplotsset{compat=newest} % "neue" Verbesserungen nutzen
\pgfplotsset{plot coordinates/math parser=false}
\usepgflibrary{shadings}
%
\usetikzlibrary{arrows}
\usetikzlibrary{decorations.pathmorphing}
\usetikzlibrary{backgrounds}
\usetikzlibrary{positioning}
\usetikzlibrary{fit}
\usetikzlibrary{petri}
\usetikzlibrary{trees}
\usetikzlibrary{mindmap}
\usetikzlibrary{shadows}
\usetikzlibrary{er}
\usetikzlibrary{shadings}
%\usetikzlibrary{}
%
\usepackage[ansinew]{inputenc}
\usepackage{verbatim}
\usepackage[active,tightpage]{preview}
\PreviewEnvironment{tikzpicture}
\setlength\PreviewBorder{0pt}%
%
\usepgfplotslibrary{groupplots}
\usetikzlibrary{pgfplots.groupplots}
%
% Eigene Colormap, verl�uft von wei� zu schwarz
\pgfplotsset{%
colormap={whiteblack}{gray(0cm)=(1); gray(1cm)=(0)}
}%
%
% Strickst�rke der Koordinatenachsen
\pgfplotsset{%
every axis/.append style={semithick}
}%
%
% Strickst�rke der Plots
\pgfplotsset{%
every axis plot/.append style={thick}
}%
%
% Schriftgr��en in Plots
\pgfplotsset{%
tick label style={font=\small},
label style={font=\small},
legend style={font=\small}
}%
%
% "rainbow spectrum" shading
\pgfdeclareverticalshading{rainbow}{100bp}
{color(0bp)=(red); color(25bp)=(red); color(35bp)=(yellow);
color(45bp)=(green); color(55bp)=(cyan); color(65bp)=(blue);
color(75bp)=(violet); color(100bp)=(violet)}
%


\begin{document}
\begin{tikzpicture}
% FederMAKRO_______________

%	SECHS PARAMETER:
				%% 1. x-Koord. Start
				%% 2. y-Koord. Start
				%% 3. Arbeitslänge
				%% 4. Windungen
				%% 5. Drahtbreite
				%% 6. Außendurchmesser

\renewcommand{\L}{\l+\D};		% Arbeitslänge
\renewcommand{\l}{8cm}			 %{\L - \D} % wirksame Länge
\newcommand{\D}{3cm};			% Mittlerer Duchmesser
\renewcommand{\d}{0.1cm};		%Drahtdurchmesser	
\newcommand{\n}{12};				%wirksame Windungen
\pgfmathsetmacro{\m}{\n-1} 				%   {\n-1};	
\newcommand{\steigung}{atan(\l/2/\n/\D)};
\coordinate(Start) at (0,0) ;	% Startpunkt
\coordinate(LO) at ($(Start) + (-\D/2,-\D/2)$);
\coordinate(RO) at ($(Start) + (\D/2,-\D/2-\l/\n/2)$);
\coordinate(RU) at ($(LO) + (\D,-\l-\l/\n/2)$);

\newcommand{\feder}{ 
				%%%ENDEN%%
		\draw[rounded corners=\d*3/2] (LO)--++(\D/2+\d/2 ,\l/4/\n)--($(Start)+(\d/2,0)$);
		\draw[rounded corners=\d] ($(LO)+(0,\d)$)--++(\D/2-\d/2,\l/4/\n)--($(Start)+(-\d/2,0)$);
		\draw[rounded corners=\d] (RU)--++(-\D/2+\d/2 ,-\l/4/\n)--($(Start)+(\d/2,-\l-\D)$);
		\draw[rounded corners=\d*3/2] ($(RU)+(0,\d)$)--++(-\D/2-\d/2,-\l/4/\n)--($(Start)+(-\d/2,-\l-\D)$);			
				%%%Hintere Federteile
		\foreach \x in {0,1,2,...,\m}{		
			\draw ($(RO)+(0,-\l/\n*\x)$)--++(-\D,-\l/\n/2)arc(-90-\steigung:-270-\steigung:\d/2)--++(\D,\l/\n/2)arc(+90-\steigung:-90-\steigung:\d/2);
			}	
				%%% Voredere Federteile
		\foreach \x in {0,1,...,\n}{		
			\draw[fill=white] ($(LO)+(0,-\l/\n*\x)$)--++(\D,-\l/\n/2)arc(-90-\steigung:90-\steigung:\d/2)--++(-\D,\l/\n/2)arc(+90-\steigung:270-\steigung:\d/2)--cycle;				
			}		
		};
		
\feder
		
\end{tikzpicture}
\end{document}