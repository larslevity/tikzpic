\documentclass[]{minimal}
\usepackage{amsmath,tikz}
\usetikzlibrary{backgrounds,calc}
\usepackage[active,tightpage]{preview}
\PreviewEnvironment{tikzpicture}
\setlength\PreviewBorder{0pt}%
\begin{document}
\begin{tikzpicture}[background rectangle/.style={fill=black},show background rectangle]


% Fibonacci numbers : f(n)={0,1,2,3,5,8,13,21,34}
%create some counters to handle the fibonacci numbers
\newcounter{a}
\newcounter{b}
\newcounter{temp}
%initialize the counters
\setcounter{a}{0};
\setcounter{b}{1};

% The spiral will start at the origin
\coordinate (0) at (0,0);

%this loop will create the fibonacci spiral.
\foreach \i in {1,2,...,10}
{	
	% Get the "name" of the last point on the spiral
    \pgfmathsetmacro{\lastpoint}{\i-1}

    % Compute the angle for this turn of the spiral
    \pgfmathsetmacro{\startangle}{mod(\i-1,4) * 90}

    % Draw this turn of the spiral and remember the point where we end 
    \draw[white] (\lastpoint) arc 
      (\startangle : \startangle +90: \value{b}) coordinate (\i);
      
      
    % Draw some frame
    \draw[black!90](\lastpoint)rectangle(\i) coordinate[pos=.5] (center);  
    \draw[black!90](center)circle(\value{b}/2);
    
 
    \draw[black!80] (\lastpoint)--(\i)node[white,midway,circle,fill=black,font=\fontsize{20}{23}\selectfont]{\the\value{b}};
	\draw[black!80] (\i) arc (\startangle +180 : \startangle+270: \value{b});
	
	%draw this part of the spiral again:
	\draw[white] (\lastpoint) arc 
      (\startangle : \startangle +90: \value{b});
    
    % Compute the next Fibonacci number
    \setcounter{temp}{\value{b}}
    \addtocounter{b}{\value{a}}
    \setcounter{a}{\value{temp}}
}

% Add some text displaying the formula for the Fibonacci numbers
 \node(eq) at ($(7) + (10.2,1.4)$) 
   [white,text width = 2cm,font=\fontsize{20}{23}\selectfont] {
   \begin{displaymath}
     f(n) = \left\{
       \begin{array}{lr}
         0 & \text{~~if } n = 0\\
         1 & \text{~~if } n = 1\\
         f(n-1) + f(n-2) & \text{~~if } n \geq 2\\
      \end{array}
     \right.
   \end{displaymath}
  };
  
 
  
\end{tikzpicture}
\end{document}