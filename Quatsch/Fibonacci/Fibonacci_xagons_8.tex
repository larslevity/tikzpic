% Fibonacci <=8

\documentclass[]{minimal}
\usepackage{amsmath,tikz}
\usetikzlibrary{backgrounds,calc}
\usepackage{ifthen}
\usepackage[active,tightpage]{preview}
\PreviewEnvironment{tikzpicture}
\setlength\PreviewBorder{0pt}%
\newboolean{greatenough}
\begin{document}

		%%%%___________Settings for the xagons
		% The colour of the xagons:
		\newcommand{\colA}{white};
		\newcommand{\colB}{purple};

\begin{tikzpicture}[background rectangle/.style={fill=\colA},show background rectangle]

% Fibonacci numbers : f(n)={0,1,2,3,5,8,13,21,34}
%create some counters to handle the fibonacci numbers
\newcounter{a}
\newcounter{b}
\newcounter{temp}
%initialize the counters
\setcounter{a}{0};
\setcounter{b}{1};

% The spiral will start at the origin
\coordinate (0) at (0,0);

%this loop will create the fibonacci spiral.
\foreach \i in {1,2,...,6}
{	
	% Get the "name" of the last point on the spiral
    \pgfmathsetmacro{\lastpoint}{\i-1}

    % Compute the angle for this turn of the spiral
    \pgfmathsetmacro{\startangle}{mod(\i-1,4) * 90}

    % Draw this turn of the spiral and remember the point where we end 
    \draw[\colA] (\lastpoint) arc 
      (\startangle : \startangle +90: \value{b}) coordinate (\i);
      
      
    % Draw some frame
    \draw[\colA!90!\colB](\lastpoint)rectangle(\i) coordinate[pos=.5] (center);  
    \draw[\colA!90!\colB](center)circle(\value{b}/2);
%%_____________________________________________________________________________________%%
	% Add some rotated xagons:	
	\ifnum\value{b} > 2
	
% The number of corners
\pgfmathtruncatemacro{\x}{\value{b}}
% The number of xagons
\newcommand{\n}{30};
% The radius of the circle were the xagons live in :
\pgfmathsetmacro{\R}{\value{b}/2};


% Compute the corners of the xagons:
\foreach \i in {1,...,\x}
	{
		\coordinate (10\i) at ($(center)+(360*\i/\x:\R)$);
	}
	
% Draw the xagon \n -times:
\foreach \i in {1,...,\n}
	{
		\pgfmathsetmacro{\density}{99-\i/\n*99}
		\pgfmathsetmacro{\position}{\value{b}/8*0.3}
		\foreach \j in {2,...,\x}				
			{
				\pgfmathtruncatemacro{\lastpoint}{10\j-1}
				\fill[\colA!\density!\colB](\lastpoint)--(10\j)--(center);
			}
		% Draw the xagon of this part of the loop
		\draw[\colA!\density!\colB,fill=\colA!\density!\colB] (10\x)--(101)--(center);		
		
		% New startpoints:
		% First we need to save the first point :
		\coordinate (z) at (101);
		\pgfmathtruncatemacro{\y}{\x-1}
		\foreach \j in {1,...,\y}
			{
				\pgfmathtruncatemacro{\next}{10\j+1}
				\coordinate (10\j) at ($(10\j)!\position!(\next)$);
			}
		%at least the last corner:
		\coordinate (10\x) at ($(10\x)!\position!(z)$);		
	}
			\fi
 %%  _________________________end xagons______________________________________________________&%%
 
 	\ifnum\value{b}>2
    \path (\lastpoint)--(\i)
    node[\colA,midway,circle,font=\fontsize{6}{8}\selectfont]{\the\value{b}};
	\coordinate (Z1) at ($(\lastpoint)!0.47!(\i)$);
	\coordinate (Z2) at ($(\lastpoint)!0.53!(\i)$);
    \draw[\colA!80!\colB] (\lastpoint)--(Z1) (Z2)--(\i);
    \fi
    \ifnum\value{b}<3
    \draw[\colA!80!\colB] (\lastpoint)--(\i)
    node[\colB,midway,circle,fill=\colA,font=\fontsize{6}{8}\selectfont]{\the\value{b}};
    \fi
	\draw[\colA!80!\colB] (\i) arc (\startangle +180 : \startangle+270: \value{b});
	
	%draw this part of the spiral again:
	\draw[\colB] (\lastpoint) arc 
   		 (\startangle : \startangle +90: \value{b});
    
    % Compute the next Fibonacci number
    \setcounter{temp}{\value{b}}
    \addtocounter{b}{\value{a}}
    \setcounter{a}{\value{temp}}
}

% Add some text displaying the formula for the Fibonacci numbers
 \node(eq) at ($(6) + (4.5,0.8)$) 
   [\colB,text width = 2cm,font=\fontsize{6}{8}\selectfont] {
   \begin{displaymath}
     f(n) = \left\{
       \begin{array}{lr}
         0 & \text{~~if } n = 0\\
         1 & \text{~~if } n = 1\\
         f(n-1) + f(n-2) & \text{~~if } n \geq 2\\
      \end{array}
     \right.
   \end{displaymath}
  };
  
 
  
\end{tikzpicture}
\end{document}