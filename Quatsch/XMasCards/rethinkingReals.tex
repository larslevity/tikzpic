% Author: Till Tantau
% Source: The PGF/TikZ manual
\documentclass[tikz]{standalone}

\usepackage{tikz}
\usepackage{xifthen}


\begin{document}



\def\xMax{3}
\def\N{30}
\def\angle{35}
\begin{tikzpicture}[scale=10, rotate=\angle]

\pgfmathsetmacro{\root}{sqrt(2)}
\def\scale{\root*1.6}
\path[rotate=-\angle, clip] (0,0)rectangle++(\root*\scale, \scale);


% decimals
%\draw (0,0)--(\xMax,0);
%\draw (0,1)--(-1,-1);
%\foreach \x in {0,0.1, ..., \xMax} {
%	\draw (\x, 0.01)--++(0,-0.02);
%}


% fractals
\foreach \x in {0, ..., \xMax} {
	% draw the one
	\fill[purple, opacity=.1] (\x, 0) circle (1);
	\foreach \denominator in {2, ..., \N} {
		\pgfmathtruncatemacro{\bb}{\denominator - 1}
		\foreach [evaluate={\gcd=gcd(\nominator,\denominator)}] \nominator in {1, ..., \bb} {
			% for convinience			
			\pgfmathsetmacro{\r}{1/\denominator * 1/\denominator}
			\path (\x + \nominator / \denominator, 0)coordinate(c);
			
			% test for greatest common divisor			
			\ifnum \gcd = 1
				\def\color{pink}
				\ifnum \nominator = 1 \def\color{purple} \fi
				\ifnum \nominator = \bb \def\color{purple} \fi
%				\ifnum \nominator = 3 \def\color{green} \fi
				\fill[color=\color, opacity=.3] (c) circle(\r);
%				\path (c) ++ (0,\r) node[above, scale = \r * 4] (label) {$ \frac{\nominator } { \denominator }$} ;
			\fi			

		}
	}

}






\end{tikzpicture}

\end{document}