\documentclass[tikz, convert={outfile=\jobname.png}]{standalone}
\usepackage{amsmath,tikz}
\usepackage{xcolor}
\usetikzlibrary{backgrounds,calc}
\usepackage{tikz-3dplot}
\begin{document}
% The colour of the xagons:
\newcommand{\colA}{white}
\newcommand{\colB}{black}

\tdplotsetmaincoords{40}{230}
\begin{tikzpicture}[tdplot_main_coords,background rectangle/.style={fill=\colA},show background rectangle]

\pgfmathsetmacro{\position}{10/100}
\def\alph{5.82}

\coordinate (center) at (0,0,0);
\coordinate (1) at ($(center)+(120:2)$);
\coordinate (2) at ($(center)+(240:2)$);
\coordinate (3) at ($(center)+(360:2)$);
%% Höhe regulieren
		\foreach \j in {1,2,3}{
			\coordinate (1\j) at ($(\j)+(0,0,-.2)$);
			%% Linien ziehen::
			\draw (1\j)--++(0,0,.2);		
		}	
\draw[] (11)--(12)--(13)--cycle;

\foreach \i in {0,1,...,10}{
	\pgfmathsetmacro{\density}{80-\i/18*100};
	\draw[fill=\colA!\density!\colB] (1)--(2)--(3)--cycle;

% New startpoints:
		% First we need to save the first point :
		\coordinate (z) at (1);
		\pgfmathtruncatemacro{\y}{3-1}
		\foreach \j in {1,...,\y}
			{
				\pgfmathtruncatemacro{\next}{\j+1}
				\coordinate (\j) at ($(\j)!\position!(\next)$);
			}
		%at least the last corner:
		\coordinate (3) at ($(3)!\position!(z)$);	


\draw[fill=\colA!\density!\colB] (1)--(2)--(3)--cycle;		
		
		%% Höhe regulieren
		\foreach \j in {1,2,3}{
			\coordinate (\j) at ($(\j)+(0,0,.2)$);
			%% Linien ziehen::
			\draw (\j)--++(0,0,-.2);		
		}

\draw[fill=\colA!\density!\colB] (1)--(2)--(3)--cycle;
}

\foreach \i in {1,2,3}{
\coordinate (1\i) at ($(1\i)+(0,0,.2)$);
}
% First we need to save the first point :
		\coordinate (z) at (11);
		\pgfmathtruncatemacro{\y}{3-1}
		\foreach \j in {1,...,\y}
			{
				\pgfmathtruncatemacro{\next}{1\j+1}
				\coordinate (2\j) at ($(1\j)!\position!(\next)$);
			}
		%at least the last corner:
		\coordinate (23) at ($(13)!\position!(z)$);	
		
		% First we need to save the first point :
		\coordinate (z) at (21);
		\pgfmathtruncatemacro{\y}{3-1}
		\foreach \j in {1,...,\y}
			{
				\pgfmathtruncatemacro{\next}{2\j+1}
				\coordinate (3\j) at ($(2\j)!\position!(\next)$);
			}
		%at least the last corner:
		\coordinate (33) at ($(23)!\position!(z)$);

%% Höhe regulieren
		\foreach \j in {1,2,3}{
			\coordinate (3\j) at ($(3\j)+(0,0,.2)$);	
		}



\draw[help lines] (11)--++(-90:5);
\draw[help lines] (21)--++(-90+\alph:5);
\tdplotdrawarc[tdplot_main_coords,->]{(21)}{4.5}{-90}{-90+\alph}{above}{$\alpha$};

\draw[help lines] (11)++(0,0,-.2)--++(-2,0,0)coordinate(X);
\draw[help lines] (12)++(0,0,-.2)--++(-2,0,0)coordinate(Y);
\draw[latex-latex] (X)--(Y) node[midway,fill=white,rectangle]{$\ell_0$};

\draw[help lines] (21)--++(\alph:-1.5)coordinate(X);
\draw[help lines] (22)--++(\alph:-1.5)coordinate(Y);
\draw[latex-latex] (X)--(Y) node[midway,fill=white,rectangle]{$\ell_1$};

\draw[help lines] (31)--++(2*\alph:-1)coordinate(X);
\draw[help lines] (32)--++(2*\alph:-1)coordinate(Y);
\draw[latex-latex] (X)--(Y) node[midway,fill=white,rectangle]{$\ell_2$};




%\path(22)node{x};

\coordinate (O) at (1,2,0);
\draw[->] (O)--++(1,0,0)node[right]{$x$};
\draw[->] (O)--++(0,1,0)node[below]{$y$};
\draw[->] (O)--++(0,0,1)node[right]{$z$};



\end{tikzpicture}
\end{document}
