

\documentclass[tikz,convert={outfile=\jobname.png}]{standalone}

%% For sans serfen Schrift:
%\usepackage[onehalfspacing]{setspace}
%\usepackage{helvet}
%\renewcommand{\familydefault}{\sfdefault}


\usepackage[utf8]{inputenc}
\usepackage[german]{babel}
\usepackage[T1]{fontenc}
\usepackage{graphicx}
\usepackage{lmodern}

\usepackage{amsmath}
\usepackage{amsfonts}
\usepackage{amssymb}
\usepackage{gensymb}
\usepackage{bm}
\usepackage{ifthen}
\usepackage{array}
\usepackage{eurosym}

\usepackage{tikz}
\usetikzlibrary{calc,patterns,
                 decorations.pathmorphing,
                 decorations.markings,
                 decorations.pathreplacing}
\usetikzlibrary{trees,arrows, scopes}
\usetikzlibrary{shapes}
\usetikzlibrary{positioning}
\usetikzlibrary{fadings}
\usepackage{tikz-3dplot}
\usetikzlibrary{matrix,fit, backgrounds}


\usepackage{pgfplots}
\usepgfplotslibrary{groupplots}
\pgfplotsset{compat=newest}

                 
\newcommand*\circled[1]{\tikz[baseline=(char.base)]{
            \node[shape=circle,draw,inner sep=2pt,solid,fill=white] (char) {#1};}}
            
            
            
\usetikzlibrary{patterns}

\pgfdeclarepatternformonly[\LineSpace]{my north east lines}{\pgfqpoint{-1pt}{-1pt}}{\pgfqpoint{\LineSpace}{\LineSpace}}{\pgfqpoint{\LineSpace}{\LineSpace}}%
{
    \pgfsetlinewidth{0.4pt}
    \pgfpathmoveto{\pgfqpoint{0pt}{0pt}}
    \pgfpathlineto{\pgfqpoint{\LineSpace + 0.1pt}{\LineSpace + 0.1pt}}
    \pgfusepath{stroke}
}


\pgfdeclarepatternformonly[\LineSpace]{my north west lines}{\pgfqpoint{-1pt}{-1pt}}{\pgfqpoint{\LineSpace}{\LineSpace}}{\pgfqpoint{\LineSpace}{\LineSpace}}%
{
    \pgfsetlinewidth{0.4pt}
    \pgfpathmoveto{\pgfqpoint{0pt}{\LineSpace}}
    \pgfpathlineto{\pgfqpoint{\LineSpace + 0.1pt}{-0.1pt}}
    \pgfusepath{stroke}
}

\newdimen\LineSpace
\tikzset{
    line space/.code={\LineSpace=#1},
    line space=3pt
}

\newcommand\centerofmass{%
    \tikz[radius=0.4em] {%
        \fill (0,0) -- ++(0.4em,0) arc [start angle=0,end angle=90] -- ++(0,-0.8em) arc [start angle=270, end angle=180];%
        \draw (0,0) circle;%
    }%
}

\usepackage{tikz-3dplot}


\begin{document}
\tdplotsetmaincoords{70.5}{215}
%\tdplotsetmaincoords{85}{215}
%\tdplotsetmaincoords{90}{180}


\begin{tikzpicture}[
	%tdplot_main_coords,
	scale=.5,
	line width=3mm,
	1/.style={fill=pink!50, draw=white},
	2/.style={fill=pink!20, draw=white},
	3/.style={fill=purple!10, draw=white},
	4/.style={fill=purple!40, draw=white},
	5/.style={fill=green!20, draw=white},
	6/.style={fill=red!20, draw=white},
	]

\def\lO{.5}


\path (0,0) coordinate(O);
\pgfmathsetmacro{\r}{sqrt(2*\lO*\lO)}
\foreach \a in {45,135,225,315}{
	\path (O)++(\a:\r)coordinate(\a);
}
\draw[2] (45)--(135)--(225)--(315)--cycle;
\draw[2] (O)++(-\lO,0)--++(\lO+\lO,0);
\draw[2] (O)++(0,-\lO)--++(0,\lO+\lO);

\def\N{15}

\foreach \n in {3,...,\N}{
	\pgfmathsetmacro{\l}{\lO*(\n-1)}
	\pgfmathsetmacro{\r}{sqrt(2*\l*\l)}
	\pgfmathtruncatemacro{\nn}{mod(\n, 6)+1}

	\path (225)++(225:\r) coordinate(X);
	\foreach \i in {1,...,\n}{
		\path[\nn] (X)--++(\l,0)coordinate(X)--++(0,\l)--++(-\l,0)--cycle;
	}
	\path (225)coordinate(X);
	\pgfmathtruncatemacro{\nx}{\n-2}
	\foreach \i in {1,...,\nx}{
		\path[\nn] (X)--++(0,\l)coordinate(X)--++(-\l,0)--++(0,-\l)--cycle;
	}
	\path (135)++(135:\r) coordinate(X);
	\foreach \i in {1,...,\n}{
		\path[\nn] (X)--++(\l,0)coordinate(X)--++(0,-\l)--++(-\l,0)--cycle;
	}
	\path (45)coordinate(X);
	\foreach \i in {1,...,\nx}{
		\path[\nn] (X)--++(0,-\l)coordinate(X)--++(\l,0)--++(0,\l)--cycle;
	}	

	\foreach \a in {45,135,225,315}{
		\path (\a)++(\a:\r)coordinate(\a);
	}
}



\end{tikzpicture}
\end{document}