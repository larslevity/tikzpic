%% Bauteil_2

\documentclass[12pt]{report}
\usepackage{tikz}
\usepackage{pgfplots}
\pgfplotsset{compat=newest} % "neue" Verbesserungen nutzen
\pgfplotsset{plot coordinates/math parser=false}
\usepgflibrary{shadings}
%
\usetikzlibrary{arrows}
\usetikzlibrary{decorations.pathmorphing}
\usetikzlibrary{backgrounds}
\usetikzlibrary{positioning}
\usetikzlibrary{fit}
\usetikzlibrary{petri}
\usetikzlibrary{trees}
\usetikzlibrary{mindmap}
\usetikzlibrary{shadows}
\usetikzlibrary{er}
\usetikzlibrary{shadings}
%\usetikzlibrary{}
%
\usepackage[ansinew]{inputenc}
\usepackage{verbatim}
\usepackage[active,tightpage]{preview}
\PreviewEnvironment{tikzpicture}
\setlength\PreviewBorder{0pt}%
%
\usepgfplotslibrary{groupplots}
\usetikzlibrary{pgfplots.groupplots}
%
% Eigene Colormap, verl�uft von wei� zu schwarz
\pgfplotsset{%
colormap={whiteblack}{gray(0cm)=(1); gray(1cm)=(0)}
}%
%
% Strickst�rke der Koordinatenachsen
\pgfplotsset{%
every axis/.append style={semithick}
}%
%
% Strickst�rke der Plots
\pgfplotsset{%
every axis plot/.append style={thick}
}%
%
% Schriftgr��en in Plots
\pgfplotsset{%
tick label style={font=\small},
label style={font=\small},
legend style={font=\small}
}%
%
% "rainbow spectrum" shading
\pgfdeclareverticalshading{rainbow}{100bp}
{color(0bp)=(red); color(25bp)=(red); color(35bp)=(yellow);
color(45bp)=(green); color(55bp)=(cyan); color(65bp)=(blue);
color(75bp)=(violet); color(100bp)=(violet)}
%

\usepackage{mathabx}

\begin{document}
\begin{tikzpicture}[scale = .7]

%% Körper

\draw[thick] (0,0)rectangle(8,2);
\foreach \x in {1,3,5,7}{
	\draw[thick] (\x,1)circle(.45);
	\draw[very thin] (\x-.7,1)--(\x+.7,1);
	\draw[very thin] (\x,.3)--(\x,1.7);
}

%% Hilflinien

\draw[very thin] (-.2,1)--(.2,1) (3.8,1)--(4.2,1) (7.8,1)--(8.2,1) (4,-.2)--(4,.7) (4,.9)--(4,1.1) (4,1.3)--(4,2.2) (-.1,0)--(-1.5,0) (-.1,2)--(-1.5,2) (0,-.1)--(0,-1.5) (1,.2)--(1,-1.5) (3,.2)--(3,-1.5) (0,2.1)--(0,3.5) (8,2.1)--(8,3.5) (7,1)--++(30:3)node[near end,rotate=30,above]{$\diameter 9$};

%% Bemaßungen

\newcommand{\drawArrowH}[3]{
	\draw[latex-latex] (#1)--(#2)node[midway,above]{$#3$};	
}

\drawArrowH{$(0,-1.3)$}{$(1,-1.3)$}{10};
\drawArrowH{$(1,-1.3)$}{$(3,-1.3)$}{20};
\drawArrowH{$(0,3.2)$}{$(8,3.2)$}{80};

\draw[latex-latex] (-1.3,0)--(-1.3,2)node[midway,above,rotate=90]{$20$};	

\draw[latex-latex] (7,1)++(30:-.45)--++(30:.9);

\draw[fill=black] (6,.5)circle(.05)coordinate(X);
\draw[very thin] (X)--(6.2,-1.3)--(7.8,-1.3)node[midway,above]{t = 5};

\end{tikzpicture}
\end{document}