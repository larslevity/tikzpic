\documentclass[12pt]{article}
%
\usepackage{tikz}
\usepackage{pgfplots}
\pgfplotsset{compat=newest} % "neue" Verbesserungen nutzen
\pgfplotsset{plot coordinates/math parser=false}
\usepgflibrary{shadings}
%
\usetikzlibrary{arrows}
\usetikzlibrary{decorations.pathmorphing}
\usetikzlibrary{backgrounds}
\usetikzlibrary{positioning}
\usetikzlibrary{fit}
\usetikzlibrary{petri}
\usetikzlibrary{trees}
\usetikzlibrary{mindmap}
\usetikzlibrary{shadows}
\usetikzlibrary{er}
\usetikzlibrary{shadings}
%\usetikzlibrary{}
%
\usepackage[ansinew]{inputenc}
\usepackage{verbatim}
\usepackage[active,tightpage]{preview}
\PreviewEnvironment{tikzpicture}
\setlength\PreviewBorder{0pt}%
%
\usepgfplotslibrary{groupplots}
\usetikzlibrary{pgfplots.groupplots}
%
% Eigene Colormap, verl�uft von wei� zu schwarz
\pgfplotsset{%
colormap={whiteblack}{gray(0cm)=(1); gray(1cm)=(0)}
}%
%
% Strickst�rke der Koordinatenachsen
\pgfplotsset{%
every axis/.append style={semithick}
}%
%
% Strickst�rke der Plots
\pgfplotsset{%
every axis plot/.append style={thick}
}%
%
% Schriftgr��en in Plots
\pgfplotsset{%
tick label style={font=\small},
label style={font=\small},
legend style={font=\small}
}%
%
% "rainbow spectrum" shading
\pgfdeclareverticalshading{rainbow}{100bp}
{color(0bp)=(red); color(25bp)=(red); color(35bp)=(yellow);
color(45bp)=(green); color(55bp)=(cyan); color(65bp)=(blue);
color(75bp)=(violet); color(100bp)=(violet)}
%
 
%
%
%Line pattern options: "dash pattern=<dash pattern>" (e.g. "dash pattern=on 2pt off 3pt on 4pt off 4pt"), "dash phase=[dash phase]", "solid", "dashed", "dotted", "dashdotted", "densely dotted", "loosely dotted", "double".
%
%
%
\begin{document}
%
\begin{tikzpicture}
%
%Rahmen
	\draw (-2,-7.5) rectangle (6,0.5);
	
%Dash-Rahmen für Vektoren
	\draw[dash pattern=on 4pt off 2pt] (-55:3.3) rectangle(-55:5);
%
%Schraffur
	\begin{scope}
	\path[clip,preaction={draw=white}] (5.8,-1)--(5.8,-1.3)--++(-0.8,0)--(-11.537:5.3)arc(-11.537:-70:5.3)--(2,-4.7)--++(0,-0.7)--++(-0.3,0)--(-70:5)arc(-70:-11.537:5)--(5.8,-1)--cycle;
	\draw[rotate=45,step=0.1,yscale=17,yshift=2] (-30,-30)grid(30,30);
	\end{scope}
%
%dashed
	\draw[dash pattern=on 20pt off 2.25pt on 0.5pt off 2.25,very thin]
		(0,0.2)--(0,-6.1);
%dick
	\draw[thick] 
		(0,-1)--(5.8,-1);
	\draw[thick]
		(-11.537:5)arc(-11.537:-70:5)--(0,-4.698);
	\draw[thick]
		(-70:5)--++(0,-0.7);
%dünn
	\draw[very thin] (0.2,0)--(-1,0);
	\draw[very thin] (0,-1)--(-0.6,-1);
	\draw[very thin]
		(0,0)--++(-70:5);
	\draw[very thin]
		(0,0)--++(-11.537:5)--++(0,-5.1);
	\draw[very thin]
		(0,-4.698)--++(-1,0);
%
	\draw[very thin] (0,0)--(-55:5)--(2.868,-5.7);
	\draw[very thin] (-52:5)--(3.078,-5.7);
%
%Vektoren
	\draw[-stealth,very thick] (-55:3.3)--(-55:4.9);
	\node[left,style={scale=0.8}] at (-55:4){$n$};
	\draw[stealth-,very thick] (-55:5)--++(0,1.3925);
	\node[below,style={scale=0.8}] at (3.05,-2.8){$p$};
	\draw[-stealth,very thick] (-55:5)--++(-0.975,0);
	\node[below,style={scale=0.8}] at(2.1,-4.1){$z$};
%%%%%%%%%%%%%%%
%Beschriftungen 
	\draw[thin,stealth-stealth]
		(-0.5,0)--(-0.5,-1);
	\node[rotate=90,style={rectangle,scale=0.8,fill=white}] 
		at(-0.5,-0.5){$t_a$};
	\draw[thin,stealth-stealth]
		(-0.9,0)--(-0.9,-4.698);
	\node[style={rectangle,scale=0.8,fill=white},rotate=90]
		at (-0.9,-2.35){$t_i$};
%
%Winkel-Bemaßungen
	\draw[thin,stealth-stealth](-11.537:2.2)to[out=258.4,in=20](-70:2.2);
	\node[rotate=50,style={rectangle,scale=0.8,fill=white}] at (-40:2.2){$\gamma'$};
%
	\draw[thin,stealth-stealth] (270:3)to[out=0,in=215](-55:3);
	\node[rotate=15,style={rectangle,scale=0.8,fill=white}] at(-75:3){$\gamma$};
%
%Die r-Bemaßungen
	\draw[thin,-stealth](0,-5.2)--++(1.7,0);
	\draw[fill=white](0,-5.2)circle(0.04);
	\node[style={rectangle,scale=0.8,fill=white}]at(.9,-5.2){$r_h$};
%
	\draw[thin,-stealth](0,-5.6)--++(2.868,0);
	\draw[fill=white](0,-5.6)circle(0.04);
	\node [style={rectangle,scale=0.8,fill=white}]at(1.5,-5.6){$r$};
%
	\draw[thin,-stealth](0,-6)--++(4.898,0);
	\draw[fill=white](0,-6)circle(0.04);
	\node[style={rectangle,scale=0.8,fill=white}]at(2.2,-6){$r_w$};
%
	\draw[thin,-stealth](3.6,-5.6)--++(-0.52,0);
	\node[style={scale=0.8},above]at(3.4,-5.6){$dr$};
%
%Rho-Bemaßung
	\draw[thin,-stealth](0,0)--(-25:5);
	\node[rotate=-25,style={rectangle,scale=0.8,fill=white}] at (-25:4){$\rho$};
%
%b-bemaßen
\draw[very thin](-11.527:5.2)--(-11.537:5.5)to[out=258.427,in=20](-70:5.5)--(-70:5.2);
\node[right,style={scale=0.8}]at(-45:5.6){$b$};
%
\end{tikzpicture}



\end{document}