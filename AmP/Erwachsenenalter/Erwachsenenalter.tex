% ---------- %
% S-Funktion %
% ---------- %
%
\documentclass[12pt]{article}
%
\usepackage{tikz}
\usepackage{pgfplots}
\pgfplotsset{compat=newest} % "neue" Verbesserungen nutzen
\pgfplotsset{plot coordinates/math parser=false}
\usepgflibrary{shadings}
%
\usetikzlibrary{arrows}
\usetikzlibrary{decorations.pathmorphing}
\usetikzlibrary{backgrounds}
\usetikzlibrary{positioning}
\usetikzlibrary{fit}
\usetikzlibrary{petri}
\usetikzlibrary{trees}
\usetikzlibrary{mindmap}
\usetikzlibrary{shadows}
\usetikzlibrary{er}
\usetikzlibrary{shadings}
%\usetikzlibrary{}
%
\usepackage[ansinew]{inputenc}
\usepackage{verbatim}
\usepackage[active,tightpage]{preview}
\PreviewEnvironment{tikzpicture}
\setlength\PreviewBorder{0pt}%
%
\usepgfplotslibrary{groupplots}
\usetikzlibrary{pgfplots.groupplots}
%
% Eigene Colormap, verl�uft von wei� zu schwarz
\pgfplotsset{%
colormap={whiteblack}{gray(0cm)=(1); gray(1cm)=(0)}
}%
%
% Strickst�rke der Koordinatenachsen
\pgfplotsset{%
every axis/.append style={semithick}
}%
%
% Strickst�rke der Plots
\pgfplotsset{%
every axis plot/.append style={thick}
}%
%
% Schriftgr��en in Plots
\pgfplotsset{%
tick label style={font=\small},
label style={font=\small},
legend style={font=\small}
}%
%
% "rainbow spectrum" shading
\pgfdeclareverticalshading{rainbow}{100bp}
{color(0bp)=(red); color(25bp)=(red); color(35bp)=(yellow);
color(45bp)=(green); color(55bp)=(cyan); color(65bp)=(blue);
color(75bp)=(violet); color(100bp)=(violet)}
%
 
%
\begin{document}
%
\begin{tikzpicture}[line cap=round,line join=round]
	\begin{axis}[%
		xtick={0,1,2,3,4,5,6},
%		x tick label style={anchor=north},
		xticklabels={15,16,17,18,19,20,21,22},
		ytick={0,0.25,0.5,0.75,1},
%		yticklabels={0, , , ,1},
		x=1cm,
		y=4cm,
		axis y line=left,
		axis x line=bottom,
		xmin=0,
		xmax=6,
		ymin=0,
		ymax=1, 
		scale only axis,
		grid=major,
		legend pos=outer north east,
		legend cell align=left,
		legend style={draw=none},
		]
		% scharfer Verlauf der Zugeh�rigkeit
		\draw[color=red] (0,0) -- (300,0);
		\draw[color=red] (300,0) -- (300,100);
		\addplot[color=red,domain=3:6] {1};
%		\addlegendimage{empty legend}
		% gradueller Verlauf der Zugeh�rigkeit
		\addplot[color=blue,domain=0:1] {0};
%		\addlegendimage{empty legend}
		\addplot[color=blue,domain=1:3] {2*((x-3+2)/(2*2))^2};
		\addplot[color=blue,domain=3:5] {1-2*((3-x+2)/(2*2))^2};
		\addplot[color=blue,domain=5:6] {1};
		%
		\addlegendentry{Vollj�hrigkeit}
%		\addlegendentry{scharfer Verlauf}
%		\addlegendentry{der Zugeh�rigkeit}
		\addlegendentry{Erwachsenenalter}
%		\addlegendentry{gradueller Verlauf}
%		\addlegendentry{der Zugeh�rigkeit}
		%
	\end{axis}
	%
	% labels
	\draw (6.05,0) node[right] {$x$\,[Jahre]};
	\draw (0,4.05) node[above] {$\mu(x)$};
	%
\end{tikzpicture}
%
\end{document} 