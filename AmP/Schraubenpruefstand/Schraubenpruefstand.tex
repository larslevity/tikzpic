%% Schraubenpruefstand

%% Beginn 31.8 1700
%% Ende 31.8 2124

\documentclass[tikz,convert={outfile=\jobname.png}]{standalone}
\input{../../tikzpic_packages.tex}


\begin{document}
\begin{tikzpicture}[scale = 1]

% draw a frame
\path (-0.1,-0.1) coordinate (0) rectangle(16.1,7.1);

%%_____________________________________________________________________
% left side of the picture
%%_____________________________________________________________________

% help lines
%\draw[help lines] (0,1.5) coordinate (llu) rectangle(8,5.5) coordinate (lro);
%\draw[dotted] (0,3.5)--(16,3.5);

%% Platte
\draw (0.1,1.5)rectangle(7.9,1.6);
\draw[fill=gray] (0.1,1.6)rectangle(7.9,1.7);
\draw (0.1,1.7)rectangle(7.9,1.8);
\draw[rounded corners = 0.5mm] (0.1,1.5)--(0,1.5)--(0,1.8)--(0.1,1.8);
\draw[rounded corners = 0.5mm] (7.9,1.5)--(8,1.5)--(8,1.8)--(7.9,1.8);

% Ständer 1
\draw (2.5,2)rectangle(3.5,1.8)rectangle(3.7,4);
\draw (2.9,2)--(3.5,2.6);

% Motor
\draw[rounded corners = 1mm] (0.2,2.9)rectangle(3.5,4.1);

\begin{scope}[xshift=.5cm]
% Ständer 2
\draw (5,1.8)rectangle(7,1.95);
\draw (6-0.15/2,1.95)rectangle(6+0.15/2,5);

% links neben Ständer 2

	\newcommand{\absy}{1.4cm}
	\newcommand{\absx}{0.075cm}
	\coordinate (start) at (6-0.15/2,3.5);
	\newcommand{\drawrect}{ \draw ($(start)+(0,\absy)$)rectangle($(start)+(-\absx,-\absy)$);}
\drawrect;
	\renewcommand{\absy}{1.2cm}
	\renewcommand{\absx}{0.15cm}
	\coordinate (start) at (6-0.15/2-0.075,3.5);
\drawrect;
	\renewcommand{\absy}{0.8cm}
	\renewcommand{\absx}{0.6cm}
	\coordinate (start) at (6-0.3,3.5);
\drawrect;

	\renewcommand{\absy}{0.19cm}
	\renewcommand{\absx}{0.2cm}
	\coordinate (start) at (6-0.3,3.5+1);
\drawrect;
	\coordinate (start) at (6-0.3,3.5-1);
\drawrect;

	\renewcommand{\absy}{0.14cm}
	\renewcommand{\absx}{0.15cm}
	\coordinate (start) at (6-0.5,3.5+1.04);
	\newcommand{\drawfasedrect}{
\draw (start)--++(0,\absy)--++(-\absx*.8,0)--++(-\absx*0.2,-\absy*0.2)--++(0,-2*\absy*0.8)--++(\absx*0.2,-\absy*0.2)--++(\absx*0.8,0)--cycle;
\draw (start)++(-\absx,.6*\absy)--++(\absx,0);
\draw (start)++(-\absx,-.6*\absy)--++(\absx,0);}
\drawfasedrect;
	\renewcommand{\absy}{0.14cm}
	\renewcommand{\absx}{0.15cm}
	\coordinate (start) at (6-0.5,3.5-1.04);
\drawfasedrect;
\end{scope}

%% Bauteil 2
\draw (3.7,3.5-0.15)rectangle(4.1,3.65);
\draw[rounded corners=0.1mm] (4.1,3.3)rectangle(4.15,3.7);
\draw[rounded corners=0.1mm] (4.15,3.32)rectangle(4.2,3.68);
\draw[fill=black] (4.15,3.4)rectangle(4.2,3.45);
\draw[rounded corners=0.1mm] (4.2,3.3)rectangle(4.25,3.7);

\draw (4.25,3.2)--(4.9,3.2)--(4.9,4)--(4.4,4)--(4.25,4-.15)--cycle;
\draw[rounded corners=.3mm] (4.4,4)rectangle(4.9,4.1);

\coordinate (start) at (4.4-.15/2,4-.15/2);
\newcommand{\absx}{.13cm}
\newcommand{\absy}{.04cm}
\newcommand{\drawrotatedrect}{
\draw[rounded corners = 0.15mm] (start)--++(\absx/2,\absx/2)--++(-\absy,\absy)--++(-\absx/2,-\absx/2)coordinate(start)--++(-\absx/2,-\absx/2)--++(\absy,-\absy)--cycle;}
\drawrotatedrect;
	\renewcommand{\absx}{.11cm}
	\renewcommand{\absy}{.04cm}
\drawrotatedrect;
	\renewcommand{\absx}{.13cm}
\drawrotatedrect;
	\renewcommand{\absx}{.05cm}
	\renewcommand{\absy}{0.2cm}
\drawrotatedrect;

\draw[rounded corners=0.1mm] (4.9,3.25)rectangle(4.95,3.75);
\draw[rounded corners=0.1mm] (4.95,3.3)rectangle(5,3.7);
\draw[fill=black] (4.95,3.4)rectangle(5,3.45);
\draw[rounded corners=0.1mm] (5,3.25)rectangle(5.2,3.75);
\draw[rounded corners=0.2mm] (5.2,3.3)rectangle(5.4,3.7);
\draw (5.4,3.5-0.15)rectangle(5.6,3.65);



%%_____________________________________________________________________
% right side of the picture
%%_____________________________________________________________________

\draw [dashed] (16-7/2,3.5)circle(3.5);
\draw [dashed] (8-4/2,3.5)circle(0.9);
\pgfmathsetmacro{\phi}{atan(3.5*(1-.9/3.5)/(6.5))}
\draw [dashed] (8-4/2,3.5)++(\phi+90:.9)--($(16-7/2,3.5)+(\phi+90:3.5)$);
\draw [dashed] (8-4/2,3.5)++(-\phi-90:.9)--($(16-7/2,3.5)+(-\phi-90:3.5)$);

% Schraubenkopf
	\renewcommand{\absy}{0.9cm}
	\renewcommand{\absx}{0.7cm}
	\coordinate (start) at (9.9,3.5);
	\newcommand{\drawfasedrect}{
\draw[thick] (start)--++(0,\absy)--++(-\absx*.8,0)--++(-\absx*0.2,-\absy*0.2)--++(0,-2*\absy*0.8)--++(\absx*0.2,-\absy*0.2)--++(\absx*0.8,0)--cycle;
\draw (start)++(-.8*\absx,.5*\absy)--++(.8*\absx,0);
\draw (start)++(-.8*\absx,.5*\absy)to[out=150,in=-150]($(start)+(-.8*\absx,\absy)$);
\draw (start)++(-.8*\absx,-.5*\absy)--++(.8*\absx,0);
\draw (start)++(-.8*\absx,-.5*\absy)to[out=270-60,in=150]($(start)+(-.8*\absx,-\absy)$);
\draw (start)++(-.8*\absx,-.5*\absy)to[out=120,in=-120]($(start)+(-.8*\absx,.5*\absy)$);
}
\drawfasedrect;

% Schaft
\renewcommand{\absy}{0.5}
\draw[thick] ($(start)+(0,\absy)$)--++(4.5,0);
\draw[thick] ($(start)+(0,-\absy)$)--++(4.5,0);


%Mutter
\renewcommand{\absy}{0.9cm}
\renewcommand{\absx}{0.9cm}
\coordinate (start) at (15.3,3.5);
\renewcommand{\drawfasedrect}{
\draw[thick] (start)--++(0,.9*\absy)--++(-.1*\absx,.1*\absy)--++(-\absx*.8,0)--++(-\absx*0.1,-\absy*0.1)--++(0,-2*\absy*0.9)--++(\absx*0.1,-\absy*0.1)--++(\absx*0.8,0)--++(\absx*.1,.1*\absy)--cycle;
\draw (start)++(-.9*\absx,.5*\absy)--++(.8*\absx,0);
\draw (start)++(-.9*\absx,.5*\absy)to[out=135,in=-135]($(start)+(-.9*\absx,\absy)$);
\draw (start)++(-.1*\absx,.5*\absy)to[out=45,in=-45]($(start)+(-.1*\absx,\absy)$);
\draw (start)++(-.9*\absx,-.5*\absy)--++(.8*\absx,0);
\draw (start)++(-.9*\absx,-.5*\absy)to[out=-135,in=135]($(start)+(-.9*\absx,-\absy)$);
\draw (start)++(-.1*\absx,-.5*\absy)to[out=-45,in=45]($(start)+(-.1*\absx,-\absy)$);
\draw (start)++(-.9*\absx,-.5*\absy)to[out=110,in=-110]($(start)+(-.9*\absx,.5*\absy)$);
\draw (start)++(-.1*\absx,-.5*\absy)to[out=70,in=-70]($(start)+(-.1*\absx,.5*\absy)$);
 }
\drawfasedrect

% Schraubenende
\draw[thick](15.3,4)--(15.6,4)--(15.75,3.85)--++(0,-.7)--(15.6,3)--(15.3,3);
\draw (15.6,4)--++(0,-1);
\draw (15.75,3.85)--++(-.45,0);
\draw (15.75,3.15)--++(-.45,0);
% Gewinde
\draw (14.2,4)--++(0,-1);
\draw (14.2,3.85)--++(.2,0);
\draw (14.2,3.15)--++(.2,0);

% Bauteil 4
\draw[pattern=north east lines] (9.9,4.05)rectangle(10.3,4.7);
\draw[pattern=north east lines] (9.9,2.95)rectangle(10.3,2.3);
\draw (10.3,4.05)--(10.3,4);
\draw (10.3,2.95)--(10.3,3);

% Bauteil 5
\draw[pattern = north west lines] (10.3,4.05)--++(.5,0)--++(0,1.3)--++(-.8,0)--++(0,-.65)--++(.3,0)--cycle;
\draw (10.8,4.05)--(10.8,4);
\draw[pattern = north west lines] (10.3,2.95)--++(.5,0)--++(0,-1.3)--++(-.8,0)--++(0,.65)--++(.3,0)--cycle;
\draw (10.8,2.95)--(10.8,3);

% Bauteil 7
\draw[pattern = north west lines] (14.4,4.05)--++(0,.35)--++(.6,0)--++(0,.95)--++(-1.2,0)--++(0,-1.3)--cycle;
\draw[pattern = north west lines] (14.4,2.95)--++(0,-.35)--++(.6,0)--++(0,-.95)--++(-1.2,0)--++(0,1.3)--cycle;
\draw (13.8,4.05)--(13.8,4);
\draw (13.8,2.95)--(13.8,3);

% Bauteil 6

\draw[rounded corners = .5mm,thick] (13.75,4.7)--++(0,.2)--++(.05,0)++(0,.1)--++(-.05,0)--++(0,.6);
\draw[rounded corners = .5mm,thick] (10.9,4.7)--++(0,.2)--++(-.1,0)++(0,.1)--++(.05,0)--++(0,.7);
\draw[thick](13.75,4.7)--(10.9,4.7);
\draw [dashed, very thin] (10.75,3.5+1.6)--++(3.15,0);
\draw (10.85,4.7+1)--++(20:.5)--++(-5:.5)--++(-20:.5)--++(10:.5)--++(5:.5)--++(-5:.3)--(13.75,5.6);
\draw (13.75,4.7)--(13.75,4);
\draw (10.9,4.7)--(10.9,4);

\draw[rounded corners = .5mm,thick] (13.75,2.3)--++(0,-.2)--++(.05,0)++(0,-.1)--++(-.05,0)--++(0,-.6);
\draw[rounded corners = .5mm,thick] (10.9,2.3)--++(0,-.2)--++(-.1,0)++(0,-.1)--++(.05,0)--++(0,-.7);
\draw[thick](13.75,2.3)--(10.9,2.3);
\draw [dashed, very thin] (10.75,3.5-1.6)--++(3.15,0);
\draw (10.85,3.5-4.7-1+3.5)--++(-20:.5)--++(5:.5)--++(20:.5)--++(-10:.5)--++(-5:.5)--++(5:.3)--(13.75,-5.6+2*3.5);
\draw (13.75,2.3)--(13.75,3);
\draw (10.9,2.3)--(10.9,3);

%%%% Schraffierung von Bauteil 6
\def\mypath{ (10.9,4.7)--++(0,.2)--++(-.1,0)--++(0,.1)--++(.05,0)--(10.85,4.7+1)--++(20:.5)--++(-5:.5)--++(-20:.5)--++(10:.5)--++(5:.5)--++(-5:.3)--(13.75,5.6)--(13.75,4.7)--cycle}
\begin{scope}
	\path[clip]\mypath;
	\foreach \x in {-10,-9.5,...,30}
	{\draw [thin] (\x,-3)--(\x+10,7);}	
	\end{scope}

\def\mypath{ (10.9,2*3.5-4.7)--++(0,-.2)--++(-.1,0)--++(0,-.1)--++(.05,0)--(10.85,-5.7+2*3.5)--++(-20:.5)--++(5:.5)--++(20:.5)--++(-10:.5)--++(-5:.5)--++(5:.3)--(13.75,-5.6+2*3.5)--(13.75,-4.7+2*3.5)--cycle}
\begin{scope}
	\path[clip]\mypath;
	\foreach \x in {-10,-9.5,...,30}
	{\draw [thin] (\x,-3)--(\x+10,7);}	
	\end{scope}
	
% Beschriftungen 3-8
%% Beschriftung 1
\draw (1,3.7)node{$\bullet$}--(3.5,6)node[above,draw,circle]{\tiny 1};
%% Beschriftung 2
\draw (4.6,3.8)node{$\bullet$}--(5,6)node[above,draw,circle]{\tiny 2};
\draw (9.5,4.2)node{$\bullet$}--(6.5,6)node[above,draw,circle,fill=white]{\tiny 3};	
\draw (10.1,4.4)node{$\bullet$}--(8,6)node[above,draw,circle,fill=white]{\tiny 4};
\draw (10.4,5)node{$\bullet$}--(9.5,6)node[above,draw,circle,fill=white]{\tiny 5};
\draw (11.5,5.3)node{$\bullet$}--(11,6)node[above,draw,circle,fill=white]{\tiny 6};
\draw (14.2,4.4)node{$\bullet$}--(12.5,6)node[above,draw,circle,fill=white]{\tiny 7};
\draw (14.7,4.2)node{$\bullet$}--(14,6)node[above,draw,circle,fill=white]{\tiny 8};


\end{tikzpicture}
\end{document}