\documentclass[tikz]{standalone}
\usepackage{tikz}
\usepackage{tikz-3dplot}
\usetikzlibrary{positioning}

\tdplotsetmaincoords{60}{120}
\tdplotsetrotatedcoords{90}{90}{90}

\begin{document}
\begin{tikzpicture}[tdplot_rotated_coords,
	Force/.style={black, thick, latex-},
	bear/.style={thick},
	scale=2
]



%%
\path (-3,0,0) coordinate (O);
\draw[->] (O) --++(.5,0,0) node[above right, scale=.8]{$x$};
\draw[->] (O) --++(0,.5,0) node[above, scale=.8]{$y$};
\draw[->] (O) --++(0,0,.5) node[right, scale=.8]{$z$};




% Shaft
\draw[line width=.5mm] (0,0,0)coordinate(1)--(0,0,.5)coordinate(A)node[above=2mm]{A}--++(0,0,1)coordinate(F)node[below left={-1.5mm and 2mm}]{C}circle(.02)--++(0,0,3)coordinate(B)node[above=2mm]{B}--++(0,0,.5)coordinate(2);

% Bearing
\draw[bear] (A)circle(.1);
\draw[bear] (A)++(-60:.1)--++(-60:.4)--++(180:.5)coordinate(AB)--++(60:.4);
\draw[bear] (AB)++(-.2,0)--++(.9,0);
\foreach \x in {-.1, .1, .3, .5}{\draw[bear] (AB)++(\x+.1,0)--++(-.1, -.1);}

\draw[bear] (B)circle(.1);
\draw[bear] (B)++(-60:.1)--++(-60:.4)--++(180:.5)coordinate(BB)--++(60:.4);

\draw[bear] (BB)++(-.2,-.1)--++(.9,0);
\foreach \x in {-.1, .1, .3, .5}{\draw[bear] (BB)++(\x+.1,-.1)--++(-.1, -.1);}



%% Forces
\draw[Force] (F)--++(0,1,0)node[above]{$F_{y}$};
\draw[Force] (F)++(-.2,0,0)--++(0,0,1)node[below]{$F_{z}$};
\draw[Force] (F)--++(1,0,0)node[right]{$F_{x}$};
%
%\draw[latex-, very thick, yForce] (B)--++(0,\sF*\FyB,0)node[below]{$F_{yB}$};
%\draw[latex-, very thick, zForce] (B)--++(0,0,\sF*\FzB)node[left]{$F_{zB}$};


\draw[help lines] (A)--++(-1.3,0);
\draw[help lines] (F)--++(-1.3,0);
\draw[help lines] (B)--++(-1.3,0);
\draw[<->] (A)++(-1.2,0)--++(0,0,1)node[midway, below]{$\ell_1$};
\draw[<->] (F)++(-1.2,0)--++(0,0,3)node[midway, below]{$\ell_2$};


\end{tikzpicture}
\end{document}