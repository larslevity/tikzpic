


\documentclass{beamer}
\usepackage{scalefnt}

\usetheme{plain}

\usepackage{tikz}
\usepackage{ifthen}

\begin{document}

\def\la{1}		% Laenge der Arme
\def\lb{1.2}		% Laenge des Bauches

\def\N{8}		% Anzahl Chambers
\def\h{.1}		% Hoehe der Chambers


\def\pi{3.1416}

\begin{frame}{}
	\begin{flushright}
	\begin{tikzpicture}[overlay,remember picture, scale=1]
		\path (0,0)coordinate(OR)coordinate(OL);
		\path [help lines, step=.2cm] (-2,-2) grid (0.2,1.6);
	\end{tikzpicture}
	\end{flushright}
\end{frame}


\foreach[count=\i] \gam/\fp in {1/1, 20/1 40/1, 60/1, 80/1, 
								80/-1, 60/-1, 40/-1, 20/-1, 1/-1, -20/-1, -40/-1,-60/-1, -80/-1,
								-80/1, -60/1, -40/1, -20/1, 1/1, 20/1 40/1, 60/1, 80/1, 
								80/-1, 60/-1, 40/-1, 20/-1, 1/-1, -20/-1, -40/-1,-60/-1, -80/-1,
								-80/1, -60/1, -40/1, -20/1, 1/1
								}{
%\foreach[count=\i] \gam/\fp in {1/1, 40/1}{
    \begin{frame}{}
    \begin{flushright}
    \begin{tikzpicture}[overlay,remember picture, scale=1]
    	
    	% Bounding Box
    	\path [help lines, step=.2cm,gray!30] (-2,-2) grid (0.2,1.6);

		\pgfmathsetmacro{\alp}{45 + \gam*.5}	% Winkel der rechten Arme
		\pgfmathsetmacro{\bet}{45 - \gam*.5}	% Winkel der linken Arme
		\pgfmathsetmacro{\ra}{\la/\alp*180/\pi}	% Radius der rechten Arme
		\pgfmathsetmacro{\rb}{\la/\bet*180/\pi}	% Radius der rechten Arme
		\pgfmathsetmacro{\signgam}{\gam/sqrt(\gam*\gam)}
		\pgfmathsetmacro{\absgam}{\gam*\signgam}
		\pgfmathsetmacro{\rB}{\signgam*\lb*180/\pi/\absgam}	% Radius des Bauches
		\pgfmathsetmacro{\gamh}{\gam*.5}		% Gamma Halbe
		
		% Start oben rechts oder links, je nach Vorzeichen von Gamma:
		\ifthenelse{0 < \fp}{
			% Rechter oberer Arm
			\draw (OR) arc (90+\alp-\gamh:90-\gamh:-\ra) coordinate(OM);
			% Linker oberer Arm
			\draw (OM) arc (-90-\gamh:-90-\gamh-\bet:\rb) coordinate(OL);
		}{
			% Linker oberer Arm
			\draw (OL) arc (-90-\bet-\gamh:-90-\gamh:\rb) coordinate(OM);
			% Rechter oberer Arm
			\draw (OM) arc (-90-\gamh:-90-\gamh+\alp:\ra) coordinate(OR);
		}
		
		% Bauch Bauch
		\draw (OM) arc (180-\gamh:180+\gamh:\rB) coordinate(UM);
		
		% Rechter unterer Arm
		\draw (UM) arc (90+\gamh:90+\gamh-\alp:\ra) coordinate(UR);
		% Linker unterer Arm
		\draw (UM) arc (90+\gamh:90+\gamh+\bet:\rb) coordinate(UL);
		
	%%%%%%%%%%%%%%%%%%%%%% Perlen %%%%%%%%%%%%%%%%%%%%%%%%%%%%%%%%%%%%%
		\begin{scope}[help lines]
		
		% Links -- alles mit beta
		\foreach \chamber in {1, 2, ..., \N}{
			\pgfmathsetmacro{\dangle}{\bet/(\N*2)}	
			\pgfmathsetmacro{\angle}{\chamber*\bet/\N}
			\pgfmathsetmacro{\blast}{\bet}
	
			% Links Unten
			\path (UM) arc (90+\gamh:90+\gamh+\angle:\rb) coordinate(cul);
			\draw (cul) ++ (90+\gamh+\angle:\h) coordinate(col) 
						arc (90+\gamh+\angle:90+\gamh+\angle-\dangle:\rb+\h)coordinate(cor) 
						++ (90+\gamh+\angle-\dangle:-\h) coordinate(cur);
			\draw (cul) to[out=90+\gamh+\angle + \blast, in = -90+\gamh+\angle - \blast] (col);
			\draw (cur) to[out=90+\gamh+\angle - \blast, in = -90+\gamh+\angle + \blast] (cor);
			
			% Links Oben
			\path (OM) arc (-90-\gamh:-90-\gamh-\angle:\rb) coordinate(cul);
			\draw (cul) ++ (-90-\gamh-\angle:\h) coordinate(col) 
						arc (-90-\gamh-\angle:-90-\gamh-\angle+\dangle:\rb+\h)coordinate(cor) 
						++ (-90-\gamh-\angle+\dangle:-\h) coordinate(cur);
			\draw (cul) to[out=-90-\gamh-\angle - \blast, in = +90-\gamh-\angle + \blast] (col);
			\draw (cur) to[out=-90-\gamh-\angle + \blast, in = +90-\gamh-\angle - \blast] (cor);
			
		}
		% Rechts -- alles mit alpha
		\foreach \chamber in {1, 2, ..., \N}{
			\pgfmathsetmacro{\dangle}{\alp/(\N*2)}	
			\pgfmathsetmacro{\angle}{\chamber*\alp/\N}
			\pgfmathsetmacro{\blast}{\alp}
	
			% Rechts Unten
			\path (UM) arc (90+\gamh:90+\gamh-\angle:\ra) coordinate(cul);
			\draw (cul) ++ (90+\gamh-\angle:\h) coordinate(col) 
						arc (90+\gamh-\angle: 90+\gamh-\angle+\dangle:\ra+\h)coordinate(cor)
						++ (90+\gamh-\angle+\dangle:-\h) coordinate(cur);
			\draw (cul) to[out=90+\gamh-\angle - \blast, in = -90+\gamh-\angle + \blast] (col);
			\draw (cur) to[out=90+\gamh-\angle + \blast, in = -90+\gamh-\angle - \blast] (cor);
			
			% Rechts Oben
			\path (OM) arc (-90-\gamh:-90-\gamh+\angle:\ra) coordinate(cul);
			\draw (cul) ++ (-90-\gamh+\angle:\h) coordinate(col) 
						arc (-90-\gamh+\angle: -90-\gamh+\angle-\dangle:\ra+\h)coordinate(cor)
						++ (-90-\gamh+\angle-\dangle:-\h) coordinate(cur);
			\draw (cul) to[out=-90-\gamh+\angle + \blast, in = 90-\gamh+\angle - \blast] (col);
			\draw (cur) to[out=-90-\gamh+\angle - \blast, in = 90-\gamh+\angle + \blast] (cor);	
		}
		
		%% Bauch \draw (OM) arc (180-\gamh:180+\gamh:\rB) coordinate(UM);
			\foreach \chamber in {1, 2, 3, ..., 7}{
			\pgfmathsetmacro{\dangle}{\gam/(\N*2)}	
			\pgfmathsetmacro{\angle}{\chamber*\gam/\N}
			
	
			% Bauch links
			\ifthenelse{0>\gam}{\def\blast{0}}{\def\blast{\gam}}
	
			\path (OM) arc (180-\gamh:180-\gamh+\angle:\rB) coordinate(cul);
			\draw (cul) ++ (180-\gamh+\angle:\h) coordinate(col) 
						arc (180-\gamh+\angle: 180-\gamh+\angle-\dangle:\rB+\h)coordinate(cor)
						++ (180-\gamh+\angle-\dangle:-\h) coordinate(cur);
			\draw (cul) to[out=180-\gamh+\angle + \blast, in = -\gamh+\angle - \blast] (col);
			\draw (cur) to[out=190-\gamh+\angle - \blast, in = -\gamh+\angle + \blast] (cor);
			
			% Bauch rechts
			\ifthenelse{0<\gam}{\def\blast{0}}{\def\blast{\gam}}
				% Means: Wenn Bauch nach rechts biegt (gam>0). Soll rechter Bauch einen \blast von 0 haben.
	
			\path (OM) arc (180-\gamh:180-\gamh+\angle:\rB) coordinate(cul);
			\draw (cul) ++ (180-\gamh+\angle:-\h) coordinate(col) 
						arc (180-\gamh+\angle: 180-\gamh+\angle-\dangle:\rB+\h)coordinate(cor)
						++ (180-\gamh+\angle-\dangle:\h) coordinate(cur);
			\draw (cul) to[out=180-\gamh+\angle - \blast, in = -\gamh+\angle + \blast] (col);
			\draw (cur) to[out=190-\gamh+\angle + \blast, in = -\gamh+\angle - \blast] (cor);
		}
		
		\end{scope}
		
	%%%%%%%%%%%%%% Haende %%%%%%%%%%%%%%%%%%%%%%%%%%%%%%%%%%%%%%%%
		\def\R{.1}			% Radius der Hände
		
		\ifthenelse{0<\fp}{	
			\def\colA{gray}
			\def\colB{white}
		}{
			\def\colA{white}
			\def\colB{gray}
		}
		\draw[fill=\colA] (OR) ++ (\alp-\gamh:\R) circle(\R);
		\draw[fill=\colB] (OL) ++ (180-\gamh-\bet:\R) circle(\R);
		
		\draw[fill=\colB] (UR) ++ (-\alp+\gamh:\R) circle(\R);
		\draw[fill=\colA] (UL) ++ (180+\gamh+\bet:\R) circle(\R);

    \end{tikzpicture}
    \end{flushright}
    \end{frame}
} % End foreach loop


\end{document}