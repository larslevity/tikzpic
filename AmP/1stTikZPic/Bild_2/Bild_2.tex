\documentclass[12pt]{article}
%
\usepackage{tikz}
\usepackage{pgfplots}
\pgfplotsset{compat=newest} % "neue" Verbesserungen nutzen
\pgfplotsset{plot coordinates/math parser=false}
\usepgflibrary{shadings}
%
\usetikzlibrary{arrows}
\usetikzlibrary{decorations.pathmorphing}
\usetikzlibrary{backgrounds}
\usetikzlibrary{positioning}
\usetikzlibrary{fit}
\usetikzlibrary{petri}
\usetikzlibrary{trees}
\usetikzlibrary{mindmap}
\usetikzlibrary{shadows}
\usetikzlibrary{er}
\usetikzlibrary{shadings}
\usetikzlibrary{patterns}
\usetikzlibrary{shapes}
\usetikzlibrary{intersections}
\usetikzlibrary{calc,decorations.markings}
\usepackage{ifthen}
%\usetikzlibrary{}
%
\usepackage[ansinew]{inputenc}
\usepackage{verbatim}
\usepackage[active,tightpage]{preview}
\PreviewEnvironment{tikzpicture}
\setlength\PreviewBorder{0pt}%
\usepackage{amsmath,bm}
%
\usepgfplotslibrary{groupplots}
\usetikzlibrary{pgfplots.groupplots}
%


%% using !!tikzlibrary{calc}!!
%
%
%
\begin{document}
%
\begin{tikzpicture}
%Rahmen
\draw (-4,-4)rectangle(4,4);
%
\coordinate (L) at (-1.8,0);
\coordinate (R) at (1.8,0);
\coordinate (O) at (0,0);
%
%
%Korpus
\begin{scope}
	\path[clip, preaction={draw=white}] (-2,-1.131) rectangle (2,2);
	\draw[fill=black](O)circle(1.8);
	\draw[fill=white](O)circle(1.6);
\end{scope}
%
%Achsen
\draw[dash pattern=on 20pt off 2.25pt on 0.5pt off 2.25,very thin]
	(-1.25,0)--(1.25,0);
\draw[dash pattern=on 20pt off 2.25pt on 0.5pt off 2.25,very thin]
	(0,3)--(0,-2);
%Korpus_Fortsetzung
\draw[very thick]($(L)+(-109.5:1.2)$)to[out=-199.5,in=270]($(L)+(180:1.2)$)to[out=90,in=230]($(L)+(130:1.3)$)to[out=410,in=180](0,2.2)to[out=0,in=130]($(R)+(50:1.3)$)to[in=90,out=-50]($(R)+(0:1.2)$)to[out=270,in=19.5]($(R)+(289.5:1.2)$)--($(L)+(-109.5:1.2)$)--cycle;
%
%dünn
\draw[thin] (225:1.6)--(0,0)--(-45:1.6);
\draw[thin] (220:3) arc(220:140:3);
%
%Bemaßungen
\draw[thin,stealth-stealth] (225:0.8) arc(225:-45:0.8);
\node[rotate=30,style={rectangle,scale=0.8,fill=white}] at(110:0.8){$\psi$};
\draw[thin,-stealth](0,0)--(200:3);
\node[rotate=20,style={rectangle,scale=0.8,fill=white}] at(200:2.2){$r_w$};
\draw[thin,-stealth](0,0)--(30:1.6);
\node[rotate=30,style={rectangle,scale=0.8,fill=white}] at(30:1.1){$r$};
\draw[thin,stealth-] (30:1.8)--(30:2.3);
\node[rotate=30,above,style={rectangle,scale=0.8,fill=white}] at(30:2.2){$dr$};
%
\end{tikzpicture}
%
\end{document}