\usepackage{tikz}
\usepackage{pgfplots}
\pgfplotsset{compat=newest} % "neue" Verbesserungen nutzen
\pgfplotsset{plot coordinates/math parser=false}
\usepgflibrary{shadings}
%
\usetikzlibrary{arrows}
\usetikzlibrary{decorations.pathmorphing}
\usetikzlibrary{backgrounds}
\usetikzlibrary{positioning}
\usetikzlibrary{fit}
\usetikzlibrary{petri}
\usetikzlibrary{trees}
\usetikzlibrary{mindmap}
\usetikzlibrary{shadows}
\usetikzlibrary{er}
\usetikzlibrary{shadings}
\usetikzlibrary{calc}
%\usetikzlibrary{}
%
\usepackage[ansinew]{inputenc}
\usepackage{verbatim}
\usepackage[active,tightpage]{preview}
\PreviewEnvironment{tikzpicture}
\setlength\PreviewBorder{0pt}%
%
\usepgfplotslibrary{groupplots}
\usetikzlibrary{pgfplots.groupplots}
%
% Eigene Colormap, verl�uft von wei� zu schwarz
\pgfplotsset{%
colormap={whiteblack}{gray(0cm)=(1); gray(1cm)=(0)}
}%
%
% Strickst�rke der Koordinatenachsen
\pgfplotsset{%
every axis/.append style={semithick}
}%
%
% Strickst�rke der Plots
\pgfplotsset{%
every axis plot/.append style={thick}
}%
%
% Schriftgr��en in Plots
\pgfplotsset{%
tick label style={font=\small},
label style={font=\small},
legend style={font=\small}
}%
%
% "rainbow spectrum" shading
\pgfdeclareverticalshading{rainbow}{100bp}
{color(0bp)=(red); color(25bp)=(red); color(35bp)=(yellow);
color(45bp)=(green); color(55bp)=(cyan); color(65bp)=(blue);
color(75bp)=(violet); color(100bp)=(violet)}
%
