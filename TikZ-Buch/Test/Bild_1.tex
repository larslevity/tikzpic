\documentclass[11pt]{standalone}

\usepackage[utf8]{inputenc}
\usepackage[german]{babel}
\usepackage{amsmath}
\usepackage{amsfonts}
\usepackage{amssymb}
\usepackage{graphicx}
\usepackage{ifthen}
\usepackage{array}
\usepackage{eurosym}

% \usepackage{emerald}
% \newcommand{\hw}[1]{{\ECFAugie #1}}


\usepackage{tikz}
\usetikzlibrary{calc,patterns,
                 decorations.pathmorphing,
                 decorations.markings,
                 decorations.pathreplacing}
\usetikzlibrary{trees,arrows}
\usetikzlibrary{shapes.geometric}
\usetikzlibrary{positioning}

                 
\newcommand*\circled[1]{\tikz[baseline=(char.base)]{
            \node[shape=circle,draw,inner sep=2pt,solid,fill=white] (char) {#1};}}
            
            
            
\usetikzlibrary{patterns}

\pgfdeclarepatternformonly[\LineSpace]{my north east lines}{\pgfqpoint{-1pt}{-1pt}}{\pgfqpoint{\LineSpace}{\LineSpace}}{\pgfqpoint{\LineSpace}{\LineSpace}}%
{
    \pgfsetlinewidth{0.4pt}
    \pgfpathmoveto{\pgfqpoint{0pt}{0pt}}
    \pgfpathlineto{\pgfqpoint{\LineSpace + 0.1pt}{\LineSpace + 0.1pt}}
    \pgfusepath{stroke}
}


\pgfdeclarepatternformonly[\LineSpace]{my north west lines}{\pgfqpoint{-1pt}{-1pt}}{\pgfqpoint{\LineSpace}{\LineSpace}}{\pgfqpoint{\LineSpace}{\LineSpace}}%
{
    \pgfsetlinewidth{0.4pt}
    \pgfpathmoveto{\pgfqpoint{0pt}{\LineSpace}}
    \pgfpathlineto{\pgfqpoint{\LineSpace + 0.1pt}{-0.1pt}}
    \pgfusepath{stroke}
}

\newdimen\LineSpace
\tikzset{
    line space/.code={\LineSpace=#1},
    line space=3pt
}

\usepackage{blindtext}

\begin{document}

\tikz\draw(0,0)circle(1);

\begin{tikzpicture}[scale=.8]

\draw[very thick,-latex] (0,0)--(0,-2)node[below]{$S_1$};

% linke Rolle

\draw[fill=gray!60,very thick] (.5,-4)coordinate(LR) circle(.5);
\draw[very thick,-latex](LR)++(-.5,0)--++(0,1.5);
\draw[very thick,-latex](LR)++(.5,0)--++(0,1);
\draw[very thick,-latex](LR)--++(0,-2)node[below]{$A$};


% Rechte Rolle
\draw[fill=gray!60,very thick] (1.5,-1)coordinate(RR) circle(.5);
\draw (RR)--(1.5,0);
\draw[very thick,-latex](RR)++(-.5,0)--++(0,-1)node[below]{$S_2$};
\draw[very thick,-latex](RR)++(.5,0)--++(0,-2)node[below]{$S_3$}coordinate(help);

\draw[very thick,latex-latex](help)++(0,-1)--++(0,-3)node[near end,left]{$B$};

% Beschricftung

\draw[|->] (3,-.5) --++(0,-2) node[right]{$x$};
\draw[blue,->] (3.5,-.8) --++(0,-1) node[below]{$v_A$};
\draw[blue,->] (4,-1.8) --++(0,1) node[above]{$v_A$};

\draw[very thick] (-.5,0)--++(2.5,0);
\foreach \x in {-0.2,.3,...,2}{
\draw (\x,0)--++(.5,.5);
}








\end{tikzpicture}
\end{document}