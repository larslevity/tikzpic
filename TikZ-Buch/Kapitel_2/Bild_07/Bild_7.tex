\documentclass[11pt]{standalone}

\usepackage[utf8]{inputenc}
\usepackage[german]{babel}
\usepackage{amsmath}
\usepackage{amsfonts}
\usepackage{amssymb}
\usepackage{graphicx}
\usepackage{ifthen}
\usepackage{booktabs}
\usepackage{multirow}
\usepackage{eurosym}

\usepackage{emerald}
\newcommand{\hw}[1]{{\ECFAugie #1}}


\usepackage{tikz}
\usetikzlibrary{calc,patterns,arrows,trees,
                decorations.pathmorphing,
                decorations.markings,
								decorations.pathreplacing}

\newcommand*\circled[1]{\tikz[baseline=(char.base)]{
            \node[shape=circle,draw,inner sep=2pt,solid,fill=white] (char) {#1};}}


\renewcommand{\familydefault}{\sfdefault}
\usepackage[scaled]{helvet}
%\usepackage{helvet}


%% Arbeitszeit:
% 13.1.16 : 1150-1250
% ges = 60min

\begin{document}
\begin{tikzpicture}[yscale=1.2,xscale=4]

%% Oberbau
\draw (0,1.5)rectangle++(.8,1.5)node[midway,align=center]{Produkt- \\ entwicklung};
\draw[-latex] (.8,2.2)--++(.2,0);
\draw[-latex] (.8,2.3)--++(.1,0)|-++(.1,2.5);

\draw (1,1.5)rectangle++(.8,1.5)node[midway,align=center]{Arbeits- \\ vorbereitung};
\draw (1,4)rectangle++(.8,1.5)node[midway,align=center]{Material- \\ wirtschaft};
\draw[-latex] (1.8,2.2)--++(.2,0);
\draw[-latex] (1.8,2.3)--++(.1,0)|-++(1.1,2.4);
\draw[-latex] (1.8,4.8)--++(1.2,0);

\draw (2,1.5)rectangle++(.8,1.5)node[midway,align=center]{Betriebsmittel- \\ entwicklung};
\draw[-latex] (2.8,2.2)--++(.2,0);
\draw[-latex] (2.8,2.3)--++(.1,0)|-++(.1,2.3);

\draw (3,1.5)rectangle++(.8,1.5)node[midway,align=center]{Investitions- \\ planung};
\draw (3,4)rectangle++(.8,1.5)node[midway,align=center]{Qualitäts- \\ management};
\draw[-latex] (3.8,2.2)--++(.2,0);
\draw[-latex] (3.8,4.8)--++(.1,0)|-++(.1,-2.5);

\draw (4,1.5)rectangle++(.8,1.5)node[midway,align=center]{Fertigung};

%%%%%%%%%%%%%%%%%%%%%%%%%%%%%%%%%%%%%%%%%%%%%
% Ebene 0
\draw (0,.5)rectangle(.8,-.6)node[midway,align=center]{\textbf{Konstruieren,} \\ \textbf{Testen usw.}};
\draw (0,-.9)rectangle(.8,-1.6)node[midway]{Ändern};
\draw (0,-2.4)rectangle++(.8,-.7)node[midway]{Ändern};
\draw (0,-4.9)rectangle++(.8,-.7)node[midway]{Ändern};

% Ebene 1
\draw (1,-.4)rectangle(1.8,-1.1)node[midway]{Planen};
\draw (1,-1.4)rectangle(1.8,-2.1)node[midway]{Ändern};
\draw (1,-2.9)rectangle++(.8,-.7)node[midway]{Ändern};
\draw (1,-4.4)rectangle++(.8,-.7)node[midway]{Anders Planen};
\draw (1,-5.4)rectangle++(.8,-.7)node[midway]{Ändern};
% Pfeile Ebene 1
\draw[-latex] (.8,-.5)--(1,-.5);
\draw[-latex, decoration={snake, amplitude = .02cm},decorate] (1,-1)--(.8,-1);
\draw[-latex, decoration={snake, amplitude = .02cm},decorate] (.8,-1.5)--(1,-1.5);
\draw[-latex, decoration={snake, amplitude = .02cm},decorate] (.8,-3)--++(.2,0);
\draw[-latex, decoration={snake, amplitude = .02cm},decorate] (1,-5)--++(-.2,0);
\draw[-latex, decoration={snake, amplitude = .02cm},decorate] (.8,-5.5)--++(.2,0);

% Ebene 2
\draw[dashed] (2,-.9)rectangle++(.8,-.7)node[midway]{Planen};
\draw (2,-1.9)rectangle++(.8,-.7)node[midway]{Planen};
\draw (2,-3.4)rectangle++(.8,-.7)node[midway]{Ändern};
\draw (2,-5.9)rectangle++(.8,-.7)node[midway]{Ändern};
% Pfeile Ebene 2
\draw[-latex,dashed] (1.8,-1)--++(.2,0);
\draw[-latex] (1.8,-2)--++(.2,0);
\draw[-latex, decoration={snake, amplitude = .02cm},decorate] (2,-2.5)--(.8,-2.5);
\draw[-latex, decoration={snake, amplitude = .02cm},decorate] (1.8,-3.5)--++(.2,0);
\draw[-latex, decoration={snake, amplitude = .02cm},decorate] (1.8,-6)--++(.2,0);

% Ebene 3
\draw[dashed] (3,-1.4)rectangle++(.8,-.7)node[midway]{Planen};
\draw[] (3,-3.9)rectangle++(.8,-.7)node[midway]{Planen};
\draw (3,-6.4)rectangle++(.8,-.7)node[midway]{Ändern};
% Pfeile Ebene 3
\draw[-latex,dashed] (2.8,-1.5)--++(.2,0);
\draw[-latex, decoration={snake, amplitude = .02cm},decorate] (2.8,-4)--++(.2,0);
\draw[-latex, decoration={snake, amplitude = .02cm},decorate] (3,-4.5)--++(-1.2,0);
\draw[-latex, decoration={snake, amplitude = .02cm},decorate] (2.8,-6.5)--++(.2,0);

% Ebene 4
\draw[dashed] (4,-1.9)rectangle++(.8,-.7)node[midway]{Fertigen};
\draw (4,-6.9)rectangle++(.8,-.7)node[midway]{\textbf{Fertigen}};
% Pfeile Ebene 4
\draw[-latex,dashed] (3.8,-2)--++(.2,0);
\draw[-latex] (3.8,-7)--++(.2,0);



%%%%% Pfeile
\draw[help lines] (.8,.5)--(5.5,.5);
\draw [help lines] (4.8,-2.6)--(5.2,-2.6);
\draw [help lines] (4.8,-7.6)--(5.5,-7.6);
\draw[very thick, -latex, dashed] (5,.5)node{$\bullet$}--++(0,-3.1);
\draw[very thick, -latex] (5.3,.5)node{$\bullet$}--++(0,-8.1)node[midway,sloped,above, align=center]{Zeit vom Projektbeginn \\ bis zur Markteinführung};
\draw[thick, dashed, latex-latex] (5,-2.6)--(5,-7.6) node[midway,sloped,above]{mögliche Zeiteinsparung};

\path (2,-.5) node[right]{\textit{Idealverlauf}};


\end{tikzpicture}
\end{document}