\documentclass[11pt]{standalone}

\usepackage[utf8]{inputenc}
\usepackage[german]{babel}
\usepackage{amsmath}
\usepackage{amsfonts}
\usepackage{amssymb}
\usepackage{graphicx}
\usepackage{ifthen}
\usepackage{array}
\usepackage{eurosym}

% \usepackage{emerald}
% \newcommand{\hw}[1]{{\ECFAugie #1}}


\usepackage{tikz}
\usetikzlibrary{calc,patterns,
                 decorations.pathmorphing,
                 decorations.markings,
                 decorations.pathreplacing}
\usetikzlibrary{trees,arrows}
\usetikzlibrary{shapes.geometric}
\usetikzlibrary{positioning}

                 
\newcommand*\circled[1]{\tikz[baseline=(char.base)]{
            \node[shape=circle,draw,inner sep=2pt,solid,fill=white] (char) {#1};}}
            
            
            
\usetikzlibrary{patterns}

\pgfdeclarepatternformonly[\LineSpace]{my north east lines}{\pgfqpoint{-1pt}{-1pt}}{\pgfqpoint{\LineSpace}{\LineSpace}}{\pgfqpoint{\LineSpace}{\LineSpace}}%
{
    \pgfsetlinewidth{0.4pt}
    \pgfpathmoveto{\pgfqpoint{0pt}{0pt}}
    \pgfpathlineto{\pgfqpoint{\LineSpace + 0.1pt}{\LineSpace + 0.1pt}}
    \pgfusepath{stroke}
}


\pgfdeclarepatternformonly[\LineSpace]{my north west lines}{\pgfqpoint{-1pt}{-1pt}}{\pgfqpoint{\LineSpace}{\LineSpace}}{\pgfqpoint{\LineSpace}{\LineSpace}}%
{
    \pgfsetlinewidth{0.4pt}
    \pgfpathmoveto{\pgfqpoint{0pt}{\LineSpace}}
    \pgfpathlineto{\pgfqpoint{\LineSpace + 0.1pt}{-0.1pt}}
    \pgfusepath{stroke}
}

\newdimen\LineSpace
\tikzset{
    line space/.code={\LineSpace=#1},
    line space=3pt
}
\usetikzlibrary{mindmap, backgrounds, shadows}

\usepackage[most]{tcolorbox}

    
%% Arbeitszeit:
% 19.12.15 : 1500-15 20 ; 1720 - 1740
% 21.12.15 : 2300--0300
% ges = 4h 40 min

\newcommand*\Circled[1]{\tikz[baseline=(char.base)]{
            \node[shape=rectangle,draw,inner sep=4pt,solid,rounded corners = .3cm,fill=white, line width = 1pt] (char) {#1};}}

\newcommand{\mybox}[2]{%
  \setbox0=\hbox{#1}%
  %\setlength{\@tempdima}{\dimexpr\wd0+2pt}%
  \begin{tcolorbox}[colframe=black,colback=white,boxrule=2pt,arc=4pt,
      left=2pt,right=2pt,top=2pt,bottom=2pt,boxsep=0pt, width=#2]
    #1
  \end{tcolorbox}
}


\begin{document}

\begin{tikzpicture}[
mindmap,
rotate=-90,
basic/.style={concept, execute at begin node=\hskip0pt  , fill=white, line width=.5ex, text=black},
concept color=black!80,
grow cyclic,
level 1/.append style={level distance=5cm,sibling angle=36},
level 2/.append style={level distance=4cm,sibling angle=45},
level 3/.append style={level distance=5cm,sibling angle=15 ,line width = .1pt}
]		





\def\riv{15.5}
\path[fill=gray!5] (0,-\riv)arc(-90:-30:\riv)--(0,0);
\def\riii{11.2}
\path[fill=gray!10] (0,-\riii)arc(-90:90:\riii);
\def\rii{6.8}
\path[fill=gray!15] (0,-\rii)arc(-90:90:\rii);
\def\ri{3.1}
\path[fill=gray!20] (0,-\ri)arc(-90:90:\ri);

\fill[gray!15] (0,-\ri)rectangle(-5.5,-\rii)node[midway,align=center,text = black]{\textbf{Abteilungen} \\ (o.ä.) hier alle \\ wesentlichen Funktionen};
\fill[gray!10] (0,-\rii)rectangle(-5.5,-\riii)node[midway,align=center,text = black]{\textbf{Gruppen} \\ (nur Beispiele)};
\fill[gray!5] (0,-\riii)rectangle(-5.5,-\riv)node[midway,align=center,text = black]{A, B, C, ... : \\ einzelne Mitarbeiter \\ gehören nicht \\ ins Organogramm};


\fill[gray!20] (0,-\ri)rectangle(-5.5,\ri);
%\draw[rounded corners = 10pt, line width = 5pt, black!80] (-90+18:\ri)arc(-90+18:90:\ri)--++(-3.7,0)--++(0,-5);
\path (180:4.5)node[align = center]{\textbf{Stabstellen} \\ ('kann' muss aber nicht)};





\path (210:3.3)node[circle,draw, line width = 2pt, minimum width = 1cm, minimum height = 1cm, text width = 2cm, fill=white]{Normung};
\path (150:3.3)node[circle,draw, line width = 2pt, minimum width = 1cm, minimum height = 1cm, text width = 2cm, align = center, fill=white]{Schutz- \\ rechte};


\node[root concept, basic]{Produktentwicklung \\  \mybox{Entwicklungsleiter}{3.5cm}} %root
child 	{ node[basic] {Forschung {\mybox{Abt.-Leiter}{2cm}}}
			child {node[basic, dashed]{... \\ {\mybox{~}{1.5cm}}}}
		}
child 	{ node[basic] {Entwicklungs- konstruktion {\mybox{Abt.-Leiter}{2cm}}}
			child {node[basic](BuK){Berechnung und Konstruktion \\ {\mybox{Gr.-Ltr.}{1.5cm}}} 
				child{node[basic]{A}} 
				child{node[basic]{B}}
				child{node[basic]{C}}
				child{node[basic]{D}} 
				child{node[basic]{E}}
				child{node[basic]{F}}				
				}
			child {node[basic](PuK){Planung und Koordination \\ {\mybox{Gr.-Ltr.}{1.5cm}}}
%				child{node[basic]{G}} 
%				child{node[basic]{H}}
%				child{node[basic]{I}}
%				child{node[basic]{J}} 
%				child{node[basic]{K}}
%				child{node[basic]{L}}			
			}
		}
child 	{ node[basic] {Musterbau {\mybox{Abt.-Leiter}{2cm}}}
			child {node[basic, dashed]{... \\ {\mybox{Gr.-Ltr.}{1.5cm}}}}
		}
child 	{ node[basic] {Versuch {\mybox{Abt.-Leiter}{2cm}}}
			child {node[basic, dashed]{... \\ {\mybox{Gr.-Ltr.}{1.5cm}}}}
		}
child 	{ node[basic] {Auftrags-konstruktion {\mybox{Abt.-Leiter}{2cm}}}
			child {node[basic]{Kupplungen \\ {\mybox{Gr.-Ltr.}{1.5cm}}}}
			child {node[basic]{Getriebe \\ {\mybox{Gr.-Ltr.}{1.5cm}}}}			
		}
%child[grow = 200]	{ node[basic] {Normung} }
%		
%child[grow = 140]	{ node[basic] {Schutzrechte} }
%		

;
\def\r{6.5}
\draw[fill=gray,opacity=.1,dashed] (BuK)--++(-90+18+36-22.5:\r)arc(-90+18+36-22.5:-90+18+36-22.5-45:\r)--(BuK);
\path (BuK)--++(-90+18+36-45:\r)node{\footnotesize Kupplungen};
\draw[fill=gray!50,opacity=.1,dashed] (BuK)--++(-90+18+36+22.5:\r)arc(-90+18+36+22.5:-90+18+36+22.5-45:\r)--(BuK);
\path (BuK)--++(-90+18+36:\r)node{\footnotesize Getriebe};


%\path[fill=gray!50,opacity=.1] (PuK)--++(+18+36-22.5:\r)arc(18+36-22.5:-45+18+36-22.5:\r)--(PuK);
%\path (PuK)--++(-90+18+36+45:\r)node{Kupplungen};
%\path[fill=gray,opacity=.1] (PuK)--++(-90+18+36+22.5:\r)arc(-90+18+36+22.5:-90+18+36+22.5-45:\r)--(PuK);
%\path (PuK)--++(-90+18+36:\r)node{Getriebe};






\end{tikzpicture}
\end{document}