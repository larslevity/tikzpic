\documentclass[11pt]{standalone}

\usepackage[utf8]{inputenc}
\usepackage[german]{babel}
\usepackage{amsmath}
\usepackage{amsfonts}
\usepackage{amssymb}
\usepackage{graphicx}
\usepackage{ifthen}
\usepackage{array}
\usepackage{eurosym}

% \usepackage{emerald}
% \newcommand{\hw}[1]{{\ECFAugie #1}}


\usepackage{tikz}
\usetikzlibrary{calc,patterns,
                 decorations.pathmorphing,
                 decorations.markings,
                 decorations.pathreplacing}
\usetikzlibrary{trees,arrows}
\usetikzlibrary{shapes.geometric}
\usetikzlibrary{positioning}

                 
\newcommand*\circled[1]{\tikz[baseline=(char.base)]{
            \node[shape=circle,draw,inner sep=2pt,solid,fill=white] (char) {#1};}}
            
            
            
\usetikzlibrary{patterns}

\pgfdeclarepatternformonly[\LineSpace]{my north east lines}{\pgfqpoint{-1pt}{-1pt}}{\pgfqpoint{\LineSpace}{\LineSpace}}{\pgfqpoint{\LineSpace}{\LineSpace}}%
{
    \pgfsetlinewidth{0.4pt}
    \pgfpathmoveto{\pgfqpoint{0pt}{0pt}}
    \pgfpathlineto{\pgfqpoint{\LineSpace + 0.1pt}{\LineSpace + 0.1pt}}
    \pgfusepath{stroke}
}


\pgfdeclarepatternformonly[\LineSpace]{my north west lines}{\pgfqpoint{-1pt}{-1pt}}{\pgfqpoint{\LineSpace}{\LineSpace}}{\pgfqpoint{\LineSpace}{\LineSpace}}%
{
    \pgfsetlinewidth{0.4pt}
    \pgfpathmoveto{\pgfqpoint{0pt}{\LineSpace}}
    \pgfpathlineto{\pgfqpoint{\LineSpace + 0.1pt}{-0.1pt}}
    \pgfusepath{stroke}
}

\newdimen\LineSpace
\tikzset{
    line space/.code={\LineSpace=#1},
    line space=3pt
}


%% Arbeitszeit:
% 13.1.16 : 1350-1420
% ges = 30min

\begin{document}
\begin{tikzpicture}

\draw[help lines] (-0.1,0)node[above right,black]{Sequentiell}--++(3.2,0)node[right,black]{Projektbeginn};

\draw[very thick] (0,0)circle(.02)--++(0,-1)to[out=-90,in = -90]++(2,-.5) to[out=90,in = 90] ++(-2,-.5);
\draw[help lines] (-.1,-2)--++(3.2,0)node[black,pos=.8,above right,align=left]{Produkt- \\ festlegung};


\foreach \y in {-2,-3,-4,-5}{
\draw[very thick,dashed] (0,\y)to[out=-90,in = -90]++(2,-.5) to[out=90,in = 90] ++(-2,-.5);
}
\path (-.1,-4.5)--++(3.2,0)node[black,pos=.8,above right,align=left]{Fertigungs- \\ vorbereitung \\ und -anlauf};
\draw[very thick,dashed,-latex] (0,-6)--++(0,-1);
\draw[help lines] (-.1,-7)--++(3.2,0)node[black,pos=.8,above right,align=left]{Markt- \\ einführung};



%%%% 
\draw[help lines] (5.9,0)node[above right,black]{Simultan}--++(3.2,0);
\draw[very thick] (6,0)circle(.02)--++(0,-1);
\foreach \y in {-1,-2,-3}{
\draw[very thick] (6,\y)to[out=-90,in = -90]++(2,-.5) to[out=90,in = 90] ++(-2,-.5);
}
\draw[help lines] (5.9,-4)--++(3.2,0);
\draw[very thick,dashed,-latex] (6,-4)--++(0,-1.5);
\draw[help lines] (5.9,-5.5)--++(3.2,0)node[black,pos=.8,above right,align=left]{Markt- \\ einführung};


\end{tikzpicture}
\end{document}