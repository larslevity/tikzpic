\documentclass[11pt]{standalone}

\usepackage[utf8]{inputenc}
\usepackage[german]{babel}
\usepackage{amsmath}
\usepackage{amsfonts}
\usepackage{amssymb}
\usepackage{graphicx}
\usepackage{ifthen}
\usepackage{booktabs}
\usepackage{multirow}
\usepackage{eurosym}

\usepackage{emerald}
\newcommand{\hw}[1]{{\ECFAugie #1}}


\usepackage{tikz}
\usetikzlibrary{calc,patterns,arrows,trees,
                decorations.pathmorphing,
                decorations.markings,
								decorations.pathreplacing}

\newcommand*\circled[1]{\tikz[baseline=(char.base)]{
            \node[shape=circle,draw,inner sep=2pt,solid,fill=white] (char) {#1};}}


\renewcommand{\familydefault}{\sfdefault}
\usepackage[scaled]{helvet}
%\usepackage{helvet}


%% Arbeitszeit:
% 13.1.16 : 1450-1520
% ges = 30 min

\begin{document}
\begin{tikzpicture}[scale=2]

\draw[help lines] (-1,0)--++(.3,0);
\draw[help lines] (-1,-1)--++(.3,0);
\draw[fill] (-.8,0)circle(.02);
\draw[-latex] (-.8,0)--++(0,-1) node[midway,left, align = right]{neue Erkenntnis, \\ Fortschritt};


\draw[dashdotted,thick] (0,.3)--++(0,-2.5);

\draw[very thick, -latex] (-1,0)node[left]{Bauform A} ..controls(-1,.3)and(.8,0).. (1,-.5);
\fill (-1,0)circle(.04);
\fill (1,-.5)++(-.1,0)--++(.1,-.05)--++(.1,.05)node[right]{Bauform B $\neq$ A}--++(-.1,0.05)--cycle;


\draw[very thick, -latex] (1,-0.5) ..controls(1,-1.2)and(-.8,-1.2).. (-1,-1);
\draw[ thick] (-1,-1)circle(.04)node[left]{Bauform C $\sim$ A};
\fill (1,-.5)++(-.1,0)--++(.1,-.05)--++(.1,.05)--++(-.1,0.05)--cycle;

\draw[very thick, -latex,dashed] (1,-1.5) ..controls(1,-2.2)and(-.8,-2.2).. (-1,-2);
\fill[white] (-1.1,-1.7)rectangle(.8,-2.3);



\draw[fill=white] (1,-1.5)++(-.1,0)--++(.1,-.05)--++(.1,.05)node[right,align = left]{künftige \\ Bauform D $\sim$ B ?}--++(-.1,0.05)--cycle;
\draw[very thick, -latex, dashed] (-1,-1) ..controls(-1,-.7)and(.8,-1).. (1,-1.5);


\draw[fill=white,thick] (-1,-1)circle(.04);

\end{tikzpicture}
\end{document}