\documentclass[11pt]{standalone}

\usepackage[utf8]{inputenc}
\usepackage[german]{babel}
\usepackage{amsmath}
\usepackage{amsfonts}
\usepackage{amssymb}
\usepackage{graphicx}
\usepackage{ifthen}
\usepackage{array}
\usepackage{eurosym}

% \usepackage{emerald}
% \newcommand{\hw}[1]{{\ECFAugie #1}}


\usepackage{tikz}
\usetikzlibrary{calc,patterns,
                 decorations.pathmorphing,
                 decorations.markings,
                 decorations.pathreplacing}
\usetikzlibrary{trees,arrows}
\usetikzlibrary{shapes.geometric}
\usetikzlibrary{positioning}

                 
\newcommand*\circled[1]{\tikz[baseline=(char.base)]{
            \node[shape=circle,draw,inner sep=2pt,solid,fill=white] (char) {#1};}}
            
            
            
\usetikzlibrary{patterns}

\pgfdeclarepatternformonly[\LineSpace]{my north east lines}{\pgfqpoint{-1pt}{-1pt}}{\pgfqpoint{\LineSpace}{\LineSpace}}{\pgfqpoint{\LineSpace}{\LineSpace}}%
{
    \pgfsetlinewidth{0.4pt}
    \pgfpathmoveto{\pgfqpoint{0pt}{0pt}}
    \pgfpathlineto{\pgfqpoint{\LineSpace + 0.1pt}{\LineSpace + 0.1pt}}
    \pgfusepath{stroke}
}


\pgfdeclarepatternformonly[\LineSpace]{my north west lines}{\pgfqpoint{-1pt}{-1pt}}{\pgfqpoint{\LineSpace}{\LineSpace}}{\pgfqpoint{\LineSpace}{\LineSpace}}%
{
    \pgfsetlinewidth{0.4pt}
    \pgfpathmoveto{\pgfqpoint{0pt}{\LineSpace}}
    \pgfpathlineto{\pgfqpoint{\LineSpace + 0.1pt}{-0.1pt}}
    \pgfusepath{stroke}
}

\newdimen\LineSpace
\tikzset{
    line space/.code={\LineSpace=#1},
    line space=3pt
}


%% Arbeitszeit:
% 11.1.16 : 1530-1605
% ges = 35min


\begin{document}
\newcommand{\Circled}[2]{\tikz\node[rectangle, rounded corners=.2cm, draw,minimum height = 0cm, minimum width = 0cm, #2]{#1};}


\begin{tikzpicture}[
edge from parent fork down,
every node/.style={rectangle, draw, align=center},
Grp/.style={minimum width = 3cm, minimum height = 5.1cm, align = left},
level 1/.append style={level distance=6cm,sibling distance=3.5cm},
level 2/.append style={level distance=3cm,sibling distance=2.5cm},
]


\node[align=center]{ Leitung \\ Produktentwicklung \\ \Circled{Entwicklungsleiter}{}}
	child { node[Grp]{Forschung \\ \\ \Circled{Gruppenltr.}{} \\ \\ \\ \\ \tikz\draw[dashed](0,0)circle(.3); \\ \\ ~} }
	child { node[Grp]{Entwicklungs- \\ konstruktion \\ \Circled{Gruppenltr.}{} \\ \\ \Circled{Projektltr. K11}{dashed} \\ \\ \tikz\draw[dashed](0,0)circle(.3); \\ \\ ~}}
	child { node[Grp]{Musterbau \\ \\ \Circled{Gruppenltr.}{} \\ \\ \\ \\ \tikz\draw[dashed](0,0)circle(.3); \\ \\ ~}}
	child { node[Grp]{Versuch \\ \\ \Circled{Gruppenltr.}{} \\ \\ \\ \\ \tikz\draw[dashed](0,0)circle(.3); \\ \\ ~}}
	child { node[Grp]{Auftrags- \\ konstruktion \\ \Circled{Gruppenltr.}{} \\ \\ \\ \\ \tikz\draw[dashed](0,0)circle(.3); \\ \\ ~}}
	;


\draw[dashed] (0,-2)--++(2,0)node[right,dashed,inner sep = .4cm]{\Circled{Projektltr. K11}{dashed, inner sep =.1cm}};

\draw[dashed] (-11.5,-6.5)node[below right, align = left, draw = none]{Projekt \\ Kupplung K11}--++(21,0)--++(.5,-.5)node[right,align = left, draw = none]{Fertigung \\ usw.}--++(-.5,-.5)--++(-21,0)--++(0,1);

\draw[dashed] (9.6,-4) circle(.3);
\path (10,-4) node[right, align = left, draw = none]{Für das \\ Projekt tätige \\ Mitarbeiter};
\end{tikzpicture}
\end{document}