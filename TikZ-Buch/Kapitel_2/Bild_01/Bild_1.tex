\documentclass[11pt]{standalone}

\usepackage[utf8]{inputenc}
\usepackage[german]{babel}
\usepackage{amsmath}
\usepackage{amsfonts}
\usepackage{amssymb}
\usepackage{graphicx}
\usepackage{ifthen}
\usepackage{array}
\usepackage{eurosym}

% \usepackage{emerald}
% \newcommand{\hw}[1]{{\ECFAugie #1}}


\usepackage{tikz}
\usetikzlibrary{calc,patterns,
                 decorations.pathmorphing,
                 decorations.markings,
                 decorations.pathreplacing}
\usetikzlibrary{trees,arrows}
\usetikzlibrary{shapes.geometric}
\usetikzlibrary{positioning}

                 
\newcommand*\circled[1]{\tikz[baseline=(char.base)]{
            \node[shape=circle,draw,inner sep=2pt,solid,fill=white] (char) {#1};}}
            
            
            
\usetikzlibrary{patterns}

\pgfdeclarepatternformonly[\LineSpace]{my north east lines}{\pgfqpoint{-1pt}{-1pt}}{\pgfqpoint{\LineSpace}{\LineSpace}}{\pgfqpoint{\LineSpace}{\LineSpace}}%
{
    \pgfsetlinewidth{0.4pt}
    \pgfpathmoveto{\pgfqpoint{0pt}{0pt}}
    \pgfpathlineto{\pgfqpoint{\LineSpace + 0.1pt}{\LineSpace + 0.1pt}}
    \pgfusepath{stroke}
}


\pgfdeclarepatternformonly[\LineSpace]{my north west lines}{\pgfqpoint{-1pt}{-1pt}}{\pgfqpoint{\LineSpace}{\LineSpace}}{\pgfqpoint{\LineSpace}{\LineSpace}}%
{
    \pgfsetlinewidth{0.4pt}
    \pgfpathmoveto{\pgfqpoint{0pt}{\LineSpace}}
    \pgfpathlineto{\pgfqpoint{\LineSpace + 0.1pt}{-0.1pt}}
    \pgfusepath{stroke}
}

\newdimen\LineSpace
\tikzset{
    line space/.code={\LineSpace=#1},
    line space=3pt
}


%% Arbeitszeit:
% 19.12.15 : - 1400-1440
% ges = 40 min

\begin{document}
\begin{tikzpicture}

\def\rect{rectangle, draw, minimum height = 4cm, minimum width = 1.5cm}


\path (0,0)node(G)[rectangle, draw, minimum height = 1.5cm, minimum width = 3cm]{Geschäftsleitung};

\foreach \x/\t in {-5/{Produkt- \\ entwick- \\ lung},-3/{Einkauf, \\ Material- \\ wirtschaft},-1/{Fertigung},1/{Verkauf, \\ Marketing},3/{Qualitäts- \\ managment},5/ Finanzen,7/Personal}{
\draw (\x,-3)node{$\bullet$}--++(0,-.3)node[below, align= center]{\t}coordinate(help)--++(0.9,0)--++(0,-5)coordinate(1);
\draw (help)--++(-.9,0)--++(0,-5)coordinate(2);
\draw[decorate,decoration={zigzag,segment length = .8cm,amplitude=.05cm}] (1)--(2);
}
\draw (-5,-3)--(7,-3);
\draw (G)--(0,-3)node[midway]{$\bullet$}coordinate[pos=.5](help);
\draw (help)--++(2,0)node[right,rectangle, draw, minimum height = 1.5cm, minimum width = 3cm,align = center]{Stabsstellen, z.B. \\ Schutzrechte};


\draw[dashed] (-8.5,-5)node[below right, align = left]{z.B. \\ Kupplungen}rectangle(8.5,-6);


\draw[dashed] (-8.5,-6.5)node[below right, align = left]{z.B. \\ Getriebe}rectangle(8.5,-7.5);


\draw[very thick, latex-] (-7.5,-4.5)--++(0,.5)node[above, align = center]{Gliederung nach \\ \textbf{Produktarten} \\ (horizontal)};


\draw[very thick, latex-] (-3,-1)--++(-.5,0)node[left, align = center]{Gliederung nach \\ \textbf{Funktionsbereichen} \\ (vertikal)};




\end{tikzpicture}
\end{document}