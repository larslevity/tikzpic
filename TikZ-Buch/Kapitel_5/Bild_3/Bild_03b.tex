\documentclass[11pt]{standalone}

\usepackage[utf8]{inputenc}
\usepackage[german]{babel}
\usepackage{amsmath}
\usepackage{amsfonts}
\usepackage{amssymb}
\usepackage{graphicx}
\usepackage{ifthen}
\usepackage{array}
\usepackage{eurosym}

% \usepackage{emerald}
% \newcommand{\hw}[1]{{\ECFAugie #1}}


\usepackage{tikz}
\usetikzlibrary{calc,patterns,
                 decorations.pathmorphing,
                 decorations.markings,
                 decorations.pathreplacing}
\usetikzlibrary{trees,arrows}
\usetikzlibrary{shapes.geometric}
\usetikzlibrary{positioning}

                 
\newcommand*\circled[1]{\tikz[baseline=(char.base)]{
            \node[shape=circle,draw,inner sep=2pt,solid,fill=white] (char) {#1};}}
            
            
            
\usetikzlibrary{patterns}

\pgfdeclarepatternformonly[\LineSpace]{my north east lines}{\pgfqpoint{-1pt}{-1pt}}{\pgfqpoint{\LineSpace}{\LineSpace}}{\pgfqpoint{\LineSpace}{\LineSpace}}%
{
    \pgfsetlinewidth{0.4pt}
    \pgfpathmoveto{\pgfqpoint{0pt}{0pt}}
    \pgfpathlineto{\pgfqpoint{\LineSpace + 0.1pt}{\LineSpace + 0.1pt}}
    \pgfusepath{stroke}
}


\pgfdeclarepatternformonly[\LineSpace]{my north west lines}{\pgfqpoint{-1pt}{-1pt}}{\pgfqpoint{\LineSpace}{\LineSpace}}{\pgfqpoint{\LineSpace}{\LineSpace}}%
{
    \pgfsetlinewidth{0.4pt}
    \pgfpathmoveto{\pgfqpoint{0pt}{\LineSpace}}
    \pgfpathlineto{\pgfqpoint{\LineSpace + 0.1pt}{-0.1pt}}
    \pgfusepath{stroke}
}

\newdimen\LineSpace
\tikzset{
    line space/.code={\LineSpace=#1},
    line space=3pt
}


%% Arbeitszeit:
% 12.02.16 : 0105 - 0120
% ges = 15 min

\begin{document}
\begin{tikzpicture}[scale=1]

\path (0,0)node(P)[rectangle,draw]{Problem};

\path (-4,-3)node(L)[rectangle,draw]{Lösungsidee 1};

\path (0,-3)node(LL)[rectangle,draw]{Lösungsidee 2};

\path (4,-3)node(LLL)[rectangle,draw]{Lösungsidee 3};

\path (7,-3)node(U)[rectangle,draw,dotted,minimum height=.6cm]{~~~~~~~~};



\path (-.5,-.8)node(1)[circle,fill=black]{~};


\path (-.1,-1.2)node(2)[circle,fill=black]{~};

\path (1,-1.3)node(3)[circle,fill=black]{~};

\path (-2,-1.8)node(4)[circle,fill=black]{~};

\path (.5,-1.7)node(5)[circle,fill=black]{~};

\path (3.7,-1.9)node(6)[circle,fill=black]{~};

\path (-3.5,-2.2)node(7)[circle,fill=black]{~};


\path (-.2,-2.3)node(8)[circle,fill=black]{~};

\path (4.5,-2.2)node(9)[circle,fill=black]{~};



\draw[-latex] (P)--(1);
\draw[-latex] (1)--(2);
\draw[-latex] (1)--(3);
\draw[-latex] (2)--(4);
\draw[-latex] (2)--(5);
\draw[-latex] (2)--(6);
\draw[-latex] (3)--(5);
\draw[-latex] (3)--(LLL);
\draw[-latex] (4)--(LL);
\draw[-latex] (4)--(7);
\draw[-latex] (5)--(8);
\draw[-latex] (6)--(9);
\draw[-latex] (7)--(L);
\draw[-latex] (8)--(LL);
\draw[dotted,-latex] (9)--(U);








\end{tikzpicture}
\end{document}