\documentclass[11pt]{standalone}

\usepackage[utf8]{inputenc}
\usepackage[german]{babel}
\usepackage{amsmath}
\usepackage{amsfonts}
\usepackage{amssymb}
\usepackage{graphicx}
\usepackage{ifthen}
\usepackage{booktabs}
\usepackage{multirow}
\usepackage{eurosym}

\usepackage{emerald}
\newcommand{\hw}[1]{{\ECFAugie #1}}


\usepackage{tikz}
\usetikzlibrary{calc,patterns,arrows,trees,
                decorations.pathmorphing,
                decorations.markings,
								decorations.pathreplacing}

\newcommand*\circled[1]{\tikz[baseline=(char.base)]{
            \node[shape=circle,draw,inner sep=2pt,solid,fill=white] (char) {#1};}}


\renewcommand{\familydefault}{\sfdefault}
\usepackage[scaled]{helvet}
%\usepackage{helvet}


%% Arbeitszeit:
% 20.01.16 : 1620 - 1700
% ges = 40 min

\begin{document}
\begin{tikzpicture}[
edge from parent fork down,
every node/.style={rectangle, draw, align=center},
Grp/.style={minimum width = 3.5cm, minimum height = 1.2cm, align = left},
A/.style={rotate=90, minimum width = 2.4cm, minimum height = 1.1cm, align = left},
B/.style={rotate=90, draw = none},
level 1/.append style={level distance=2cm,sibling distance=4cm},
level 2/.append style={level distance=2.9cm,sibling distance=1.3cm},
level 3/.append style={level distance=3.3cm,sibling distance=.7cm},
]


\node[draw=none]{~}
	child { node[Grp]{Funktion}
		child{ node[A] {Einstell- \\ genauigkeit}}
		child{ node[A] {Wiederhol- \\ genauigkeit}}
		child{ node[A] {Verschleiß- \\ sicherheit}}
	}
	child { node[Grp]{niedrige Kosten}
		child{ node[A] {Herstell- \\ kosten}
			child { node[B]{Teilelieferant}}
			child { node[B]{Montage}}
		}
		child{ node[A] {Betriebs- \\ kosten}}	
	}
	child { node[Grp]{Markt}
		child{ node[A] {Markt- \\ größe}}
		child{ node[A] {Markt- \\ trend}}
		child{ node[A] {geringe \\ Konkurrenz}}	
	}
	child { node[Grp]{Unternehmensziele}
		child{ node[A] {Firmen- \\ Programm}}
		child{ node[A] {Kapazitäts- \\ auslastung}
			child { node[B]{Fertigung}}
			child { node[B]{Entwicklung}}
		}	
	}
	;

\path [fill=white] (-1,0)rectangle(1,-.98);


\end{tikzpicture}
\end{document}