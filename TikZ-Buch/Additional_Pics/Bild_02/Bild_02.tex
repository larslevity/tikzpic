\documentclass[11pt]{standalone}

\usepackage[utf8]{inputenc}
\usepackage[german]{babel}
\usepackage{amsmath}
\usepackage{amsfonts}
\usepackage{amssymb}
\usepackage{graphicx}
\usepackage{ifthen}
\usepackage{array}
\usepackage{eurosym}

% \usepackage{emerald}
% \newcommand{\hw}[1]{{\ECFAugie #1}}


\usepackage{tikz}
\usetikzlibrary{calc,patterns,
                 decorations.pathmorphing,
                 decorations.markings,
                 decorations.pathreplacing}
\usetikzlibrary{trees,arrows}
\usetikzlibrary{shapes.geometric}
\usetikzlibrary{positioning}

                 
\newcommand*\circled[1]{\tikz[baseline=(char.base)]{
            \node[shape=circle,draw,inner sep=2pt,solid,fill=white] (char) {#1};}}
            
            
            
\usetikzlibrary{patterns}

\pgfdeclarepatternformonly[\LineSpace]{my north east lines}{\pgfqpoint{-1pt}{-1pt}}{\pgfqpoint{\LineSpace}{\LineSpace}}{\pgfqpoint{\LineSpace}{\LineSpace}}%
{
    \pgfsetlinewidth{0.4pt}
    \pgfpathmoveto{\pgfqpoint{0pt}{0pt}}
    \pgfpathlineto{\pgfqpoint{\LineSpace + 0.1pt}{\LineSpace + 0.1pt}}
    \pgfusepath{stroke}
}


\pgfdeclarepatternformonly[\LineSpace]{my north west lines}{\pgfqpoint{-1pt}{-1pt}}{\pgfqpoint{\LineSpace}{\LineSpace}}{\pgfqpoint{\LineSpace}{\LineSpace}}%
{
    \pgfsetlinewidth{0.4pt}
    \pgfpathmoveto{\pgfqpoint{0pt}{\LineSpace}}
    \pgfpathlineto{\pgfqpoint{\LineSpace + 0.1pt}{-0.1pt}}
    \pgfusepath{stroke}
}

\newdimen\LineSpace
\tikzset{
    line space/.code={\LineSpace=#1},
    line space=3pt
}


%% Arbeitszeit:
% 30.03.16 : 2150 - 2210
% 02:04:16 : 1050 - 1111
% ges = 41 min

\begin{document}
\begin{tikzpicture}[scale=1]



\fill[left color=white, right color=white, middle
        color=gray!50] (0,0) rectangle (2.5,-1.2) node[midway, below]{kurzfristig} ;
\fill[left color=white, right color=white, middle
        color=gray!50] (2.5,0) rectangle (6,-1.2) node[midway, below]{langfristig} ;
\fill[left color=white, right color=white, middle
        color=gray!50] (6,0) rectangle (12,-1.2) node[midway, below]{Strategische Planung} ;



\draw[-latex] (0,0)--(11.5,0)node[below]{Zeit [Jahre]};
\draw[-latex] (0,0)--(0,5)node[sloped, above left, pos=.9]{Möglichkeiten};

\foreach \x in {0,1,...,10}{
\ifthenelse{0=\x}{\def\te{heute}}{\def\te{\x}}
\draw(\x,0)--++(0,-.2)node[below]{\te};
}


\draw[thick] (0,0)to[out=1, in = 215] (10,5);

\begin{scope}[dashed]


\draw (0,.9)coordinate(x)-|(4,0);
\draw[solid] (x)circle(.025)--++(45:.5)node[right,fill=white,opacity = .8]{Neues Produkt / neue Maschinen};


\draw (0,.15)coordinate(x)-|(1.5,0);
\draw[solid] (x)circle(.025)--++(45:.5)node[right,fill=white,opacity = .8]{Produkt- / Verfahrensverbesserung};

\draw (0,4.3)coordinate(x)-|(9,0);
\draw[solid] (x)circle(.025)--++(-60:.7)node[right,fill=white,opacity = .8,align=left]{Neue Produktgruppe / \\ neuer Fertigungsbetrieb};

\draw[solid] (9.5,4.65)circle(.025)--++(-70:.8)node[below,fill=white,opacity = .8,align=left]{Planungsspielraum, \\ gleichzeitig \\ Planungsunsicherheit};


\end{scope}

\end{tikzpicture}
\end{document}