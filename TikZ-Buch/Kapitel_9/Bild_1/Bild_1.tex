\documentclass[11pt]{standalone}

\usepackage[utf8]{inputenc}
\usepackage[german]{babel}
\usepackage{amsmath}
\usepackage{amsfonts}
\usepackage{amssymb}
\usepackage{graphicx}
\usepackage{ifthen}
\usepackage{array}
\usepackage{eurosym}

% \usepackage{emerald}
% \newcommand{\hw}[1]{{\ECFAugie #1}}


\usepackage{tikz}
\usetikzlibrary{calc,patterns,
                 decorations.pathmorphing,
                 decorations.markings,
                 decorations.pathreplacing}
\usetikzlibrary{trees,arrows}
\usetikzlibrary{shapes.geometric}
\usetikzlibrary{positioning}

                 
\newcommand*\circled[1]{\tikz[baseline=(char.base)]{
            \node[shape=circle,draw,inner sep=2pt,solid,fill=white] (char) {#1};}}
            
            
            
\usetikzlibrary{patterns}

\pgfdeclarepatternformonly[\LineSpace]{my north east lines}{\pgfqpoint{-1pt}{-1pt}}{\pgfqpoint{\LineSpace}{\LineSpace}}{\pgfqpoint{\LineSpace}{\LineSpace}}%
{
    \pgfsetlinewidth{0.4pt}
    \pgfpathmoveto{\pgfqpoint{0pt}{0pt}}
    \pgfpathlineto{\pgfqpoint{\LineSpace + 0.1pt}{\LineSpace + 0.1pt}}
    \pgfusepath{stroke}
}


\pgfdeclarepatternformonly[\LineSpace]{my north west lines}{\pgfqpoint{-1pt}{-1pt}}{\pgfqpoint{\LineSpace}{\LineSpace}}{\pgfqpoint{\LineSpace}{\LineSpace}}%
{
    \pgfsetlinewidth{0.4pt}
    \pgfpathmoveto{\pgfqpoint{0pt}{\LineSpace}}
    \pgfpathlineto{\pgfqpoint{\LineSpace + 0.1pt}{-0.1pt}}
    \pgfusepath{stroke}
}

\newdimen\LineSpace
\tikzset{
    line space/.code={\LineSpace=#1},
    line space=3pt
}


%% Arbeitszeit:
% 14.01.16 : 1330-1410
% ges = 40 min

\begin{document}
\begin{tikzpicture}[xscale=6, yscale=4]

\def\mypath{(.1,.2)--(.9,.5)}
\draw[very thick] \mypath node[right, align = center]{Stand der Technik};
\draw[thick, dash pattern=on 10pt off 2 pt](.1,.4)--(.9,.7)node[right]{Erfindungshöhe};

\foreach[count = \i] \pos/\dy in {.4/.1,.433/.05, .466/.12, .5/.3, .533/.04, .566/.11, .6/.03, .633/.17 }{
	\path \mypath coordinate[pos=\pos](X);
	\draw (X)--++(0,\dy)coordinate(X);
	\fill (X)circle(0.01)coordinate(\i);
	\ifthenelse{\NOT \i=4}{
	\draw[help lines] (\i)--(.7,.2);
	}{}	
	}

\pgfmathsetmacro{\y}{.1 + (.5-.2)/.8*.4+.05}
\pgfmathsetmacro{\ys}{6/4}
\draw[dashed,yscale=\ys] (.515,\y)circle(.18) (.515,\y-.18)--(.515,0)node[below,align = center]{betrachteter \\ Zeitpunkt};



\draw[-latex] (0,0)--++(1,0)node[below left]{Zeit};
\draw[-latex] (0,0)--++(0,1)node[pos=.99,sloped, above left, align = center]{Summe bzw. \\ Höhe von Ideen};


\path[draw, help lines] (4)--++(60:.1)node[above,black]{Patentfähige Idee};
\path (.7,.2) node[right, align = center]{Nicht patentfähige Ideen}; 

\end{tikzpicture}
\end{document}