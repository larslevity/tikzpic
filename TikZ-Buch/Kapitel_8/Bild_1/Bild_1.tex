\documentclass[11pt]{standalone}

\usepackage[utf8]{inputenc}
\usepackage[german]{babel}
\usepackage{amsmath}
\usepackage{amsfonts}
\usepackage{amssymb}
\usepackage{graphicx}
\usepackage{ifthen}
\usepackage{booktabs}
\usepackage{multirow}
\usepackage{eurosym}

\usepackage{emerald}
\newcommand{\hw}[1]{{\ECFAugie #1}}


\usepackage{tikz}
\usetikzlibrary{calc,patterns,arrows,trees,
                decorations.pathmorphing,
                decorations.markings,
								decorations.pathreplacing}

\newcommand*\circled[1]{\tikz[baseline=(char.base)]{
            \node[shape=circle,draw,inner sep=2pt,solid,fill=white] (char) {#1};}}


\renewcommand{\familydefault}{\sfdefault}
\usepackage[scaled]{helvet}
%\usepackage{helvet}


%% Arbeitszeit:
% 15.1.16 : 1600-1700
% ges = 60 min

\begin{document}
\begin{tikzpicture}[scale=.4]

\path (22,10)node[gray!60,scale=2,align=center]{Äußere \\ Merkmale};
\path (22,4.5)node[gray!60,scale=2,align=center]{Innere Wesens- \\ bereiche};
\path (22,0)node[gray!60,scale=2,align=center]{Tiefenbereich};



\begin{scope}

\path[clip] (-8.5,13)--(8.5,13)--(8.5,-2.5)--(13,-2.5)--(13,-7.5)--(-8.5,-7.5)--cycle;

\foreach \r/\t in {11/Leistung u. Handeln, 10/Handschrift, 9/Sprache, 8/Mimik u. Gestik,  7/Körperbau,6/Sinne,5/Denken, 4/Fühlen, 3/ Wollen,  2/Erleben, 1/{Wesenskern (unbewusst) \\ (Persönlichkeit, Selbst, "Es")}}{
\pgfmathsetmacro{\brightness}{(7-\r)/7*100}
\ifthenelse{0 > \brightness}{\def\brightness{0}}{}
\ifthenelse{\r=6}{\def\opt{very thick}}{\def\opt{}}
\draw[fill=black!\brightness, \opt] (0,0)circle(\r+1);
\draw[very thick,latex-,black] (100:\r+1)--(100:\r+2);
}

\end{scope}


\foreach \r/\t in {11/Leistung u. Handeln, 10/Handschrift, 9/Sprache, 8/Mimik u. Gestik,  7/Körperbau,6/Sinne,5/Denken, 4/Fühlen, 3/ Wollen,  2/Erleben, 1/{Wesenskern (unbewusst)}}{
\draw[help lines] (0,\r+1-.5)coordinate(X)--++(9,0)node[right,black]{\t};
\draw[fill=white] (X)circle(.15);
}
\path (9,.5)node[right]{(Persönlichkeit,};
\path (9,-.5)node[right]{Selbst, Es)};



%%% 
\draw (0,7)--++(30,0);
\draw (0,2)--++(30,0);
\draw (0,-2)--++(30,0);


\path (100:13)node[above, align = center]{Durchgängigkeit der \\ Wesenseigenschaften (typengebunden)};

\draw (0,7)--++(-10,0);
\draw[latex-] (-9.9,0)--++(0,7)node[sloped,above,midway,align=right]{zunehmender Anteil \\ des Unbewussten};



\end{tikzpicture}
\end{document}