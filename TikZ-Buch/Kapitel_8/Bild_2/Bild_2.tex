\documentclass[11pt]{standalone}

\usepackage[utf8]{inputenc}
\usepackage[german]{babel}
\usepackage{amsmath}
\usepackage{amsfonts}
\usepackage{amssymb}
\usepackage{graphicx}
\usepackage{ifthen}
\usepackage{booktabs}
\usepackage{multirow}
\usepackage{eurosym}

\usepackage{emerald}
\newcommand{\hw}[1]{{\ECFAugie #1}}


\usepackage{tikz}
\usetikzlibrary{calc,patterns,arrows,trees,
                decorations.pathmorphing,
                decorations.markings,
								decorations.pathreplacing}

\newcommand*\circled[1]{\tikz[baseline=(char.base)]{
            \node[shape=circle,draw,inner sep=2pt,solid,fill=white] (char) {#1};}}


\renewcommand{\familydefault}{\sfdefault}
\usepackage[scaled]{helvet}
%\usepackage{helvet}


%% Arbeitszeit:
% 15.1.16 : 1700 - 1735
% ges = 35 min

\begin{document}
\begin{tikzpicture}[scale=4.2]


\foreach[count=\i] \t in {{ungenau \\ wirklich- \\ keitsfremd}, {idealistisch \\ wertstrebend}, {fantasielos \\ stur}, {sachlich \\ beharrlich}, {konsequent \\ kristallklar}, {unanschaulich \\ kalt}, {präzise \\ realistisch}, {materialistisch \\ berechnend}, {fantasievoll \\ beweglich}, {unsachlich \\ sprunghaft}, {inkonsequent \\ verschwommen}, {anschaulich \\ einfühlend}} {
\ifthenelse{\i>4}{
\ifthenelse{\i>10}{\pgfmathsetmacro{\x}{mod(\i,2)}}{
\pgfmathsetmacro{\x}{mod(\i+1,2)}}}{
\pgfmathsetmacro{\x}{mod(\i,2)}}
\ifthenelse{1=\x}{\def\s{-} \def\opt{-latex}}{\def\s{+}\def\opt{-<<}}
\pgfmathsetmacro{\a}{360/24+(\i-1)*360/12}
\path (\a:1)node[align=center]{\Large{\textbf{\s}} \\ \t};
\draw[\opt] (0,0)--(\a:.7);
}


\draw[help lines] (0,0)circle(1.4);
\foreach \a/\t in {0/organisch,60/logisch,120/abstrakt,180/detailorientiert,240/intuitiv, 300/anschaulich}{
\draw[help lines] (0,0)--(\a:1.5);
\path (\a+30:1.5)--++(\a+120:.01)node[sloped,midway]{\t};
}

\draw[very thick,double distance = 0.1cm,line cap=round] (100:1.7) arc (100:145:1.7);
\path(122.5:1.8)--++(122.5+90:.01)node[sloped,midway]{2. Schwerpunkt};

\draw[very thick,double distance = 0.1cm,line cap=round] (280:1.7) arc (280:145+180:1.7);
\path(122.5+180:1.8)--++(122.5+180+90:.01)node[sloped,midway]{1. Schwerpunkt};


\foreach \a/\t in {45/A,225/L, -45/P}{
\path (\a:2.2)node[scale=3]{\textbf{\t}};
}


\end{tikzpicture}
\end{document}