\documentclass[11pt]{standalone}

\usepackage[utf8]{inputenc}
\usepackage[german]{babel}
\usepackage{amsmath}
\usepackage{amsfonts}
\usepackage{amssymb}
\usepackage{graphicx}
\usepackage{ifthen}
\usepackage{array}
\usepackage{eurosym}

% \usepackage{emerald}
% \newcommand{\hw}[1]{{\ECFAugie #1}}


\usepackage{tikz}
\usetikzlibrary{calc,patterns,
                 decorations.pathmorphing,
                 decorations.markings,
                 decorations.pathreplacing}
\usetikzlibrary{trees,arrows}
\usetikzlibrary{shapes.geometric}
\usetikzlibrary{positioning}

                 
\newcommand*\circled[1]{\tikz[baseline=(char.base)]{
            \node[shape=circle,draw,inner sep=2pt,solid,fill=white] (char) {#1};}}
            
            
            
\usetikzlibrary{patterns}

\pgfdeclarepatternformonly[\LineSpace]{my north east lines}{\pgfqpoint{-1pt}{-1pt}}{\pgfqpoint{\LineSpace}{\LineSpace}}{\pgfqpoint{\LineSpace}{\LineSpace}}%
{
    \pgfsetlinewidth{0.4pt}
    \pgfpathmoveto{\pgfqpoint{0pt}{0pt}}
    \pgfpathlineto{\pgfqpoint{\LineSpace + 0.1pt}{\LineSpace + 0.1pt}}
    \pgfusepath{stroke}
}


\pgfdeclarepatternformonly[\LineSpace]{my north west lines}{\pgfqpoint{-1pt}{-1pt}}{\pgfqpoint{\LineSpace}{\LineSpace}}{\pgfqpoint{\LineSpace}{\LineSpace}}%
{
    \pgfsetlinewidth{0.4pt}
    \pgfpathmoveto{\pgfqpoint{0pt}{\LineSpace}}
    \pgfpathlineto{\pgfqpoint{\LineSpace + 0.1pt}{-0.1pt}}
    \pgfusepath{stroke}
}

\newdimen\LineSpace
\tikzset{
    line space/.code={\LineSpace=#1},
    line space=3pt
}


%% Arbeitszeit:
% 21.11.15 : 1430- 1440



\begin{document}
\begin{tikzpicture}[scale=.6]


\draw[-latex] (0,0)--(18,0)node[below]{Wochen};

\foreach \x in {0,4,10,14,16}{
\draw (\x,0)node[below]{\x}--++(0,.5);
}
\foreach \x in {1,2,6,8,12}{
\draw (\x,0)node[below]{\x}--++(0,.2);
}
\foreach \x/\t in {0/{Zeichnen und \\ Berechnen},4.5/{Muster \\ fertigen},10/{Versuche},14/{Bericht}}{
\draw (\x,0)++(0,2)node[right,align=left]{\t};
}

\draw[help lines] (4,-1)--++(0,-2);
\draw (4,-4)--++(5,0) node[right]{($\rightarrow$ Übergang zum Netzplan)} ;
\foreach \x in {4,9}{
\draw (\x,-4)node[below]{\x}--++(0,.5);
}
\foreach \x in {6,8}{
\draw (\x,-4)--++(0,.2);
}
\path (4.5,-2) node[right,align=left,black]{Versuchsstand \\ aufbauen};

\end{tikzpicture}

\end{document}