\documentclass[11pt]{standalone}

\usepackage[utf8]{inputenc}
\usepackage[german]{babel}
\usepackage{amsmath}
\usepackage{amsfonts}
\usepackage{amssymb}
\usepackage{graphicx}
\usepackage{ifthen}
\usepackage{array}
\usepackage{eurosym}

% \usepackage{emerald}
% \newcommand{\hw}[1]{{\ECFAugie #1}}


\usepackage{tikz}
\usetikzlibrary{calc,patterns,
                 decorations.pathmorphing,
                 decorations.markings,
                 decorations.pathreplacing}
\usetikzlibrary{trees,arrows}
\usetikzlibrary{shapes.geometric}
\usetikzlibrary{positioning}

                 
\newcommand*\circled[1]{\tikz[baseline=(char.base)]{
            \node[shape=circle,draw,inner sep=2pt,solid,fill=white] (char) {#1};}}
            
            
            
\usetikzlibrary{patterns}

\pgfdeclarepatternformonly[\LineSpace]{my north east lines}{\pgfqpoint{-1pt}{-1pt}}{\pgfqpoint{\LineSpace}{\LineSpace}}{\pgfqpoint{\LineSpace}{\LineSpace}}%
{
    \pgfsetlinewidth{0.4pt}
    \pgfpathmoveto{\pgfqpoint{0pt}{0pt}}
    \pgfpathlineto{\pgfqpoint{\LineSpace + 0.1pt}{\LineSpace + 0.1pt}}
    \pgfusepath{stroke}
}


\pgfdeclarepatternformonly[\LineSpace]{my north west lines}{\pgfqpoint{-1pt}{-1pt}}{\pgfqpoint{\LineSpace}{\LineSpace}}{\pgfqpoint{\LineSpace}{\LineSpace}}%
{
    \pgfsetlinewidth{0.4pt}
    \pgfpathmoveto{\pgfqpoint{0pt}{\LineSpace}}
    \pgfpathlineto{\pgfqpoint{\LineSpace + 0.1pt}{-0.1pt}}
    \pgfusepath{stroke}
}

\newdimen\LineSpace
\tikzset{
    line space/.code={\LineSpace=#1},
    line space=3pt
}

\usepackage{booktabs}

\usepackage{ragged2e}
\usepackage{array}


%% Arbeitszeit:
% 21.11.15 : 1950-2040
% ges: 50min



\begin{document}

\begin{tabular}{ >{\RaggedRight}p{6cm}  >{\RaggedRight}p{6cm} >{\RaggedRight}p{6cm} }
 \toprule
\textbf{Planungsphase} &  & \textbf{Projekt-Entscheidung}  \\ 
 \midrule
\textbf{Entwicklungsphase} & Projekt starten (E) ~~~~~ ~~~~~~~~~~~ PL und Team ernennen (E,L) &   \\ 
 \midrule
Anforderungsliste erstellen &   &   \\ 
 \midrule
  & Anfo.Liste verteilen Anforderungs-Besprechung einberufen &   \\ 
 \midrule
  &   & Anforderungs-Entscheidung (E,Q,S,V,Info an F) \\ 
 \midrule
\textbf{Funktionsmuster} konzipieren, entwerfen, konstruieren Je ein Muster bauen Funktionsfähigkeit testen & Änderungen zur Anfo-Liste sammeln &   \\ 
 \midrule
  & Ergebnisse zusammenstellen Konzept-Besprechung einberufen &   \\ 
 \midrule
  &   & Konzept-Entscheidung (E,F,V,S,Q) \\ 
 \midrule
\textbf{Labormuster} konstruieren Einige Labormuster bauen Funktion, Fertigung, Montage testen & Ggf. Patentstelle informieren &   \\ 
 \midrule
  & Ergebnisse zusammenstellen Prototyp-Besprechung einberufen &   \\ 
 \midrule
  &   & Prototyp-Entscheidung (insb. Bauteile, Fertigungs-, Prüfverfahren) (E,F,Q,S) \\ \\
 
 
 \toprule
 
 
\textbf{Ausarbeitungsphase} & Fertigungs-Projektleiter ernennen (F) Informations-Bespr. einberufen &   \\ 
 \midrule
Informationsbesprechung (interne Produktinformationen) & Anregungen übernehmen &   \\ 
 \midrule
\textbf{Prototypen} bauen (L), anwendungsnah testen (E, V), Prüfmittel entwickeln (Q), Fertigung, Montage testen (F) &   &   \\ 
 \midrule
  & Ergebnisse zusammenstellen Vorserien-Bespr. einberufen &   \\ 
 \midrule
  &   & Vorserien-Entscheidung (E,F,Q,S) \\ 
 \midrule
\textbf{Vorserie} fertigen (Produktion) (F) Vorserienprodukte prüfen (jeweils E/F/Q/S/V [Kunden]) Produkte ggf. überarbeiten/prüfen & Änderungen dokumentieren &   \\ 
 \midrule
  & Ergebnissbericht erstellen, Freigabe-Besprechung einberufen &   \\ 
 \midrule
  &   & Freigabe-Entscheidung (zur Serienfertigung) (E,F,G,M,Q,S,V) \\ 
 \bottomrule


\end{tabular}
\end{document}