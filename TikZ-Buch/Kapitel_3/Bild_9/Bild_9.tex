\documentclass[11pt]{standalone}

\usepackage[utf8]{inputenc}
\usepackage[german]{babel}
\usepackage{amsmath}
\usepackage{amsfonts}
\usepackage{amssymb}
\usepackage{graphicx}
\usepackage{ifthen}
\usepackage{booktabs}
\usepackage{multirow}
\usepackage{eurosym}

\usepackage{emerald}
\newcommand{\hw}[1]{{\ECFAugie #1}}


\usepackage{tikz}
\usetikzlibrary{calc,patterns,arrows,trees,
                decorations.pathmorphing,
                decorations.markings,
								decorations.pathreplacing}

\newcommand*\circled[1]{\tikz[baseline=(char.base)]{
            \node[shape=circle,draw,inner sep=2pt,solid,fill=white] (char) {#1};}}


\renewcommand{\familydefault}{\sfdefault}
\usepackage[scaled]{helvet}
%\usepackage{helvet}
\usepackage{ulem}

%% Arbeitszeit:
% 21.11.15 : 1700-1800
% ges: 60min



\begin{document}
\begin{tikzpicture}

\def\xabs{14}
\def\zhi{.6}
\def\zhii{1}
\pgfmathsetmacro{\zhhi}{.5*\zhi}
\pgfmathsetmacro{\zhhii}{.5*\zhii}

\pgfmathsetmacro{\Bi}{.3*\xabs}
\pgfmathsetmacro{\Bii}{.2*\xabs}
\pgfmathsetmacro{\Biii}{.4*\xabs}

% Oberer Rahmen
\draw [very thick] (0,0)--++(0,-\zhi-\zhii-\zhii-\zhii-\zhii)coordinate(ML);
\draw [very thick] (\xabs,0)--++(0,-\zhi-\zhii-\zhii-\zhii-\zhii)coordinate(MR);

% Erste Zeile
\draw[very thick] (0,0)node[below right]{\texttt{\underline{Freigabeformular}}}rectangle++(\Bi,-\zhi);
\draw[very thick] (\Bi,0)node[below right]{\texttt{{Produktgruppe}}}rectangle++(\Bi,-\zhi);
\draw[very thick] (\Bi+\Bi,0)node[below right]{\texttt{{Nr.:}}}rectangle++(\Bii,-\zhi);
\draw[very thick] (\Bi+\Bi+\Bii,0)node[below right]{\texttt{{Verteiler:}}}rectangle++(\Bii,-\zhi);
% Zweite Zeile
\draw (0,-\zhi)node[below right]{\texttt{Antargsteller:}}rectangle++(\Bi+\Bi,-\zhii);
\draw (\Bi+\Bi,-\zhi)node[below right]{\texttt{Datum:}}rectangle++(\Bii,-\zhii)coordinate(help);
\draw (help) rectangle++(\Bii,\zhhii);
\draw (help)++(0,\zhhii) rectangle++(\Bii,\zhhii);
% Dritte Zeile
\draw (0,-\zhi-\zhii)node[below right]{\texttt{Betrifft: (Produkt, Teil)}}rectangle++(\Bi+\Bi,-\zhii);
\draw (\Bi+\Bi,-\zhi-\zhii)node[below right]{\texttt{Sach-Nr.:}}rectangle++(\Bii,-\zhii)coordinate(help);
\draw (help) rectangle++(\Bii,\zhhii);
\draw (help)++(0,\zhhii) rectangle++(\Bii,\zhhii);
% Vierte Zeile
\draw (0,-\zhi-\zhii-\zhii)node[below right]{\texttt{Kalkulierte Stückzahl:}}rectangle++(\Biii,-\zhii);
\draw (\Biii,-\zhi-\zhii-\zhii)node[below right,align=left]{\texttt{Vorgesehener} \\ \texttt{Freigabetermin:}}rectangle++(\Biii,-\zhii)coordinate(help);
\draw (help) rectangle++(\Bii,\zhhii);
\draw (help)++(0,\zhhii) rectangle++(\Bii,\zhhii);
% Fuenfte Zeile
\path (0,-\zhi-\zhii-\zhii-\zhii)node[below right]{\texttt{Erläuterungen: (Unterlagen sind beigefügt)}};

% Cut off
\path (ML) coordinate(X)coordinate (start);
\def\n{20}
\foreach \i in {1,2,...,\n}{
\pgfmathsetmacro{\angle}{-90+rnd*180}
\pgfmathsetmacro{\radian}{\xabs/(\n+1)}
\pgfmathsetmacro{\x}{\i*\radian}
\path (start)++(\x,0)++(\angle:.1)coordinate(Xn);
\draw [] (X)--(Xn)coordinate(X);
}
\draw (X)--(MR);


% Unterer Rahmen
\draw [very thick] (ML)++(0,-\zhi)--++(0,-\zhii-\zhi-\zhii-\zhi-\zhi) coordinate(UL);
\draw [very thick] (MR)++(0,-\zhi)coordinate(MRi)--++(0,-\zhii-\zhi-\zhii-\zhi-\zhi) coordinate(UR);
% Cut off
\path (ML)++(0,-\zhi) coordinate(X)coordinate (start);
\def\n{20}
\foreach \i in {1,2,...,\n}{
\pgfmathsetmacro{\angle}{-90+rnd*180}
\pgfmathsetmacro{\radian}{\xabs/(\n+1)}
\pgfmathsetmacro{\x}{\i*\radian}
\path (start)++(\x,0)++(\angle:.1)coordinate(Xn);
\draw [] (X)--(Xn)coordinate(X);
}
\draw (X)--(MRi);
% Sechste Zeile
\draw (ML)++(0,-\zhi-\zhii)coordinate(6)node[below right]{\texttt{Freigabevermerk:}}rectangle++(\xabs,-\zhi);
\draw [very thick] (6)--++(\xabs,0);

% Siebte Zeile
\draw (6)++(0,-\zhii-\zhi)--++(\xabs,0);
\draw (6)++(0,-\zhi)coordinate(7)node[below right]{\texttt{Bereich:}};
\foreach \x/\t in {0/\texttt{Ja},1/{\texttt{Ja, *} \\ \texttt{falls}},2/\texttt{Nein*},3/\texttt{Datum},4/\texttt{Unterschrift}}{
\pgfmathsetmacro{\xx}{\Biii+\x*(\xabs-\Biii-\zhii-\zhi)/5}
\draw (7)++(\xx,0)node[below right,align=left]{\t}--++(0,-\zhii-\zhi);
}
% Achte Zeile
\draw (7)++(0,-\zhii-\zhi)coordinate(9)--++(\xabs,0);
% Neunte Zeile
\path (9) node[below right]{\texttt{*Begründung/Bedingung}};


% Cut off
\path (UL) coordinate(X)coordinate (start);
\def\n{30}
\foreach \i in {1,2,...,\n}{
\pgfmathsetmacro{\angle}{-90+rnd*180}
\pgfmathsetmacro{\radian}{\xabs/(\n+1)}
\pgfmathsetmacro{\x}{\i*\radian}
\path (start)++(\x,0)++(\angle:.1)coordinate(Xn);
\draw [] (X)--(Xn)coordinate(X);
}
\draw (X)--(UR);


\end{tikzpicture}

\end{document}