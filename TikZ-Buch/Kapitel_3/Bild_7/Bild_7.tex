\documentclass[11pt]{standalone}

\usepackage[utf8]{inputenc}
\usepackage[german]{babel}
\usepackage{amsmath}
\usepackage{amsfonts}
\usepackage{amssymb}
\usepackage{graphicx}
\usepackage{ifthen}
\usepackage{array}
\usepackage{eurosym}

% \usepackage{emerald}
% \newcommand{\hw}[1]{{\ECFAugie #1}}


\usepackage{tikz}
\usetikzlibrary{calc,patterns,
                 decorations.pathmorphing,
                 decorations.markings,
                 decorations.pathreplacing}
\usetikzlibrary{trees,arrows}
\usetikzlibrary{shapes.geometric}
\usetikzlibrary{positioning}

                 
\newcommand*\circled[1]{\tikz[baseline=(char.base)]{
            \node[shape=circle,draw,inner sep=2pt,solid,fill=white] (char) {#1};}}
            
            
            
\usetikzlibrary{patterns}

\pgfdeclarepatternformonly[\LineSpace]{my north east lines}{\pgfqpoint{-1pt}{-1pt}}{\pgfqpoint{\LineSpace}{\LineSpace}}{\pgfqpoint{\LineSpace}{\LineSpace}}%
{
    \pgfsetlinewidth{0.4pt}
    \pgfpathmoveto{\pgfqpoint{0pt}{0pt}}
    \pgfpathlineto{\pgfqpoint{\LineSpace + 0.1pt}{\LineSpace + 0.1pt}}
    \pgfusepath{stroke}
}


\pgfdeclarepatternformonly[\LineSpace]{my north west lines}{\pgfqpoint{-1pt}{-1pt}}{\pgfqpoint{\LineSpace}{\LineSpace}}{\pgfqpoint{\LineSpace}{\LineSpace}}%
{
    \pgfsetlinewidth{0.4pt}
    \pgfpathmoveto{\pgfqpoint{0pt}{\LineSpace}}
    \pgfpathlineto{\pgfqpoint{\LineSpace + 0.1pt}{-0.1pt}}
    \pgfusepath{stroke}
}

\newdimen\LineSpace
\tikzset{
    line space/.code={\LineSpace=#1},
    line space=3pt
}


%% Arbeitszeit:
% 21.11.15 : 1440-1600
% 04.12.15 : 1450-1455
% ges = 85min


\begin{document}
\begin{tikzpicture}

\def\x{8}
\def\y{2}
\def\ys{1}
\def\xs{1}
\def\arrshift{.2}

% Define the points:


\path (0,0)node(Konstr)[rectangle,draw,align=center]{Verantwortliche \\ Konstruktions- \\ stelle};
\path (-\x,0)node(Antragsteller)[rectangle,draw,align=center]{Antrag- \\ steller};
\path (\x,0)node(Kompetente)[rectangle,draw,align=center]{Kompetente \\ Stellen};
\path (\x,-\y)node(Besprechungsteilnehmer)[rectangle,draw,align=center]{Verantwortliche \\ Besprechnungs- \\ teilnehmer};
\path (\x,-\y-\y)node(Verantwortliche)[rectangle,draw,align=center]{Verantwortliche \\ Stellen};
\path (\xs,-\y-\ys)node(Zeichner)[rectangle,draw,align=center]{Konstrukteur, \\ Zeichner};

\path (0,-\y-\y-\y)node(Steuerung)[rectangle,draw,align=center]{Zentrale \\ Steuerungsstelle \\ (+Vervielfältigung)};
\path (\xs+\xs,-\y-\y-\y-\y-\ys)node(Ausfuehrer)[rectangle,draw,align=center]{Ausführende \\ Stellen \\ im Betrieb};
\path (-\xs-\xs,-\y-\y-\y-\y-\ys)node(VerantwortlicheBet)[rectangle,draw,align=center]{Verantwortliche \\ Stellen \\ im Betrieb};


% Pfeile
\draw [-latex] (Antragsteller.east)--(Konstr)node[midway,above]{\circled{1} Antrag (schriftlich)};
\draw [latex-latex] (Konstr)--(Kompetente)node[midway,above]{\circled{2} Ggf. Rückfragen};
\draw [latex-latex,very thick] (Konstr.-15)--++(4,0)node[midway,below,align = center]{\circled{6} Besprechnung / \\ Entscheidung}|-(Besprechungsteilnehmer);
\draw [-latex,very thick] (Konstr.-37)--(Zeichner)node[midway,left]{\circled{7}}coordinate[pos=.5](help);
\fill[black] (help)circle(.05);
\draw [-latex] (help)--++(3.5,0)node[midway,below]{Informationen} |- (Verantwortliche);
\draw [-latex,very thick] (Zeichner)--(Steuerung.38.7)node[midway,left]{\circled{8}}node[midway,right,align=left]{Geänderte \\ Zeichnungen};
\draw [-latex,dashed,very thick] (Konstr.-140)--(Steuerung.140)node[midway,left,align=right] {\circled{3} \\ Antrags- \\ Formular};
\draw [-latex,thick] (Konstr.-150)--++(0,-.5) -| (Antragsteller.-70)node[pos=.15,below]{\circled{7} Ggf. Ablehnung};
\draw [-latex,dashed,very thick] (VerantwortlicheBet.west)--++(-2,0)|-(Konstr.195)node[pos=.25,left,align=right] {\circled{5} \\ Stellungs- \\ nahme};
\draw[-latex,dashed] (Steuerung.-150)coordinate(help)--++(-150:.3)-|(Antragsteller.-110)node[pos=.8,right,align=left]{\circled{4} \\ Informationen};
\draw [-latex,very thick,dashed] (help)--(VerantwortlicheBet)node[midway,left]{\circled{4}};
\draw [-latex,very thick,dashed] (help)--(VerantwortlicheBet.50);
\draw [-latex,very thick,dashed] (help)--(VerantwortlicheBet.36)node[midway,right]{Verteilung};
\draw [-latex,very thick] (Steuerung.-28)coordinate(help)--(Ausfuehrer.126)coordinate[midway](help2);
\draw [-latex,very thick] (help)--(Ausfuehrer.140);
\draw [-latex,very thick] (help)--(Ausfuehrer);
\path (help2)node[]{\circled{9}};

\draw [fill] (Ausfuehrer.50)--++(0,1.5)circle(.05)coordinate[pos=.5](help2)coordinate(help);
\draw [-latex] (Ausfuehrer.50)--(help2);
\draw (Ausfuehrer.36)--(help)coordinate[pos=.5](help2);
\draw [-latex] (Ausfuehrer.36)--(help2);
\draw (Ausfuehrer)--(help)coordinate[pos=.5](help2);
\draw [-latex] (Ausfuehrer)--(help2);
\draw [-latex] (help)|-(Steuerung)node[pos=.3,right]{\circled{10} Alte Zeichnungen};

\draw (Ausfuehrer)--++(20:2)node[right]{\begin{tabular}{ll}
z.B: & - Lager \\
& - Kalkulation \\
& - Qualitätssicherung \\
& - Fertigung \\
& - Montage \\
& - Produktmanager 
\end{tabular}};

\end{tikzpicture}

\end{document}