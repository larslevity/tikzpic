\documentclass[11pt]{standalone}

\usepackage[utf8]{inputenc}
\usepackage[german]{babel}
\usepackage{amsmath}
\usepackage{amsfonts}
\usepackage{amssymb}
\usepackage{graphicx}
\usepackage{ifthen}
\usepackage{array}
\usepackage{eurosym}

% \usepackage{emerald}
% \newcommand{\hw}[1]{{\ECFAugie #1}}


\usepackage{tikz}
\usetikzlibrary{calc,patterns,
                 decorations.pathmorphing,
                 decorations.markings,
                 decorations.pathreplacing}
\usetikzlibrary{trees,arrows}
\usetikzlibrary{shapes.geometric}
\usetikzlibrary{positioning}

                 
\newcommand*\circled[1]{\tikz[baseline=(char.base)]{
            \node[shape=circle,draw,inner sep=2pt,solid,fill=white] (char) {#1};}}
            
            
            
\usetikzlibrary{patterns}

\pgfdeclarepatternformonly[\LineSpace]{my north east lines}{\pgfqpoint{-1pt}{-1pt}}{\pgfqpoint{\LineSpace}{\LineSpace}}{\pgfqpoint{\LineSpace}{\LineSpace}}%
{
    \pgfsetlinewidth{0.4pt}
    \pgfpathmoveto{\pgfqpoint{0pt}{0pt}}
    \pgfpathlineto{\pgfqpoint{\LineSpace + 0.1pt}{\LineSpace + 0.1pt}}
    \pgfusepath{stroke}
}


\pgfdeclarepatternformonly[\LineSpace]{my north west lines}{\pgfqpoint{-1pt}{-1pt}}{\pgfqpoint{\LineSpace}{\LineSpace}}{\pgfqpoint{\LineSpace}{\LineSpace}}%
{
    \pgfsetlinewidth{0.4pt}
    \pgfpathmoveto{\pgfqpoint{0pt}{\LineSpace}}
    \pgfpathlineto{\pgfqpoint{\LineSpace + 0.1pt}{-0.1pt}}
    \pgfusepath{stroke}
}

\newdimen\LineSpace
\tikzset{
    line space/.code={\LineSpace=#1},
    line space=3pt
}


%% Arbeitszeit:
% 21.01.16 : 1810 - 1835
% ges = 25 min

\begin{document}
\begin{tikzpicture}[xscale=2.5]

\renewcommand{\hw}[1]{#1}
\newcommand{\te}[1]{#1}

\foreach \y/\ti/\t/\tii in {
1/\hw{Kaffeebohnen}/\hw{Kaffeemehl} / \hw{(Stoff)},
0/ \hw{Energie (elektr.)}/ \te{\hw{Wärme} \\ \hw{Geräusch}} / \hw{(Energie)},
-1/ \hw{Schaltsignal} / \hw{Mahlende} / \hw{(Information)}
}{
\path[] (-2,\y)node[left=.2cm]{\ti}--++(1,0)node[midway,above]{\tii};
\path[] (1,\y)--++(1,0)node[right=.2cm,align=left]{\t};
}

\foreach \y/\opt in {1/dashed,
0/very thick,
-1/}{
\draw[\opt,black] (-2,\y)--++(1,0);
\draw[-latex] (-1.1,\y)--++(.1,0);
\draw[\opt,black] (1,\y)--++(1,0);
\draw[-latex] (1.9,\y)--++(.1,0);


}


\draw(-1,1.2)rectangle(1,-1.2)node[midway,align = center]{\hw{Kaffebohnen} \\ \hw{zerkleinern}};

\end{tikzpicture}
\end{document}