\documentclass[11pt]{standalone}

\usepackage[utf8]{inputenc}
\usepackage[german]{babel}
\usepackage{amsmath}
\usepackage{amsfonts}
\usepackage{amssymb}
\usepackage{graphicx}
\usepackage{ifthen}
\usepackage{booktabs}
\usepackage{multirow}
\usepackage{eurosym}

\usepackage{emerald}
\newcommand{\hw}[1]{{\ECFAugie #1}}


\usepackage{tikz}
\usetikzlibrary{calc,patterns,arrows,trees,
                decorations.pathmorphing,
                decorations.markings,
								decorations.pathreplacing}

\newcommand*\circled[1]{\tikz[baseline=(char.base)]{
            \node[shape=circle,draw,inner sep=2pt,solid,fill=white] (char) {#1};}}


\renewcommand{\familydefault}{\sfdefault}
\usepackage[scaled]{helvet}
%\usepackage{helvet}


%% Arbeitszeit:
% 31.01.16 : 0220 - 240
% ges = 40 min

\begin{document}
\begin{tikzpicture}[scale=5.5]


\draw[-latex] (0,0)--(1.2,0)node[below]{\%};
\draw[-latex] (0,0)--(0,1.2)node[left]{\%};

\foreach \x in {0,50,100}{
\pgfmathsetmacro{\xx}{\x/100}
\draw (\xx,0)++(0,.01)--++(0,-.02)node[below]{\x};
}


\foreach \y in {0,50,75,80,95,100}{
\pgfmathsetmacro{\yy}{\y/100}
\draw (0,\yy)++(.01,0)--++(-.02,0)node[left]{\y};
}

\draw[help lines] (0,.78)-|(.215,0);
\draw[help lines] (0,1)-|(1,0);
\draw[help lines] (0,.95)-|(.42,0);

\path (0,0) coordinate(last);
\foreach \x in {0,0.01,...,1}{
\pgfmathsetmacro{\y}{1-exp(-\x*7)}
\draw (last)--(\x,\y)coordinate(last);
}
\draw(last)--(1,1);

\foreach \x/\t in {.1/A,.3/B,.7/C}{
\path(\x,.2)node{\t};
}

\path (-.15,0)--++(0,1.2)node[midway,sloped,above]{Prozentuale Herstellungskosten};


\end{tikzpicture}
\end{document}