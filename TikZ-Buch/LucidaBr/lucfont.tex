%%
%% This is file `lucfont.tex',
%% generated with the docstrip utility.
%%
%% The original source files were:
%%
%% lucidabr.dtx  (with options: `T1,lucfont')
%% 
%% Copyright 1995, 1996 Sebastian Rahtz
%% Copyright 1997, 1998 Sebastian Rahtz, David Carlisle
%% Copyright 2005 TeX Users Group
%%
%% This file is part of the lucidabr package.
%%
%% This work may be distributed and/or modified under the
%% conditions of the LaTeX Project Public License, either version 1.3
%% of this license or (at your option) any later version.
%% The latest version of this license is in
%%   http://www.latex-project.org/lppl.txt
%% and version 1.3 or later is part of all distributions of LaTeX
%% version 2003/12/01 or later.
%%
%% This work has the LPPL maintenance status "maintained".
%%
%% The Current Maintainer of this work is the TeX Users Group
%% (http://tug.org/lucida).
%%
%% The list of all files belonging to the lucidabr package is
%% given in the file `manifest.txt'.
%%
%% The list of derived (unpacked) files belonging to the distribution
%% and covered by LPPL is defined by the unpacking scripts (with
%% extension .ins) which are part of the distribution.
%%
\ProvidesFile{lucfont.tex}
 [2005/11/29 v4.3 %
 Lucida Bright text font test
 (SPQR/DPC/TUG)]
\documentclass{article}
\usepackage[T1]{fontenc}
\begin{document}
\title{All the Lucida text fonts}
\author{prepared by Sebastian Rahtz}
\date{February 19th 1995}
\maketitle
\def\test#1#2#3#4#5{%
 \item[#1/#2/#3]#4 (#5):
 {\fontfamily{#1}\fontseries{#2}\fontshape{#3}\selectfont
 Animadversion for a giraffe costs \pounds123. Wa\ss\ ist
 das f\"ur ein Klopf?
 We are often na{\"\i}ve vis-\`{a}-vis
the d{\ae}monic ph{\oe}nix's official r\^{o}le in fluffy souffl\'{e}s}
}

\begin{description}
\test{hlx}{b}{it}{hlxdi8t}{LucidaFax-DemiItalic}
\test{hlx}{b}{n}{hlxd8t}{LucidaFax-Demi}
\test{hlx}{m}{it}{hlxrir8t}{LucidaFax-Italic}
\test{hlx}{m}{n}{hlxr8t}{LucidaFax}

\test{hlh}{b}{it}{hlcdib8t}{LucidaBright-DemiItalic}
\test{hlh}{b}{n}{hlcdb8t}{LucidaBright-Demi}
\test{hlh}{m}{it}{hlcrib8t}{LucidaBright-Italic}
\test{hlh}{m}{n}{hlcrb8t}{LucidaBright}

\test{hlce}{m}{it}{hlcrie8t}{LucidaCalligraphy-Italic}

\test{hlcf}{m}{n}{hlcrf8t}{LucidaBlackletter}

\test{hlcn}{m}{it}{hlcrin8t}{LucidaCasual-Italic}
\test{hlcn}{m}{n}{hlcrn8t}{LucidaCasual}

\test{hlst}{b}{n}{hlsbt8t}{LucidaSans-TypewriterBold}
\test{hlst}{b}{sl}{hlsbot8t}{LucidaSans-TypewriterBoldOblique}

\test{hls}{ub}{it}{hlsbi8t}{LucidaSans-BoldItalic}
\test{hls}{ub}{n}{hlsb8t}{LucidaSans-Bold}
\test{hls}{b}{it}{hlsdi8t}{LucidaSans-DemiItalic}
\test{hls}{b}{n}{hlsd8t}{LucidaSans-Demi}
\test{hls}{m}{it}{hlsri8t}{LucidaSans-Italic}
\test{hls}{m}{n}{hlsr8t}{LucidaSans}

\test{hlct}{b}{n}{hlcbt8t}{LucidaTypewriterBold}
\test{hlct}{b}{sl}{hlcbot8t}{LucidaTypewriterOblique}
\test{hlcw}{m}{it}{hlcriw8t}{LucidaHandwriting-Italic}

\end{description}
\end{document}
\endinput
%%
%% End of file `lucfont.tex'.
