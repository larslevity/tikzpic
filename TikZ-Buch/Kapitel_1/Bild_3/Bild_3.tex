\documentclass[11pt]{standalone}

\usepackage[utf8]{inputenc}
\usepackage[german]{babel}
\usepackage{amsmath}
\usepackage{amsfonts}
\usepackage{amssymb}
\usepackage{graphicx}
\usepackage{ifthen}
\usepackage{booktabs}
\usepackage{multirow}
\usepackage{eurosym}

\usepackage{emerald}
\newcommand{\hw}[1]{{\ECFAugie #1}}


\usepackage{tikz}
\usetikzlibrary{calc,patterns,arrows,trees,
                decorations.pathmorphing,
                decorations.markings,
								decorations.pathreplacing}

\newcommand*\circled[1]{\tikz[baseline=(char.base)]{
            \node[shape=circle,draw,inner sep=2pt,solid,fill=white] (char) {#1};}}


\renewcommand{\familydefault}{\sfdefault}
\usepackage[scaled]{helvet}
%\usepackage{helvet}


%% Arbeitszeit:
% 04.12.15 : 1957-2020;
% ges = 23min

\begin{document}
\begin{tikzpicture}[scale = 3]

\draw[-latex] (0,0)--++(4.5,0)node[below]{Zeit};
\draw[-latex] (0,0)--++(0,2)node[pos=.99,sloped,above left]{Entwicklungsorientierung};

\foreach \x/\t/\tt/\ttt in {1/{schwer, \\ massebetont}/{Material- \\ orientiert}/{Material},2/{energiebetont}/{Energie- \\ orientiert}/{Energie},3/{informations- \\ betont}/{Informations-/ \\ Wissens- \\ orientiert}/{Information},4/{Bewusstsein \\ empfingungs- \\ betont}/{Geistige \\ Orientierung \\ ???}/{Psyche ???}}{
\pgfmathsetmacro{\xx}{(\x-1)+.1}
\ifthenelse{\x<4}{\def\opt{solid}
\draw[-latex] (\xx+1,1.45)--++(.1,0);
}{\def\opt{dashed}}
\draw[\opt] (\xx,.1) .. controls (\xx+.3,1).. (\xx+1.6,.1);
\path (\xx+.2,.3)node[right,align=center]{\t};
\draw[\opt] (\xx+.1,1.2)rectangle(\xx+1,1.7)node[midway,align=center]{\tt};
\path (\xx+.1+.45,1.8)node[above]{\textbf{\ttt} };
}

\foreach \x/\t in {.5/1800,1.7/1900,2.9/2000}{\path (\x,0)node[below]{\t};}

\end{tikzpicture}
\end{document}