\documentclass[11pt]{standalone}

\usepackage[utf8]{inputenc}
\usepackage[german]{babel}
\usepackage{amsmath}
\usepackage{amsfonts}
\usepackage{amssymb}
\usepackage{graphicx}
\usepackage{ifthen}
\usepackage{booktabs}
\usepackage{multirow}
\usepackage{eurosym}

\usepackage{emerald}
\newcommand{\hw}[1]{{\ECFAugie #1}}


\usepackage{tikz}
\usetikzlibrary{calc,patterns,arrows,trees,
                decorations.pathmorphing,
                decorations.markings,
								decorations.pathreplacing}

\newcommand*\circled[1]{\tikz[baseline=(char.base)]{
            \node[shape=circle,draw,inner sep=2pt,solid,fill=white] (char) {#1};}}


\renewcommand{\familydefault}{\sfdefault}
\usepackage[scaled]{helvet}
%\usepackage{helvet}


%% Arbeitszeit:
% 04.12.15 : 2020- 2050;
% 05.12.15 : 1820 - 1855

% ges = 30 + 35 = 65 min

\begin{document}
\begin{tikzpicture}[scale = 1]




% Kosten kurve
\fill[gray!60] (0,0)to[out=0,in = 180](2,-.7)to[out=0,in = 180+50](3,0)--cycle;
\draw (0,0)to[out=0,in = 180](2,-.7)to[out=0,in = 180+50](3,0)to[out=50,in = 180](4.5,.7)to[out=0,in=-5+180](8,0);
\fill[gray!60] (3,0)to[out=50,in = 180](4.5,.7)to[out=0,in=-5+180](5,.67)--(5,0)--cycle;
\draw (0,0)to[out=0,in = 180](2,-.7)to[out=0,in = 180+50](3,0)to[out=50,in = 180](4.5,.7)to[out=0,in=-5+180](8,0);

% kumulierter Gewinn
\draw [dash pattern = on 10pt off 3pt on 1pt off 3pt] (0,0)to[out=0,in = 180] (3,-.5) to [out=0,in = 30+180](5,0)to[out=30,in=180](9,1)node[right,align=left]{kumulierter \\ Gewinn};

% Umsatz
\draw (2,0)to[out=0,in = 180](6,4)to[out=0,in=-50+180](9,2)node[right]{Umsatz};





% Achsen

\draw [-latex] (0,0)--(10,0)node[below]{Zeit};
\draw[-latex] (0,0)--(0,5)node[above,sloped,pos=.7,align=left]{Umsatz / Gewinn};

\path (0,5) coordinate(last);

\draw[dashed] (1.2,5)coordinate(next)--++(0,-5.5);
\path (last)--(next)coordinate(last)node[midway,left,rotate =90]{\footnotesize Entwicklung};

\draw[dashed] (2,5)coordinate(next)--++(0,-5.9);
\path (last)--(next)coordinate(last)node[midway,left,rotate = 90,align=center]{\footnotesize Fertigungsvorbereitung};

\draw[dashed] (2.7,5)coordinate(next)--++(0,-5.6);
\path (last)--(next)coordinate(last)node[midway,left,rotate = 90,align=center]{\footnotesize{Markteinführung}};

\draw[dashed] (4,5)coordinate(next)--++(0,-5.6);
\path (last)--(next)coordinate(last)node[midway,left,rotate = 90,align=center]{\footnotesize{Wachstum}};

\draw[dashed] (5.5,5)coordinate(next)--++(0,-5.1);
\path (last)--(next)coordinate(last)node[midway,left,rotate = 90,align=center]{\footnotesize{Reife}};

\draw[dashed] (7,5)coordinate(next)--++(0,-5.1);
\path (last)--(next)coordinate(last)node[midway,left,rotate = 90,align=center]{\footnotesize{Sättigung}};

\draw[dashed] (9,5)coordinate(next)++(0,-5.6);
\path (last)--(next)coordinate(last)node[midway,left,rotate = 90,align=center]{\footnotesize{Abstieg}};


% Anmerkungen
\draw[thick, latex-latex] (4.6,0)--++(0,.72)node[below right,midway,rotate=45]{Gewinn};

\draw[thick, latex-latex] (4.6,.72)--++(0,2.45)node[below right,midway,rotate=45]{Kosten};



\path (3,-1.5)node(text)[align=center]{Flächen gleich \\ (Erlös = Kosten)};
\draw[help lines] (2,-.5)--(text);
\draw[help lines] (4,.5)--(text);

\path (0,-1.5)node(text)[align=center]{Kosten};
\draw[help lines] (1,-.38)--(text);


\path (6,-1.5)node(text)[align=center]{ab hier \\ Gewinn};
\draw[help lines] (3,0)--(text);

\path (10,-1.5)node(text)[align=center]{Break-Even-Point \\ ab hier kumulierter Gewinn $>$ 0};
\draw[help lines] (5,0)--(text);











\end{tikzpicture}
\end{document}