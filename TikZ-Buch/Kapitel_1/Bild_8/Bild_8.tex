\documentclass[11pt]{standalone}

\usepackage[utf8]{inputenc}
\usepackage[german]{babel}
\usepackage{amsmath}
\usepackage{amsfonts}
\usepackage{amssymb}
\usepackage{graphicx}
\usepackage{ifthen}
\usepackage{array}
\usepackage{eurosym}

% \usepackage{emerald}
% \newcommand{\hw}[1]{{\ECFAugie #1}}


\usepackage{tikz}
\usetikzlibrary{calc,patterns,
                 decorations.pathmorphing,
                 decorations.markings,
                 decorations.pathreplacing}
\usetikzlibrary{trees,arrows}
\usetikzlibrary{shapes.geometric}
\usetikzlibrary{positioning}

                 
\newcommand*\circled[1]{\tikz[baseline=(char.base)]{
            \node[shape=circle,draw,inner sep=2pt,solid,fill=white] (char) {#1};}}
            
            
            
\usetikzlibrary{patterns}

\pgfdeclarepatternformonly[\LineSpace]{my north east lines}{\pgfqpoint{-1pt}{-1pt}}{\pgfqpoint{\LineSpace}{\LineSpace}}{\pgfqpoint{\LineSpace}{\LineSpace}}%
{
    \pgfsetlinewidth{0.4pt}
    \pgfpathmoveto{\pgfqpoint{0pt}{0pt}}
    \pgfpathlineto{\pgfqpoint{\LineSpace + 0.1pt}{\LineSpace + 0.1pt}}
    \pgfusepath{stroke}
}


\pgfdeclarepatternformonly[\LineSpace]{my north west lines}{\pgfqpoint{-1pt}{-1pt}}{\pgfqpoint{\LineSpace}{\LineSpace}}{\pgfqpoint{\LineSpace}{\LineSpace}}%
{
    \pgfsetlinewidth{0.4pt}
    \pgfpathmoveto{\pgfqpoint{0pt}{\LineSpace}}
    \pgfpathlineto{\pgfqpoint{\LineSpace + 0.1pt}{-0.1pt}}
    \pgfusepath{stroke}
}

\newdimen\LineSpace
\tikzset{
    line space/.code={\LineSpace=#1},
    line space=3pt
}



%% Arbeitszeit:
% 07.12.15 : 1246 - 
% ges = 33min

\begin{document}
\begin{tikzpicture}[scale=5]

\draw[-latex,thick] (0,0)--(0,1)node[pos=.99,above left,sloped]{Wert, Nutzwert};

\draw [thick] (0,0)--(1.6,0);


\draw[dash pattern=on 10pt off 3pt on 1pt off 3pt,thick] (0,.7)--(1.96,.7)coordinate[pos=.98](help);
\draw[latex-] (help)--++(0,.15)node[above]{Angebot};
\draw[latex-] (help)--++(0,-.15)node[below]{Nachfrage};



\draw[thick] (.105,.75)--(.125,.7)--(.15,.8)--(.3,.8)node[midway,above]{Preis};


% Wertz für Hersteller

\draw [] (.2,0)rectangle(.5,.6)node[midway,rotate=90]{Kosten}coordinate(help);
\draw[dashed, help lines] (help)--++(.5,0);
\path (.2,1.2)++(.15,0)node[below,align=center]{Wert für \\ Hersteller};
\draw [latex-latex] (.55,0)--++(0,.7)node[midway,sloped,below,align=center]{Nutzwert für \\ den Hersteller};





% Wert für Abnehmer
\draw [] (1.1,0)rectangle++(.3,.8)node[midway,rotate=90]{Kosten}coordinate(help);
\draw[dashed, help lines] (help)++(-.3,0)--++(-.5,0) (help)--++(.15,0);
\path (1.1,1.2)++(.15,0)node[below,align=center]{Wert für \\ Abnehmer};


\draw [latex-latex] (1.45,0)--++(0,.8)node[midway,sloped,below,align=center]{Nutzwert für \\ den Abnehmer};



% Werte Summe

\draw [] (.2+.45,0.6)rectangle++(.3,.2);
\path (.2+.45,1.2)++(.15,0)node[below,align=center]{Werte- \\ summe};



\end{tikzpicture}
\end{document}