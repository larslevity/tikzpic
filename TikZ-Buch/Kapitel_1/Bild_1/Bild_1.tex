\documentclass[11pt]{standalone}

\usepackage[utf8]{inputenc}
\usepackage[german]{babel}
\usepackage{amsmath}
\usepackage{amsfonts}
\usepackage{amssymb}
\usepackage{graphicx}
\usepackage{ifthen}
\usepackage{array}
\usepackage{eurosym}

% \usepackage{emerald}
% \newcommand{\hw}[1]{{\ECFAugie #1}}


\usepackage{tikz}
\usetikzlibrary{calc,patterns,
                 decorations.pathmorphing,
                 decorations.markings,
                 decorations.pathreplacing}
\usetikzlibrary{trees,arrows}
\usetikzlibrary{shapes.geometric}
\usetikzlibrary{positioning}

                 
\newcommand*\circled[1]{\tikz[baseline=(char.base)]{
            \node[shape=circle,draw,inner sep=2pt,solid,fill=white] (char) {#1};}}
            
            
            
\usetikzlibrary{patterns}

\pgfdeclarepatternformonly[\LineSpace]{my north east lines}{\pgfqpoint{-1pt}{-1pt}}{\pgfqpoint{\LineSpace}{\LineSpace}}{\pgfqpoint{\LineSpace}{\LineSpace}}%
{
    \pgfsetlinewidth{0.4pt}
    \pgfpathmoveto{\pgfqpoint{0pt}{0pt}}
    \pgfpathlineto{\pgfqpoint{\LineSpace + 0.1pt}{\LineSpace + 0.1pt}}
    \pgfusepath{stroke}
}


\pgfdeclarepatternformonly[\LineSpace]{my north west lines}{\pgfqpoint{-1pt}{-1pt}}{\pgfqpoint{\LineSpace}{\LineSpace}}{\pgfqpoint{\LineSpace}{\LineSpace}}%
{
    \pgfsetlinewidth{0.4pt}
    \pgfpathmoveto{\pgfqpoint{0pt}{\LineSpace}}
    \pgfpathlineto{\pgfqpoint{\LineSpace + 0.1pt}{-0.1pt}}
    \pgfusepath{stroke}
}

\newdimen\LineSpace
\tikzset{
    line space/.code={\LineSpace=#1},
    line space=3pt
}

\usepackage{tikz-3dplot}


%% Arbeitszeit:
% 04.12.15 : 1920-1953
% ges = 33min

\begin{document}
\tdplotsetmaincoords{55}{120}
\begin{tikzpicture}[tdplot_main_coords,scale = 5]



\tdplotsetrotatedcoords{0}{0}{0}
\fill[gray!20,tdplot_rotated_coords] (0,0)--++(1,0)--++(0,1)--++(-1,0)--cycle;
\tdplotsetrotatedcoords{0}{0}{0}
\fill[gray!60,opacity = .3] (0,0,0)--(0,0,1)--(1,1,1)--(1,1,0)--cycle;


\draw[-latex] (0,0,0)--++(1,0,0)node[left]{Fähigkeiten};
\draw[-latex] (0,0,0)--++(0,1,0)node[right]{Wissen};
\draw[-latex] (0,0,0)--++(0,0,1)node[above]{Motivation, Zeit};
\draw[-latex] (0,0,0)--++(1,1,0)node[below]{Können};

\draw[-latex,thick] (0,0,0)--(1,1,1) node[right,align=center]{Leistungsfähigkeit eines \\ Produktentwicklers / \\ Konstrukteurs};


\draw[dashed, help lines] (1,0,0)--(1,0,1)--(0,0,1)--(0,1,1)--(0,1,0);
\draw[-latex,dashed] (0,0,0)--(1,0,1)node[pos=.4,below left,align=center]{\footnotesize Talent,\\ \footnotesize Phantasie, \\  \footnotesize Training,\\ \footnotesize Übung}node[ left,align=right]{Fähigkeitserwerb, \\ Ausbildung};

\draw[-latex,dashed] (0,0,0)--(0,1,1)node[ right,align=left]{Wissenserwerb, \\ Schulung} 
node[pos=.6,above left,align=center]{\footnotesize Verständnis,\\ \footnotesize Kreativität, \\ \footnotesize Intelligenz};




\end{tikzpicture}
\end{document}