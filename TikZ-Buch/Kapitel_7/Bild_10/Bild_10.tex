\documentclass[11pt]{standalone}

\usepackage[utf8]{inputenc}
\usepackage[german]{babel}
\usepackage{amsmath}
\usepackage{amsfonts}
\usepackage{amssymb}
\usepackage{graphicx}
\usepackage{ifthen}
\usepackage{booktabs}
\usepackage{multirow}
\usepackage{eurosym}

\usepackage{emerald}
\newcommand{\hw}[1]{{\ECFAugie #1}}


\usepackage{tikz}
\usetikzlibrary{calc,patterns,arrows,trees,
                decorations.pathmorphing,
                decorations.markings,
								decorations.pathreplacing}

\newcommand*\circled[1]{\tikz[baseline=(char.base)]{
            \node[shape=circle,draw,inner sep=2pt,solid,fill=white] (char) {#1};}}


\renewcommand{\familydefault}{\sfdefault}
\usepackage[scaled]{helvet}
%\usepackage{helvet}


%% Arbeitszeit:
% 21.01.16 : 2050-2115
% ges = 25 min

\begin{document}
\begin{tikzpicture}[scale=2]

\def\xa{1.2}
\def\xb{2}
\def\xc{3}

\def\xd{4}


\draw[-latex] (0,0)node{+}node[left,align=right]{Ausgangspunkt \\ (Bezuhspunkt)} -- (\xa,1)coordinate(X)node[midway,above left]{$M_1$};
\draw[-latex] (X) -- (\xb,.8)coordinate(X)node[midway,above]{$M_2$};
\draw[-latex] (X) -- (\xc,-.5)coordinate(X)node[midway,right]{$M_3$};
\draw[-latex] (X) -- (0,0)node[midway,below]{$M_0$};

\draw[help lines] (\xd,-.5)node[black,below,align=center]{Ausgangslinie \\ (Bezugslinie)} --++(0,1.5)coordinate(X);
\draw[help lines,-latex] (X)--++(.5,0)node[above]{$+$};
\draw[help lines,-latex] (X)--++(-.5,0)node[above]{$-$};

\path[help lines] (X)++(0,.3)node{Zählrichtung};



\draw[-latex] (\xd,.8)--++(\xa,0)node[midway,above]{$M_1$}coordinate(X);
\draw[help lines] (X) ++ (0,.05)--++(0,-.5) ++ (0,.05)coordinate(X); 
\draw[-latex] (X)--++(-\xb+\xa,0)node[midway,above]{$M_2$}coordinate(X);
\draw[help lines] (X) ++ (0,.05)--++(0,-.5) ++ (0,.05)coordinate(X); 
\draw[-latex] (X)--++(\xc-\xa,0)node[midway,above]{$M_3$}coordinate(X);
\draw[help lines] (X) ++ (0,.05)--++(0,-.5) ++ (0,.05)coordinate(X);
\draw[-latex] (X)--(\xd,-.4)node[midway,above]{$M_0$};  


\end{tikzpicture}
\end{document}