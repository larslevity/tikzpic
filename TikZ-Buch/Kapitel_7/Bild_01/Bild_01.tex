\documentclass[11pt]{standalone}

\usepackage[utf8]{inputenc}
\usepackage[german]{babel}
\usepackage{amsmath}
\usepackage{amsfonts}
\usepackage{amssymb}
\usepackage{graphicx}
\usepackage{ifthen}
\usepackage{booktabs}
\usepackage{multirow}
\usepackage{eurosym}

\usepackage{emerald}
\newcommand{\hw}[1]{{\ECFAugie #1}}


\usepackage{tikz}
\usetikzlibrary{calc,patterns,arrows,trees,
                decorations.pathmorphing,
                decorations.markings,
								decorations.pathreplacing}

\newcommand*\circled[1]{\tikz[baseline=(char.base)]{
            \node[shape=circle,draw,inner sep=2pt,solid,fill=white] (char) {#1};}}


\renewcommand{\familydefault}{\sfdefault}
\usepackage[scaled]{helvet}
%\usepackage{helvet}


%% Arbeitszeit:
% 21.01.16 : 1200-1245
% ges = 45 min

\begin{document}

\begin{tikzpicture}[
edge from parent fork down,
edge from parent/.style = {draw,-latex},
every node/.style={rectangle, draw, align=center},
A/.style={minimum width = 3cm, minimum height = 1cm, align = center},
B/.style={minimum width = 5cm, minimum height = 1cm, align = center},
level 1/.append style={level distance=2cm,sibling distance=4cm},
level 2/.append style={level distance=2cm,sibling distance=5cm},
level 3/.append style={level distance=2cm,sibling distance=3cm},
]


\node[A] at (0,0) {Kunde}
	child { node[A]{Anspruchsklasse}
		child{ node(T)[A] {Termin}}
		child{ node(F)[B] {Forderung / Erwartungen}}
		child{ node(P)[A] {Preis}}
	}
	;


\begin{scope}[edge from parent fork up, grow = up
]
\node[A] at (0,-11) {Lieferant}
	child { node[A]{Produkt}
		child{ node(PP)[A] {Kosten}}		
		child{ node(FF)[B] {Beschaffenheit}}
		child{ node(TT)[A] {Termin}}
	}
	;
\end{scope}


\draw[latex-latex] (T)--(TT);
\draw[latex-latex] (P)--(PP);
\foreach \i in {-2,-1.5,...,2}{
\draw[latex-latex] (FF)++(\i,.5)--++(0,2);
}
\path (FF)++(0,1.5)node[fill=white,draw=none]{Qualitätsmerkmale};



\end{tikzpicture}
\end{document}