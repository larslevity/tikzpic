\documentclass[11pt]{standalone}

\usepackage[utf8]{inputenc}
\usepackage[german]{babel}
\usepackage{amsmath}
\usepackage{amsfonts}
\usepackage{amssymb}
\usepackage{graphicx}
\usepackage{ifthen}
\usepackage{booktabs}
\usepackage{multirow}
\usepackage{eurosym}

\usepackage{emerald}
\newcommand{\hw}[1]{{\ECFAugie #1}}


\usepackage{tikz}
\usetikzlibrary{calc,patterns,arrows,trees,
                decorations.pathmorphing,
                decorations.markings,
								decorations.pathreplacing}

\newcommand*\circled[1]{\tikz[baseline=(char.base)]{
            \node[shape=circle,draw,inner sep=2pt,solid,fill=white] (char) {#1};}}


\renewcommand{\familydefault}{\sfdefault}
\usepackage[scaled]{helvet}
%\usepackage{helvet}


%% Arbeitszeit:
% 21.01.16 : 1505 - 1540
% ges = 35 min

\begin{document}
\begin{tikzpicture}[scale=1.2]

\draw[help lines,fill=gray!10] (0,0)rectangle(5.8,6);
\draw[help lines,fill=gray!20] (8,0)rectangle(12,6);

\def\pathI{(0,0) .. controls (1.8,2) and (2.5,3.4) .. (4,3.4)coordinate[pos=.55](XX) .. controls (5.5,3.4) and (7,1) .. (10,0)}
\fill[white, opacity=.8] \pathI;

\draw[thick,fill=white,opacity=.8] (2,0) .. controls (10,.5) and (8.5,6) .. (10,6)coordinate[pos=.55](X) ..controls (12,6) and (11,0) .. (12,0);

\draw[thick] \pathI;

\draw[-latex] (0,0)--++(0,6.5)node[midway,sloped,above]{kostenteilige Fehlerquote $[$\%$]$};
\draw[-latex] (0,0)--++(12.5,0)node[below left]{Phasen im Produktlebenslauf};


\foreach \x/\t in {2/Definition,4/Entwicklung,6/Ablaufplan,8/Fertigung,10/Prüfung,12/Einsatz}{
\draw[help lines] (\x,0)--++(0,6);
\path (\x,6)++(-1,.3)node[]{\t};
}
\foreach \y in {2,4,6}{
\draw[help lines] (0,\y)--++(12,0);
}



\draw[help lines] (XX)--++(-45:.5)node[below right, black, align=left,fill=white]{Entstehung \\ der Fehler};


\draw[help lines] (X)--++(-45:.5)node[below right, black, align=left,fill=white]{Behebung \\ der Fehler};

\draw (.1,7)rectangle(5.7,8)node[midway,align=center]{75 \% der \\ enstandenen Fehlern};


\draw (8.1,7)rectangle(11.9,8)node[midway,align=center]{80 \% der \\ behobenen Fehlern};




\end{tikzpicture}
\end{document}