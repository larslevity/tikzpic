\documentclass[11pt]{standalone}

\usepackage[utf8]{inputenc}
\usepackage[german]{babel}
\usepackage{amsmath}
\usepackage{amsfonts}
\usepackage{amssymb}
\usepackage{graphicx}
\usepackage{ifthen}
\usepackage{array}
\usepackage{eurosym}

% \usepackage{emerald}
% \newcommand{\hw}[1]{{\ECFAugie #1}}


\usepackage{tikz}
\usetikzlibrary{calc,patterns,
                 decorations.pathmorphing,
                 decorations.markings,
                 decorations.pathreplacing}
\usetikzlibrary{trees,arrows}
\usetikzlibrary{shapes.geometric}
\usetikzlibrary{positioning}

                 
\newcommand*\circled[1]{\tikz[baseline=(char.base)]{
            \node[shape=circle,draw,inner sep=2pt,solid,fill=white] (char) {#1};}}
            
            
            
\usetikzlibrary{patterns}

\pgfdeclarepatternformonly[\LineSpace]{my north east lines}{\pgfqpoint{-1pt}{-1pt}}{\pgfqpoint{\LineSpace}{\LineSpace}}{\pgfqpoint{\LineSpace}{\LineSpace}}%
{
    \pgfsetlinewidth{0.4pt}
    \pgfpathmoveto{\pgfqpoint{0pt}{0pt}}
    \pgfpathlineto{\pgfqpoint{\LineSpace + 0.1pt}{\LineSpace + 0.1pt}}
    \pgfusepath{stroke}
}


\pgfdeclarepatternformonly[\LineSpace]{my north west lines}{\pgfqpoint{-1pt}{-1pt}}{\pgfqpoint{\LineSpace}{\LineSpace}}{\pgfqpoint{\LineSpace}{\LineSpace}}%
{
    \pgfsetlinewidth{0.4pt}
    \pgfpathmoveto{\pgfqpoint{0pt}{\LineSpace}}
    \pgfpathlineto{\pgfqpoint{\LineSpace + 0.1pt}{-0.1pt}}
    \pgfusepath{stroke}
}

\newdimen\LineSpace
\tikzset{
    line space/.code={\LineSpace=#1},
    line space=3pt
}


%% Arbeitszeit:
% 21.01.16 : 15 40 - 1610 
% 1945 - 2025
% ges = 70 min

\begin{document}
\begin{tikzpicture}[scale=1.2]



\def\mypath {(12,6)..controls (11,1) and (8,.5) .. (0,0)};

\begin{scope}[xshift=-.3cm,yshift=.3cm]
\draw \mypath;
\path (12,6)coordinate(last);
\foreach[count=\x] \i/\j in {.15/0.05,.51/.3,.68/.6, .8/.75 , .93/.85}{
\path[] \mypath coordinate[pos=\j](end)coordinate[pos=\i](start);
\path[fill=white] (end)rectangle(last);
\path (start)coordinate(\x)coordinate(last);
}
\path[fill=white] (last)rectangle(-.1,-.1);


\foreach[count=\x] \t/\i in {100.-/.14, 10.-/.42, 1.-/.58, ~/.72, 10.-/.9}{
\path[decoration={markings,mark=at position \i with \arrow{latex}},postaction=decorate] \mypath;
\path (\x) node[above left]{\t};
} 

\end{scope}



\path[help lines] (10,4)node(A)[scale=1.6]{Fehlerentdeckung};
\path[help lines] (2,4)node(B)[scale=1.6]{Fehlerverhütung};
\draw[very thick, help lines,-latex] (A)--(B);


\path[help lines] (2,3) node[align=center]{entwickeln \\ und planen};

\path[help lines] (6,3) node[align=center]{beschaffen \\ und herstellen};



\draw[-latex] (0,0)--++(0,6.5)node[midway,sloped,above]{Kosten pro Fehler};
\draw[-latex] (0,0)--++(12.5,0)node[below left]{Phasen im Produktlebenslauf};

\foreach \x/\t in {2/Definition,4/Entwicklung,6/Ablaufplan,8/Fertigung,10/Prüfung,12/Einsatz}{
\path (\x,6)++(-1,0)node[]{\t};
}

\draw[help lines,dotted] (4,-.3)--++(0,6.8);
\draw[help lines,dotted] (8,-.3)--++(0,6.8);

\draw[thick] \mypath;




\end{tikzpicture}
\end{document}