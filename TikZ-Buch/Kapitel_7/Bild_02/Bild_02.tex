\documentclass[11pt]{standalone}

\usepackage[utf8]{inputenc}
\usepackage[german]{babel}
\usepackage{amsmath}
\usepackage{amsfonts}
\usepackage{amssymb}
\usepackage{graphicx}
\usepackage{ifthen}
\usepackage{array}
\usepackage{eurosym}

% \usepackage{emerald}
% \newcommand{\hw}[1]{{\ECFAugie #1}}


\usepackage{tikz}
\usetikzlibrary{calc,patterns,
                 decorations.pathmorphing,
                 decorations.markings,
                 decorations.pathreplacing}
\usetikzlibrary{trees,arrows}
\usetikzlibrary{shapes.geometric}
\usetikzlibrary{positioning}

                 
\newcommand*\circled[1]{\tikz[baseline=(char.base)]{
            \node[shape=circle,draw,inner sep=2pt,solid,fill=white] (char) {#1};}}
            
            
            
\usetikzlibrary{patterns}

\pgfdeclarepatternformonly[\LineSpace]{my north east lines}{\pgfqpoint{-1pt}{-1pt}}{\pgfqpoint{\LineSpace}{\LineSpace}}{\pgfqpoint{\LineSpace}{\LineSpace}}%
{
    \pgfsetlinewidth{0.4pt}
    \pgfpathmoveto{\pgfqpoint{0pt}{0pt}}
    \pgfpathlineto{\pgfqpoint{\LineSpace + 0.1pt}{\LineSpace + 0.1pt}}
    \pgfusepath{stroke}
}


\pgfdeclarepatternformonly[\LineSpace]{my north west lines}{\pgfqpoint{-1pt}{-1pt}}{\pgfqpoint{\LineSpace}{\LineSpace}}{\pgfqpoint{\LineSpace}{\LineSpace}}%
{
    \pgfsetlinewidth{0.4pt}
    \pgfpathmoveto{\pgfqpoint{0pt}{\LineSpace}}
    \pgfpathlineto{\pgfqpoint{\LineSpace + 0.1pt}{-0.1pt}}
    \pgfusepath{stroke}
}

\newdimen\LineSpace
\tikzset{
    line space/.code={\LineSpace=#1},
    line space=3pt
}


%% Arbeitszeit:
% 21.01.16 : 1440-1505
% ges = 25 min

\begin{document}
\begin{tikzpicture}[scale=3.8]

\draw[-latex] (0,0)--++(0,1.7)node[pos=.99,left]{Wert (\%)};
\draw[-latex] (0,0)--++(3.5,0)node[below left]{Festigkeit};

\foreach \y/\t in {.5/50,1/100,1.5/150}{
\draw[help lines] (0,\y)node[left,black]{\t}--++(3.3,0)node[pos=.6,fill=white]{//};
}

\draw (.5,0)node[above left]{\textbf{A}}arc(175:100:1.4)coordinate[pos=.45](A)coordinate[pos=.9](B);
\draw[latex-] (A)--++(0,.25)node[pos=.4,left,align=right]{Untererfüllung \\ hart bestraft};
\draw[latex-] (B)--++(0,-.21)node[pos=.05,above,align=center]{Übererfüllung \\ kärglich belohnt};
\draw[latex-] (1.06,0)--++(0,1.6)node[above,align=center]{Festigkeitsforderung \\ Wäscheleine};


%%
\draw (2,0)node[above left]{\textbf{B}}arc(175:100:1.4)coordinate[pos=.45](A)coordinate[pos=.9](B);
\draw[latex-] (1.06+1.5,0)--++(0,1.6)node[above,align=center]{Festigkeitsforderung \\ Bergseil};



\end{tikzpicture}
\end{document}